\section{Äquiaffine Geometrie}
\subsection{Definition}
	\begin{Definition}[Parallelotop im affinen Raum]
		Seien $ (A,V,\tau) $ ein $ n $-dimensionaler affiner Raum und $ p_0\in A $; das von einer Familie $ (v_j)_{j\in \{1,\dots,n\}} $ in $ V $ aufgespannte \emph{Parallelotop} oder \emph{Spat} ist die (über dem abstrakten Würfel $ \mathbb{Z}_2^n $ indizierte) Familie
			\[ p: \mathbb{Z}_2^n\to A, \epsilon \mapsto p_\epsilon := p_0 +\sum_{j=1}^{n}v_j\epsilon_j. \]
		Ist $ \omega\in\Lambda^nV^*\setminus\{0\} $, so ist das zugehörige \emph{(Spat-)Volumen} von $ (p_\epsilon)_{\epsilon \in \mathbb{Z}_2^n} $
			\[ \vol(p):= \omega(v_1,\dots,v_n). \]
	\end{Definition}
%%% Grafik Parallelogramm
\begin{figure}[H]
	\centering
\definecolor{zzttqq}{rgb}{0.6,0.2,0.}
\definecolor{uuuuuu}{rgb}{0.26,0.26,0.26}
\definecolor{qqqqff}{rgb}{0.,0.,1.}
\begin{tikzpicture}[line cap=round,line join=round,>=triangle 45,x=5.0cm,y=3.0cm,]
\clip(0.6,0.7) rectangle (2.8,2.35);
\fill[color=zzttqq,fill=zzttqq,fill opacity=0.1] (1.,1.) -- (1.5,2.) -- (2.5,2.) -- (2.,1.) -- cycle;
\draw [->] (1.,1.) -- (1.5,2.);
\draw [->] (1.,1.) -- (2.,1.);

\draw [fill=qqqqff] (1.,1.) circle (2.5pt);
\draw[color=qqqqff] (0.96,0.94) node {p(0,0)};
\draw [fill=uuuuuu] (2.5,2.) circle (1.5pt);
\draw[color=uuuuuu] (2.6,2.0) node {p(1,1)};
\draw [fill=uuuuuu] (2.,1.) circle (1.5pt);
\draw[color=uuuuuu] (2.,1) node {p(1,0)};
\draw [fill=uuuuuu] (1.5,2.) circle (1.5pt);
\draw[color=uuuuuu] (1.58,2.05) node {p(0,1)};
\draw[color=black] (1.2,1.55) node {$v_2$};
\draw[color=black] (1.5,1) node {$v_1$};

\end{tikzpicture}
\end{figure}
%%%%% ENDE Grafik Parallelogramm %%%%%
\subsection{Bemerkung \& Definition}
	\begin{Definition}[Parallelogramm/Flächeninhalt]
		Im Falle $ n=2 $ heißt ein Parallelotop $ p $ auch \emph{Parallelogramm}, sein Spatvolumen auch sein \emph{Flächeninhalt}. Der Flächeninhalt ist orientiert:
		Mit
			\[ \epsilon = (\epsilon_j)_{j\in \{1,2\}}\cong (\epsilon_1,\epsilon_2) \]
		und
			\[ v_1 = p_{(1,0)}-p_{(0,0)}\text{ und } v_2 = p_{(0,1)}-p_{(0,0)} \]
		ändert der Flächeninhalt das Vorzeichen, wenn man die Kantenvektoren vertauscht:
			\[ \vol(p) = \omega(v_1,v_2) = -\omega(v_2,v_1) = \vol(p') \]
		mit $ p'_\epsilon = p_\epsilon \circ \tau_{12} $.
	\end{Definition}
\paragraph{Bemerkung}
	Für ein Dreieck $ \{a,b,c\} $ wählt man eine Orientierung, e.g. $ (a,b,c) $, und setzt den Flächeninhalt
		\[ F(a,b,c) := \omega(b-a,c-a)\cdot\frac{1}{2} \]
	d.h. als halben Flächeninhalt des Parallelogramms
		\[ p(0,0) = 0, p(1,0) = b, p(0,1) = c \text{ und } p(1,1) = a(-1)+b\cdot 1+c\cdot 1. \]
	Dieser Flächeninhalt ist wohldefiniert, d.h. er hängt nur vom Dreieck und der gewählten Orientierung ab -- insbesondere ist für Permutationen $ \sigma $ von $ \{a,b,c\} $ mit $ \sgn(\sigma) = +1 $ der Flächeninhalt gleich dem ursprünglichen. (Siehe Aufgabe)
\paragraph{Bemerkung}
	Ist $ A $ eine affine Ebene mit Flächenmessung $ \vol $ und $ \{a,b,c\} $ ein nicht-degeneriertes Dreieck, also ein baryzentrisches Bezugssystem, so gilt (Cramersche Regel)
		\[ \forall s\in A: s=a\cdot \frac{F(s,b,c)}{F(a,b,c)}+ b\cdot\frac{F(a,s,c)}{F(a,b,c)} +c\cdot\frac{F(a,b,s)}{F(a,b,c)}\]
	Diese Flächenformel für die baryzentrischen Koordinaten eines Punktes $ s $ ist unabhängig von der (gewählten) Flächenmessung, das sich verschiedene Flächenmessungen nur um einen Faktor unterscheiden (der unabhängig vom Dreieck ist): $ \dim \Lambda^2V^*=1 $
%%% Grafik Dreieck mit Punkt s %%%%%
\begin{figure}[H]\centering
	\definecolor{wwqqcc}{rgb}{0.4,0.,0.8}
	\definecolor{ffzztt}{rgb}{1.,0.6,0.2}
	\definecolor{qqqqff}{rgb}{0.,0.,1.}
	\begin{tikzpicture}[line cap=round,line join=round,>=triangle 45,x=2.0cm,y=1.5cm]
	\fill[color=ffzztt,fill=ffzztt,fill opacity=0.1] (1.,1.) -- (2.64,3.96) -- (3.92,0.42) -- cycle;
	\fill[color=wwqqcc,fill=wwqqcc,fill opacity=0.1] (2.64,3.96) -- (2.7,1.6) -- (3.92,0.42) -- cycle;
	\draw [color=wwqqcc] (2.64,3.96)-- (2.7,1.6);
	\draw [color=wwqqcc] (2.7,1.6)-- (3.92,0.42);
	\draw [color=wwqqcc] (3.92,0.42)-- (2.64,3.96);
	\draw (1.,1.)-- (2.7,1.6);
	\draw [fill=qqqqff] (1.,1.) circle (2.5pt);
	\draw[color=qqqqff] (0.8,1) node {$a$};
	\draw [fill=qqqqff] (2.64,3.96) circle (2.5pt);
	\draw[color=qqqqff] (2.7,4.2) node {$b$};
	\draw [fill=qqqqff] (3.92,0.42) circle (2.5pt);
	\draw[color=qqqqff] (4.2,0.4) node {$c$};
	\draw [fill=qqqqff] (2.7,1.6) circle (2.5pt);
	\draw[color=qqqqff] (2.8,1.84) node {$s$};
	\end{tikzpicture}
\end{figure}
%%%% Ende Grafik Dreieck mit Punkt s %%%%%
\subsection{Definition}
	\begin{Definition}[Äquiaffine Transformation]
		Eine affine Abbildung $ \alpha:A\to A' $ zwischen AR $ A $ und $ A' $ mit Volumenmessungen $ \vol $ und $ \vol' $ heißt \emph{volumentreu}, falls für alle Parallelotope $ p $ in $ A $ gilt
			\[ \vol'(\alpha\circ p)=\vol(p) \]
		Eine \emph{äquiaffine Transformation} ist eine volumentreue Affinität.
	\end{Definition}
\paragraph{Bemerkung}
	Ist $ \alpha:A\to A' $ affin mit linearem Anteil $ \lambda:V \to V' $ und $ p $ ein von einer Familie $ (v_j)_{j\in\{1,\dots,n\}} $ von Vektoren aufgespanntes Parallelotop in $ A $, so ist
	 \begin{align*}
	 p':= \alpha\circ p:\mathbb{Z}_2^n\to A', \epsilon \mapsto p'_\epsilon &= \alpha(p_0)+\lambda(\sum_{j=1}^{n}v_j\epsilon_j)\\
	 &= \alpha(p_0)+\sum_{j=1}^{n}\lambda(v_j)\epsilon_j
	 \end{align*}
	ein von der Familie $ (\lambda(v))_{j\in \{1,\dots,n\}} $ von Vektoren in $ V' $ aufgespanntes Parallelotop in $ A' $. Damit ist $ \vol'(\alpha\circ p) $ ein sinnvoller Ausdruck, also der Begriff "`volumentreu"' sinnvoll für affine Abbildungen.
\paragraph{Bemerkung}
	Offenbar bilden die volumentreuen Affinitäten eine Gruppe: eine Untergruppe der affinen Gruppe (nach Untergruppenkriterium).
\subsection{Äquiaffine Geometrie}
	\begin{Definition}[Äquiaffine Geometrie]
	Ist $ (A,V,\tau,\vol) $ ein mit einem Spatvolumen versehener Affiner Raum, so bestimmt die auf $ A $ operierende Gruppe der volumentreuen Affinitäten (der äquiaffinen Transformationen) eine \emph{äquiaffine Geometrie}.
	\end{Definition}
\subsection{Bemerkung \& Definition}
	\begin{Definition}[Volumenverzerrung]
		Ist $ \alpha: A\to A $ Affinität eines AR $ A $ mit Volumenmessung $ \vol $, so gibt es genau eine Zahl $ \delta(\alpha)\in K^x $, sodass für jedes Parallelotop $ p $ in $ A $ gilt
			\[ \vol(\alpha\circ p) = \delta(\alpha)\vol(p), \]
		wobei $ \delta(\alpha) $ die \emph{Volumenverzerrung} genannt wird.
		
		Die Volumenverzerrung $ \delta(\alpha) $ hängt nur vom linearen Anteil $ \lambda\in \End(V) $ ab: für ein von einer Basis $ (v_j)_{j\in \{1,\dots,n\}} $ von $ V $ aufgespanntes Parallelotop ist
			\[ \delta(\alpha) = \frac{\omega(\lambda(v_1),\dots,\lambda(v_n))}{\omega(v_1,\dots,v_n)} \]
	\end{Definition}
\subsection{Definition}
	\begin{Definition}[Determinante eines Endomorphismus]
		Seien $ f\in\End(V) $, wobei $ \dim V=n $, und $ \omega\in \Lambda^nV^*\setminus \{0\} $. Dann heißt
		\[ \det f:= \frac{f^*\omega}{\omega} \]
	die Determinante von $ f $, wobei
		\[ f^*\omega: V^n\to K, (v_1,\dots,v_n)\mapsto \omega(f(v_1),\dots,f(v_n)). \]
	\end{Definition}
\paragraph{Bemerkung}
	Offenbar ist $ f^*\omega\in \Lambda^nV^* = [\omega] $ wegen $ \dim Lambda^nV*=1 $, d.h.
		\[ \exists! x\in K: f^*\omega = \omega \cdot x \text{ ($(\omega) $ ist Basis von $ \Lambda^nV^* $)} \]
	dieses $ x $ ist die Determinante von $ f $, also $ \det f = x $.
	
	Alternativ: Die Abbildung
		\[ B\mapsto  \frac{f^*\omega(B)}{\omega(B)}\in K\]
	ist konstant $ \equiv x $, unabhängig von der Basis $ B $.
\paragraph{Bemerkung}
	Da $ f^*(\omega x) = (f^*\omega)x $ für $ x\in K $, liefert jedes $ \omega \in Lambda^nV^*\setminus\{0\} $ die gleiche Determinante $ \det f $: die Determinante bzw. Volumenverzerrung ist unabhängig von der gewählten Volumenform $ \omega\in \Lambda^nV^*\setminus \{0\} $, d.h. der "`Referenzvolumenmessung"'.
\subsection{Determinantenmultiplikationssatz}
	\begin{Satz}[Determinantenmultiplikationssatz]
		Für $ f,g\in \End(V) $ gilt
			\[ \det(f\circ g) = \det f \cdot \det g. \]
	\end{Satz}
\paragraph{Beweis}
	Mit $ \omega\in\Lambda^nV^*\setminus\{0\} $ berechnet man
		\[ (f\circ g)^*\omega = g^*(f^*\omega) = g^*(\omega\cdot \det f) = (g^*\omega)\det f = \omega \det g\cdot \det f \]