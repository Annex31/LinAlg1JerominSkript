\chapter{Volumenmessung}
	Grundlegende Idee: Wir definieren ein Spat- oder Parallelotop-Volumen.

	Algebraisch: Dieses Volumen kann dann benutzt werden, um zu testen, wann ein Spat/Parallelotop "`zusammenklappt"'.
\section{Determinantenformen}
	Idee: Für den Flächeninhalt $ F(v,w) $ eines von zwei Vektoren $ v,w\in V $ aufgespannten Parallelogramms gilt
	\begin{gather*}
		F{(vx,w)} = F{(v,w)}x\\
		F(v+v',w) = F(v,w)+F(v',w)
	\end{gather*}
	und entsprechend für das zweite Argument; außerdem verschwindet der Flächeninhalt, wenn das Parallelogramm "`zusammenklappt"', also insbesondere gilt
		\[ w=v\Rightarrow F(v,w)=0 \]
	Die folgende Definition verallgemeinert diese Eigenschaften:

\definecolor{ttttff}{rgb}{0.2,0.2,1}
\definecolor{ttfftt}{rgb}{0.2,1,0.2}
\definecolor{uququq}{rgb}{0.25,0.25,0.25}
\definecolor{qqqqff}{rgb}{0,0,1}
\begin{tikzpicture}[line cap=round,line join=round,>=triangle 45,scale=1.5]
\draw[->,color=black] (-1.1,0) -- (7.03,0);
\draw[->,color=black] (0,-0.34) -- (0,4.89);

\clip(-1.1,-0.34) rectangle (7.03,4.89);
\fill[color=ttfftt,fill=ttfftt,fill opacity=0.5] (2,2) -- (1.33,3.75) -- (2.83,3.21) -- (3.51,1.46) -- cycle;
\fill[color=ttttff,fill=ttttff,fill opacity=0.1] (2,2) -- (1.33,3.75) -- (3.59,2.94) -- (4.26,1.19) -- cycle;
\draw [->] (2,2) -- (3.51,1.46);
\draw [->] (2,2) -- (1.33,3.75);
\draw [->] (3.51,1.46) -- (2.83,3.21);
\draw [->] (1.33,3.75) -- (2.83,3.21);
\draw [->] (2,2) -- (4.26,1.19);
\draw [->] (1.33,3.75) -- (3.59,2.94);
\draw [->] (4.26,1.19) -- (3.59,2.94);
\draw [color=ttfftt] (2,2)-- (1.33,3.75);
\draw [color=ttfftt] (1.33,3.75)-- (2.83,3.21);
\draw [color=ttfftt] (2.83,3.21)-- (3.51,1.46);
\draw [color=ttfftt] (3.51,1.46)-- (2,2);
\draw [color=ttttff] (2,2)-- (1.33,3.75);
\draw [color=ttttff] (1.33,3.75)-- (3.59,2.94);
\draw [color=ttttff] (3.59,2.94)-- (4.26,1.19);
\draw [color=ttttff] (4.26,1.19)-- (2,2);
\begin{scriptsize}
\fill [color=qqqqff] (2,2) circle (1.5pt);
\fill [color=qqqqff] (1.33,3.75) circle (1.5pt);
\fill [color=qqqqff] (3.51,1.46) circle (1.5pt);
\draw[color=black] (2.79,1.84) node {$v$};
\draw[color=black] (1.8,2.91) node {$w$};
\fill [color=uququq] (2.83,3.21) circle (1.5pt);
\fill [color=uququq] (4.26,1.19) circle (1.5pt);
\fill [color=uququq] (3.59,2.94) circle (1.5pt);
\draw[color=ttttff] (3.65,1.31) node {$vx$};
\end{scriptsize}
\end{tikzpicture}

%-------------------Begin Addition mit einem Vektor ----------------  
\begin{figure}[H]\centering
\tdplotsetmaincoords{0}{0} %-27
\begin{tikzpicture}[yscale=1,tdplot_main_coords]

\def\xstart{0} %x Koordinate der Startposition der Grafik
\def\ystart{0} %y Koordinate der Startposition der Grafik
\def\myscale{0.9} %ändert die Größe der Grafik (Skalierung der Grafik) 

\def\xstartdraw{(\xstart + 1.5)} %xKoordinate des Referenzstartpunktes (in dieser Zeichnung: a)
\def\ystartdraw{(\ystart + 3.5)}%yKoordinate des Referenzstartpunktes (in dieser Zeichnung: a)

\def\balkenhoehe{(5.3)}% Länge des vertikalen blauen Balkens
\def\balkenlaenge{(10)}% Länge des horizontalen blauen Balkens
\def\balkenbreite{0.4} %Balkenbreite

%---------Begin Balken----------
\def\drehwinkel{0}
\node (VekV) at ({\xstart+0.7*cos(\drehwinkel)-\balkenbreite*sin(\drehwinkel)},{\ystart+0.5*sin(\drehwinkel)+\balkenbreite*cos(\drehwinkel)})[right, xshift=1,color=blue] {$V=\mathbb{R}^2$};
\node (AffA) at ({\xstart+(\balkenlaenge-1)*cos(\drehwinkel)},{\ystart+(\balkenlaenge-1)*sin(\drehwinkel)+\balkenbreite*cos(\drehwinkel)})[color=red] {$A$};

\path[ shade, top color=white, bottom color=blue, opacity=.6] 
    ({\xstart},{\ystart},0)  -- ({\xstart - \balkenbreite * cos(\drehwinkel)- (-\balkenbreite+0)*sin(\drehwinkel)},{\ystart - \balkenbreite * sin(\drehwinkel)+ (-\balkenbreite+0)*cos(\drehwinkel)},0)  -- ({\xstart - \balkenbreite * cos(\drehwinkel)- (\balkenhoehe+0.5)*sin(\drehwinkel)},{\ystart - \balkenbreite * sin(\drehwinkel)+ (\balkenhoehe+0.5)*cos(\drehwinkel)},0) -- ({\xstart - 0 * cos(\drehwinkel)- (\balkenhoehe+0)*sin(\drehwinkel)},{\ystart - 0 * sin(\drehwinkel)+ (\balkenhoehe+0)*cos(\drehwinkel)},0) -- cycle;
        
\path[ shade, right color=white, left color=blue, opacity=.6] 
	({\xstart},{\ystart},0)  -- ({\xstart - \balkenbreite * cos(\drehwinkel)- (-\balkenbreite+0)*sin(\drehwinkel)},{\ystart - \balkenbreite * sin(\drehwinkel)+ (-\balkenbreite+0)*cos(\drehwinkel)},0) --
    ({\xstart + (\balkenlaenge+0.5) * cos(\drehwinkel)- (-\balkenbreite+0)*sin(\drehwinkel)},{\ystart + (\balkenlaenge+0.5) * sin(\drehwinkel)+ (-\balkenbreite+0)*cos(\drehwinkel)},0) --   
    ({\xstart + \balkenlaenge * cos(\drehwinkel)},{\ystart + \balkenlaenge * sin(\drehwinkel)},0)--
    cycle;       
%---------End Balken----------
\def\lightoffset{0.2*\myscale} %offeset der Vektoren

%Punkte Definition
\node (pointa1) at ({\xstartdraw},{\ystartdraw}) {};
\node (pointa2) at ({\xstartdraw+(1 *\myscale)},{\ystartdraw-(2.0*\myscale)}) {};
\node (pointb1) at ($(pointa1) + (3.0*\myscale,-1.0*\myscale) $) {};
\node (pointb2) at ($(pointb1) + (1.0*\myscale,-2.0*\myscale) $) {};

\node (pointc1) at ($(pointa1) + (6.5*\myscale,1.3*\myscale) $) {};
\node (pointc2) at ($(pointa2) + (6.5*\myscale,1.3*\myscale) $) {};

\node (pointFvwi) at ($(pointb2) + (-0.9*\myscale,0.9*\myscale) $) {};
\node (pointFvwa) at ($(pointb2) + (-1.5*\myscale,-0.3*\myscale) $) {};

\node (pointFvswi) at ($(pointc2) + (-2.9*\myscale,-1.2*\myscale) $) {};
\node (pointFvswa) at ($(pointc2) + (-1.3*\myscale,-1.9*\myscale) $) {};

\node (pointfgi) at ($(pointa1) + (2.2*\myscale,-0.2*\myscale) $) {};
\node (pointfga) at ($(pointfgi) + (-1.4*\myscale,1.8*\myscale) $) {};



%Flächen füllen
%blaue Flaeche
\fill[color=ttttff,fill=ttttff,fill opacity=0.15] (pointa1.center) -- (pointb1.center) -- (pointc1.center) -- (pointc2.center) -- (pointb2.center)-- (pointa2.center)-- cycle;
%gruene Flaeche
\fill[color=ttfftt,fill=ttfftt,fill opacity=0.5] (pointa1.center) -- (pointc1.center) -- (pointc2.center) -- (pointa2.center)-- cycle;

%Vektoren blau
\draw[-{>[scale=1,length=10,width=6]},shorten >=2pt, shorten <=2pt,line width=0.2pt,color=blue] (pointa1) -- (pointb1);
\draw[-{>[scale=1,length=10,width=6]},shorten >=2pt, shorten <=2pt,line width=0.2pt,color=blue] (pointa2) -- (pointb2);
\node [color=blue] (pointlabelg1) at ($(pointa1)!0.5!(pointb1)$) [above, xshift=0, yshift=0] {$v$} ;
\node [color=blue] (pointlabelg2) at ($(pointa2)!0.5!(pointb2)$) [above, xshift=0, yshift=0] {$v$} ;

\draw[-{>[scale=1,length=10,width=6]},shorten >=2pt, shorten <=2pt,line width=0.2pt,color=blue] (pointb1) -- (pointc1);
\draw[-{>[scale=1,length=10,width=6]},shorten >=2pt, shorten <=2pt,line width=0.2pt,color=blue] (pointb2) -- (pointc2);
\node [color=blue] (pointlabelg3) at ($(pointb1)!0.5!(pointc1)$) [above, xshift=0, yshift=0] {$v'$} ;
\node [color=blue] (pointlabelg4) at ($(pointb2)!0.5!(pointc2)$) [above, xshift=0, yshift=0] {$v'$} ;

\draw[-{>[scale=1,length=10,width=6]},shorten >=2pt, shorten <=2pt,line width=0.2pt,color=blue] (pointa2) -- (pointa1);
\draw[-{>[scale=1,length=10,width=6]},shorten >=2pt, shorten <=2pt,line width=0.2pt,color=blue] (pointb2) -- (pointb1);
\draw[-{>[scale=1,length=10,width=6]},shorten >=2pt, shorten <=2pt,line width=0.2pt,color=blue] (pointc2) -- (pointc1);

\node [color=blue] (pointlabelga2a1) at ($(pointa2)!0.5!(pointa1)$) [left, xshift=0, yshift=0] {$w$} ;
\node [color=blue] (pointlabelgb2b1) at ($(pointb2)!0.5!(pointb1)$) [right, xshift=0, yshift=0] {$w$} ;
\node [color=blue] (pointlabelgc2c1) at ($(pointc2)!0.5!(pointc1)$) [right, xshift=0, yshift=0] {$w$} ;

%Vektoren gruen
\draw[-{>[scale=1,length=10,width=6]},shorten >=4pt, shorten <=4pt,line width=0.2pt,color=green] (pointa1) -- (pointc1);
\draw[-{>[scale=1,length=10,width=6]},shorten >=4pt, shorten <=4pt,line width=0.2pt,color=green] (pointa2) -- (pointc2);
\node [color=green] (pointlabelac1) at ($(pointa1)!0.5!(pointc1)$) [above, xshift=0, yshift=0] {$v+v'$} ;
\node [color=green] (pointlabelac2) at ($(pointa2)!0.5!(pointc2)$) [above, xshift=0.5, yshift=0] {$v+v'$} ;

%Punkte malen
\draw[fill,color=red] (pointa1) circle [x=1cm,y=1cm,radius=0.08]node[above, xshift=0, yshift=0]{};
\draw[fill,color=red] (pointb1) circle [x=1cm,y=1cm,radius=0.08]node[above, xshift=0, yshift=0]{};
\draw[fill,color=red] (pointa2) circle [x=1cm,y=1cm,radius=0.08]node[below, xshift=5, yshift=0]{};
\draw[fill,color=red] (pointb2) circle [x=1cm,y=1cm,radius=0.08]node[below, xshift=5, yshift=0]{};
\draw[fill,color=red] (pointc1) circle [x=1cm,y=1cm,radius=0.08]node[below, xshift=5, yshift=0]{};
\draw[fill,color=red] (pointc2) circle [x=1cm,y=1cm,radius=0.08]node[below, xshift=5, yshift=0]{};

\draw[->,shorten >=2pt, shorten <=2pt,line width=0.2pt,color=blue] (pointFvwa) -- (pointFvwi);
\draw[->,shorten >=2pt, shorten <=2pt,line width=0.2pt,color=blue] (pointFvswa) -- (pointFvswi);
\draw[->,shorten >=2pt, shorten <=2pt,line width=0.2pt,color=green] (pointfga) -- (pointfgi);

\node [color=blue] (pointlabelFvwl) at (pointFvwa) [xshift=0.5, yshift=-0.5] {$F(v,w)$} ;
\node [color=blue] (pointlabelFvswl) at (pointFvswa) [xshift=-0.5,  yshift=-5] {$F(v',w)$} ;
\node [color=green] (pointlabelFvswl) at (pointfga) [xshift=35 ] {$F(v+v',w)=\textcolor{blue}{F(v,w)+F(v',w)}$} ;

\end{tikzpicture}
\end{figure}
%-------------------End Addition mit einem Vektor ----------------

\subsection{Definition}
	\begin{Definition}[Linearform/Determinantenform]
	Sei $ V $ ein $ K $-VR. Eine Abbildung $ \omega:V^m\to K $ heißt
		\begin{itemize}
		\item \emph{$ m $-linear}, bzw. eine \emph{$ m $-(Linear-)Form}, falls $ \omega $ in jedem Argument linear ist, d.h.
			\[ \forall i=1,\dots, m: V\ni v_i\mapsto \omega(v_1,\dots,v_{i-1},v_i,v_{i+1},\dots,v_m)\in K \]
			ist linear;
		\item \emph{alternierend}, falls $ \omega(v_1,\dots,v_m)=0 $ wann immer zwei Vektoren gleich sind, d.h.
			\[ v_i = v_j \text{ für } i\neq j \Rightarrow \omega(v_1,\dots, v_m) = 0. \] 
		\end{itemize}
	Die Menge der alternierenden $ m $-Formen wird mit $ \Lambda^mV^* $ bezeichnet. Ist $ \dim V = n $, so heißt ein $ \omega\in \Lambda^nV^* $ auch Determinantenform.
	\end{Definition}

\paragraph{Beispiel}
	Jede Linearform $ \omega\in V^* $ ist eine (alternierende) 1-Form, $ \Lambda^1V^*=V^* $.
\paragraph{Bemerkung}
	$ \Lambda^mV^* $ ist für jedes $ m\in \mathbb{N} $ selbst ein $ K $-VR. 
\subsection{Lemma}
	\begin{Lemma}
		Für eine alternierende $ m $-Form $ \omega \in \Lambda^mV^* $ und $ i\neq j $ gilt:
		\begin{enumerate}[(i)]
			\item $ \omega(\dots,v_i,\dots,v_j,\dots) = -\omega (\dots,v_j,\dots,v_i,\dots)$;
			\item $ \omega(\dots,v_i,\dots,v_is+v_j,\dots) = \omega(\dots,v_i,\dots,v_j,\dots) $ für $ s\in K $;
			\item und $ \omega(v_1,\dots,v_m)=0 $, falls $ (v_i)_{i\in \{1,\dots,m\}} $ linear abhängig ist.
		\end{enumerate}
\paragraph{Beweis}
	Seien $ v_1,\dots,v_m\in V $ und $ i,j\in \{1,\dots,m\} $ mit $ i\neq j $. Dann gilt:
		\begin{align*}
			0 &= \omega(\dots,v_i+v_j,\dots,v_i+v_j,\dots)\\
			&= \omega(\dots,v_i,\dots,v_i,\dots)+\omega(\dots,v_j,\dots,v_j)+\omega(\dots,v_i,\dots,v_j,\dots)+\omega(\dots,v_j,\dots,v_i,\dots)\\
			&= \omega(\dots,v_i,\dots,v_j,\dots)+\omega(\dots,v_j,\dots,v_i,\dots)
		\end{align*}
	und
		\begin{align*}
		0&=\omega(\dots,v_i,\dots,v_is,\dots)\\
		&= \omega(\dots,v_i,\dots,v_is+v_j-v_j,\dots)\\
		&= \omega(\dots,v_i,\dots,v_is+v_j,\dots)-\omega(\dots,v_i,\dots,v_j,\dots)
		\end{align*}
	Dies beweist (i) und (ii). Ist die Familie $ (v_i)_{i\in \{1,\dots,m\}} $ linear abhängig, o.B.d.A
		\[ v_m = \sum_{i=1}^{m-1}v_ix_i \in [(v_i)_{i\in \{1,\dots,m-1\}}] \]
	so gilt
		\begin{gather*}
		\omega(v_1,\dots,v_m)=\omega(v_1,\dots,v_{m-1},\sum_{i=1}^{m-1}v_ix_i)\\
		= \sum_{i=1}^{m-1}\omega(v_1,\dots,v_{m-1},v_i)x_i (=0)
		\end{gather*}
	womit (iii) bewiesen ist.
	\end{Lemma}
\paragraph{Bemerkung}
	(i) liefert eine äquivalente Formulierung von "`alternierend"' für $ m $-Linearformen, wenn $ \Char (K)\neq 2 $.
	Nämlich: sind $ v_1,\dots,v_m\in V $ mit $ v_i=v_j $ für $ i\neq j $, so gilt
		\begin{align*}
		0 &= \omega(\dots,v_i\dots,v_j)+\omega(\dots,v_j,\dots,v_i,\dots)\\
		&= 2\omega(\dots,v_i,\dots,v_j,\dots) \Rightarrow 0 = \omega(\dots,v_i,\dots,v_j,\dots)
		\end{align*}
\paragraph{Buchhaltung}
	Benutzt man (vgl. Gausssches Eliminationsverfahren) die Elementarmatrizen
	\begin{gather*}
		D_i = (d_{kl}) \in Gl(m); d_{kl} = \delta_{kl}+(d-1)\delta_{ik}\delta_{il}\quad (d\in K^\times);\\
		T_{ij} = (t_{kl}) \in Gl(m); t_{kl} = \delta_{kl}-(\delta_{ik}-\delta_{jk})(\delta_{il}-\delta_{jl});\\
		S_{ij} = (s_{kl})\in Gl(m); s_{kl}=\delta_{kl}+s\delta_{ik}\delta_{jl}\quad (s\in K)
	\end{gather*}
	und beschreibt man eine Familie $ (v_i)_{i\in \{1,\dots,m\}} $ von Vektoren $ v_i\in V $ durch ein $ m $-Tupel $ A=(v_1,\dots,v_m) $ von Werten der Familie, so lassen sich die Homogenität und Eigenschaften (i) und (ii) des Lemmas einfach schreiben als
		\[ \omega(AD_i) = \omega(A)d,\ \omega(AT_{ij}) = -\omega(A),\ \omega(AS_{ij}(s)) = \omega(A) \]
\subsection{Wiederholung \& Definition}
	Die bijektiven Abbildungen (Permutationen)
		\[ \sigma: I\to I, i\mapsto \sigma(i),\text{ der Menge }I = \{1,\dots,m\} \]
	bilden eine Gruppe (mit der Komposition als Verknüpfung), die Permutationsgruppe $ S_m $ der Menge $ I $.
	\begin{Definition}[Transposition]
	Eine Transposition $ \tau_{ij} \in S_m, i\neq j $ ist eine Permutation, die zwei Indizes vertauscht,
		\[ \tau_{ij}:I\to I, k\mapsto \tau_{ij}(k):=
		\begin{cases}
			j, &\text{falls } k=i,\\
			i, &\text{falls } k=j,\\
			k & \text{sonst}.
		\end{cases} \]
	Jede Permutation ist eine Komposition von Transpositionen, wie man leicht durch Induktion über m zeigt:
	
	Ist $ \sigma(m) = i<m$, so ist $ \tau_{im}\circ \sigma $ eine Permutation, die $ m $ fixiert, also
		\[ \tau_{im}\circ\sigma\mid_{\{1,\dots,m-1\}}\in S_{m-1} \]
	\end{Definition}
\paragraph{Bemerkung}
	Die Eigenschaft (i) des Lemmas, $ \omega(AT_{ij})=-\omega(A) $, lässt sich mit $ \tau_{ij} $ dann formulieren als
		\[ \omega(v_{\tau_{ij}(1)}, \dots, v_{\tau_{ij}(m)}) = - \omega(v_1,\dots,v_m) \]
	Da jede Permutation $ \sigma\in S_m $ Komposition von Transpositionen ist, folgt
		\[ \forall \sigma\in S_m: \omega(v_{sigma}(1),\dots,v_{sigma(m)})=\pm \omega(v_1,\dots,v_{m}). \]
	Frage: Was ist das Vorzeichen bzw. wie kann man es berechnen?
\subsection{Lemma \& Definition}
	\begin{Definition}[Signum einer Permutation]
	Das Signum einer Permutation $ \sigma\in S_m $ ist die Zahl
		\[ \operatorname{sgn}\sigma := \prod_{i<j} \frac{\sigma(i)-\sigma(j)}{i-j}\in \{\pm 1\}; \]
	ist $ \operatorname{sgn}\sigma = 1 $, so heißt $ \sigma $ gerade, sonst ungerade. Signum liefert einen Gruppenhomomorphismus
		\[ \operatorname{sgn}: S_m\to (\{\pm 1\},\cdot). \]
	\end{Definition}
\paragraph{Beispiel}
	Eine Transposition $ \tau_{ij} $ ist eine ungerade Permutation, da
		\[ \operatorname{sgn}\tau_{ij} = \prod_{k<l}\frac{\tau_{ij}(k)-\tau_{ij}(l)}{k-l} = \frac{j-i}{i-j}\prod_{k\neq i,j}\frac{i-k}{j-k}\frac{j-k}{i-k} = -1 \]