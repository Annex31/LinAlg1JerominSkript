% % % %Kapitel 1 - Lineare Räume und Abbildungen % % % %
\chapter{Lineare Räume und Abbildungen}
\section{Von Geometrie zu Algebra}
	Euklids führte in den \glqq Elementen\grqq{} (ca. 300 v. Chr.) das bis heute gültige Schema ein:
	\begin{itemize}
		\item Definition
		\item Axiom/Postulat
		\item Lehrsatz
		\item Beweis
	\end{itemize}

\paragraph{Parallelenaxiom/-problem (Euklid, Formulierung nach Playfair):}
	Es existiert genau eine Parallele $ g' $ zum Punkt $ P \notin g $ zur Geraden $ g $.

	Kann das Axiom aus den anderen Axiomen hergeleitet/bewiesen werden? Nein, denn es existieren nichteuklidische, hyperbolische Geometrien (18. Jh.) in denen es mehrere derartige Parallelen gibt. Als Beispiel lässt sich eine Geometrie anführen, die nicht auf einer Ebene sondern auf einem Kreis operiert. Dort lassen sich zu einer Sekante mehrere parallele Sekanten betrachten (also Sekanten, die die ursprüngliche nicht schneiden).

	\begin{figure}[H]
		\begin{minipage}{.45\textwidth}
			\begin{tikzpicture}[line cap=round,line join=round,>=triangle 45,x=1.0cm,y=1.0cm]
				\clip(-1.69,-0.64) rectangle (4.14,2.83);
				\draw [domain=-1.69:4.14] plot(\x,{(-1--1*\x)/1});
				\draw [domain=-1.69:4.14] plot(\x,{(-0--1*\x)/1});
				\draw (0.6,1) node[] {P};
				\draw (1.58,0.16) node[] {g};
				\draw (1.52,1.78) node[] {g'};
				\begin{scriptsize}
				\fill [color=blue] (1,1) circle (2pt);
				\end{scriptsize}
			\end{tikzpicture}
		\end{minipage}
		\begin{minipage}{.45\textwidth}
			\begin{tikzpicture}[line cap=round,line join=round,>=triangle 45,x=1.0cm,y=1.0cm]
				\clip(-2.24,-3.38) rectangle (3.15,1.76);
				\draw(0,0) circle (1cm);
				\draw (-0.94,0.35)-- (0.66,0.75);
				\draw (-0.13,0.74) node[] {g};
				\draw (0.32,-0.56) node[] {P};
				\draw (-1,-0.02)-- (0.88,-0.48);
				\draw (-0.35,-0.94)-- (0.91,0.42);
				\begin{scriptsize}
				\fill [color=blue] (0.22,-0.32) circle (1.5pt);
				\end{scriptsize}
			\end{tikzpicture}
		\end{minipage}
	\end{figure}

\paragraph{Was ist eine Geometrie?}
	Eine Geometrie ist durch eine Menge X und eine auf X operierende Transformationsgruppe gegeben.
%%%%%%%%%%%%%%%%% BEGINN VO3-20151013 %%%%%%%%%%%%%%%%%%%%%

\paragraph{Definition:}
	Ein Paar $(G,\circ)$ bestehend aus einer Menge $G$ und einer Verknüpfung $(\circ : G\times G \to G) : (g,h) \mapsto g \circ h$ heißt Gruppe, falls:

	\begin{enumerate}[(i)]
		\item $\forall f,g,h\in G : f\circ (g\circ h) = (f\circ g)\circ h$ (Assoziativität)
		\item $\exists e\in G\forall g\in G : e\circ g = g$ (Existenz eines neutralen Elements)
		\item $\forall g \in G \exists g^{-1} \in G : g^{-1}\circ g = e$ (Existenz eines inversen Elements)
	\end{enumerate}
	
	Die Gruppe heißt kommutativ oder abelsch, falls zusätzlich gilt:
	\begin{equation*}
		\forall g,h\in G: g\circ h = h\circ g \text{ (Kommutativität)}
	\end{equation*}

\paragraph{Bemerkung:}
	Das ist eine axiomatische Definition, d.h. der Begriff \glqq Gruppe\grqq{} wird durch (aus vielen (!) Beispielen abstrahierten) \glqq Rechenregeln\grqq{} definiert.
\paragraph{Beispiel:}
	Die rationalen Zahlen $\mathbb{Q}$ bilden mit der Addition eine Gruppe $(\mathbb{Q} ,+)$.
	Die rationalen Zahlen ohne $0$, $\mathbb{Q}^{\times} := \mathbb{Q}\setminus \{D\}$, bilden mit der Multiplikation eine Gruppe $(\mathbb{Q}^\times ,\cdot)$.

\paragraph{Definition:}
	Sind $(G,\circ )$ eine Gruppe und $X$ eine Menge, so heißt eine Abbildung
	\begin{equation*}
		\cdot : G\times X\to X, (g,x)\mapsto g\cdot x
	\end{equation*}
	
	eine Gruppenoperation (von $(G,\circ )$ auf $X$), falls

	\begin{enumerate}[(i)]
		\item $\forall g,h\in G :\forall x\in X: g\cdot (h\cdot x) = (g\circ h)\cdot x$ (entspricht nicht der Assoziativität!)
		\item $\forall x\in X: e\cdot x = x$ für das neutrale Element $e$ der Gruppe $(G,\circ )$
	\end{enumerate}
	$(G,\circ )$ heißt dann Transformationsgruppe von X.

\paragraph{Bemerkung:}
	Operiert $G$ (kurz für $(G,\circ )$, aus dem Zusammenhang ersichtlich) auf $X$, so ist für jedes $g\in G$ die Abbildung $g:X\to X, x\mapsto g\cdot x$ eine bijektive Abbildung von $X$ auf sich. Wegen der Axiome (i) und (ii) aus der Definition erhält man $g^{-1}: X\to X$ als Inverse der Abbildung.
\paragraph{Beispiel und Definition:}
	Die bijektiven Abbildungen einer Menge $X$ auf sich, $G:= \{g:X\to X\mid g \text{ bij}\}$, bilden (mit der Komposition $\circ$) eine (Transformations-)Gruppe $(G,\circ )$ (die auf $X$ operiert): die Permutationsgruppe oder symmetrische Gruppe $S_X$ von $X$. Für $X=\{1,2,...,n\}$ schreibt man auch $S_n$ statt $S_{\{1,...,n\}}$.
\paragraph{Bemerkung:}
	Im Gegensatz zu allgemeinen Abbildungen stimmen in (Permutations-)Gruppen Links- und Rechtsinverse stets überein.
\paragraph{Lemma:}
	Das neutrale Element einer Gruppe $(G,\circ )$ ist eindeutig und $\forall g\in G: g\circ e = g$. Weiters: 
	\begin{equation*}
		\forall g\in G \exists ! g^{-1} \in G: g^{-1}\circ g = g \circ g^{-1} = e
	\end{equation*}

\paragraph{Beweis:}
	Sei $g\in G$ gegeben und (gemäß Gruppenaxiom (iii)):
	\begin{itemize}
		\item $h:= g^{-1}$ (Linksinverse von $g$)
		\item $k:= h^{-1}$ (Linksinverse von $h$)
	\end{itemize}
	
	Damit berechnen wir (multiplikative Schreibweise: $a\circ b = ab$):
	\begin{gather*}
		hg = e = kh = k((hg)h) = k(h(gh)) = (kh)(gh) = gh\\
	\text{und }	ge = g(hg) = (gh)g = eg
	\end{gather*}
	
	Jedes (links-)neutrale Element $e$ ist also auch rechtsneutral:\hfill
	$\forall g\in G: eg = ge = g$
	
	und ist $e'\in G$ auch neutrales Element, dann:\hfill
	$ e' = ee' = e'e = e $

	Weiters ist jedes (Links-)Inverse auch rechtsinvers:\hfill
	$ \forall g \in G: gg^{-1}=g^{-1}g = e $

	und sind $h,h'\in G$ Inverse von $g\in G$, so gilt:\hfill
	$ h' = h'(gh) = (h'g)h = h $

	
	d.h. Eindeutigkeit des Inversen.

\subsection{Körper}
\paragraph{Definition (Körper):}
	Ein Tripel $(K,+,\cdot)$, bestehend aus einer Menge $K$ und zwei Verknüpfungen
	\begin{align*}
		+:&K\times K\to K,(x,y)\mapsto x+y\\
		\cdot : &K\times K\to K, (x,y)\mapsto xy
	\end{align*}
	
	heißt Körper, falls:
	\begin{enumerate}[(i)]
		\item $(K,+)$ ist abelsche Gruppe (mit neutralem Element $0$ und inversem Element $-x$ von $x$)
		\item $(K^\times,\cdot)$ ist abelsche Gruppe (mit neutralem Element $1$ und inversem Element $\frac{1}{x} = x^{-1}$ von $x\in K^\times$)
		\item die Distributivgesetze gelten:
	\end{enumerate}
	
	\begin{equation*}
		\forall x,y,z\in K :\begin{cases}x\cdot (y+z) = xy+xz\\ (x+y)\cdot z = xz+yz \end{cases}
	\end{equation*}

\paragraph{Bemerkung:}
	In einem Körper gilt stets:
	\begin{equation*}
		0\cdot x = x\cdot 0 = 0 \Rightarrow
		0\cdot x = (0+0)\cdot x = 0\cdot x + 0\cdot x \Rightarrow
		0 = 0\cdot x + (-(0\cdot x)) \Rightarrow 0 = 0\cdot x
	\end{equation*}
	
	Insbesondere folgt damit: $\forall x,y\in K: xy = yx$ (nicht nur für $K^\times$ (Axiom)).
\paragraph{Beispiel:}
	Die rationalen Zahlen $\mathbb{Q}$, die reellen Zahlen $\mathbb{R}$ und die komplexen Zahlen $\mathbb{C}$ bilden mit den üblichen Verknüpfungen Körper.

\paragraph{Bemerkung und Beispiel}
	Aufgrund der Axiome (i) und (ii) enthält $ K $ mindestens 2 Elemente.
	\begin{equation*}
		\# K \geq 2,
	\end{equation*}
	
	nämlich:
	\begin{itemize}
		\item $ 0 $, das neutrale Elemente bezüglich $+$
		\item $1 (\neq 0)$, das neutrale Elemente (in $K^\times$ = $K\setminus\{0\}$) bezüglich $\cdot$
	\end{itemize}
	
	Es gibt auch einen Körper mit genau 2 Elementen $(\{0,1\},+,\cdot)$, wobei

	\begin{minipage}{0.45\textwidth}
		\begin{equation*}
			\begin{tabular}{c|c|c}
				$+$ & 0 & 1\\\hline
				0 & 0 & 1\\
				1 & 1 & 0\\
			\end{tabular}
		\end{equation*}
	\end{minipage}
	\begin{minipage}{0.45\textwidth}
		\begin{equation*}
			\begin{tabular}{c|c|c}
				$\cdot$ & 0 & 1\\\hline
				0 & 0 & 1\\
				1 & 1 & 1\\
			\end{tabular}
		\end{equation*}
	\end{minipage}
	
	Dieser Körper wird auch $\mathbb{Z}_2$ bezeichnet.

\paragraph{Bemerkung und Definition:}
	In $\mathbb{Z}_2: 1 + 1 = 0$. Allgemeiner definiert man die Charakteristik eines Körpers $(K,+,\cdot)$ (mit neutralen Elementen 0 und 1 von + bzw. $\cdot$) durch

	\begin{equation*}
		Char(K):=
		\begin{cases}
			0,\text{falls } \forall n \in \mathbb{N}^\times: \sum_{j = 1}^{n} 1 \neq 0\\
			\min\{n \in \mathbb{N}^\times\mid \sum_{j = 1}^{n} 1 = 1+ ... + 1 = 0\}
		\end{cases}
	\end{equation*}
	
	z.B. $Char(\mathbb{Z}_2) = 2$, da
	\begin{gather*}
		\{n\in\mathbb{N}^\times\mid 1+...+1=0\}=\\
		=\{n\in\mathbb{N}^\times\mid n=0 \text{ mod } 2\}=\\
		=\{n\in\mathbb{N}^\times\mid n \text{ gerade}\}\\
		\text{und damit: }	\min\{n\in\mathbb{N}^\times\mid 1+...+1=0\}=2
	\end{gather*}
	
	Wir werden mitunter $Char(K,+,\cdot)\neq 0$ oder (öfter) $Char(K,+,\cdot)=2$ ausschließen (müssen).

\section{Translationen und Vektoren}

\begin{tikzpicture}[scale=1.5,>=triangle 45]
	\draw[->,color=black] (-0.1,0) -- (10,0);
	\draw[->,color=black] (0,-0.1) -- (0.,4);
	
	\coordinate[label=left:$x$] (x) at (1,2);
	\coordinate[label=below:$y\equal\tau_v(x)$] (y) at (5,1.5);
	\coordinate[label=above:$y'\equal\tau_w(x)$] (y') at (2,3.5);
	\coordinate (z) at (6,3);
	
	\draw [fill] (x) circle (.5pt);
	\draw [fill] (y') circle (.5pt);
	\draw [fill] (y) circle (.5pt);
	\draw [fill] (z) circle (.5pt);
	
	\draw [->] (x) to node[below left]{$ v $} (y);
	\draw [->] (x) --node[above left]{$ w $} (y');
	\draw [->] (y) --node[below right]{$ w $} (z);
	\draw [->] (y') --node[above right]{$ v $} (z);
	\draw [->] (x) --node[above]{$ v+w $} node[below]{$ w+v $} (z);
	
	\draw (z) node[above right] {$z = \tau_w(y)=(\tau_w\circ\tau_v)(x)=\tau_{w+v}(x)$};
	\draw (z) node[below right] {$z' = \tau_v(y')=(\tau_v\circ\tau_w)(x) = \tau_{v+w}(x)$};
	\draw (4,0.5) node[] {Translationen \glqq der\grqq{} Ebene bilden eine abelsche Gruppe};
\end{tikzpicture}

\begin{tikzpicture}[scale=1.5, >=triangle 45]
	\draw[->,color=black] (-0.1,0) -- (9,0);
	\draw[->,color=black] (0,-0.1) -- (0,4);
	
	\coordinate[label=left:$ x $] (x) at (1,3);
	\coordinate[label=right:$ y \equal \tau_v(x) \equal (\tau_{\frac{v}{2}} \circ \tau_{\frac{v}{2}})(x) \equal \tau_{\frac{v}{2} + \frac{v}{2}}(x) $] (y) at (3,1);
	
	\draw [fill] (x) circle (.5pt);
	\draw [fill] (y) circle (.5pt);
	
	\draw [->] (1,2.8) --node[left]{$v$} (3,.8);
	\draw [->] (1,3.2) --node[above] {$\frac{v}{2}$} (2,2.2);
	\draw [->] (2,2.2) --node[above] {$\frac{v}{2}$} (3,1.2);
	
	\draw (5,3) node[text width = 7cm] {Translationen kann man \glqq strecken\grqq{}, sodass gewisse Rechengesetze gelten.};
\end{tikzpicture}

\paragraph{Definition:}
	Sei $K$ ein Körper. Eine Menge $V$ mit zwei Abbildungen

	\begin{align*}
		 +&: V \times V \to V:(v,u)\mapsto v+w,\\
		 \cdot &: K \times V \to V:(x,v)\mapsto vx,
	\end{align*}
	
	heißt Vektorraum über $K$ ($K$-VR), falls gilt:
	\begin{enumerate}[(i)]
		\item $(V,+)$ ist eine abelsche Gruppe
		\item $\forall v\in V: v\cdot 1=v$ und\\
			$\forall x,y \in K, \forall v\in V= (vx)y = v(xy)$
		\item $\forall x,y \in K \forall v\in V: v(x+y) = vx + vy$\\
			$\forall x\in K \forall v\in V: (v+w)x = vx + wx$
	\end{enumerate}

\paragraph{Bemerkung:}
	Wir notieren die Skalarmultiplikation als Rechtsmultiplikation (Skalar steht rechts):
	\begin{equation*}
		\cdot: K \times V \to V : (x,v) \mapsto vx
	\end{equation*}

\paragraph{Beispiel:}
	Die Translationen eines affinen Raumes bilden einen Vektorraum (vgl. mit der Skizze oben): Diese Beispiel wird im nächsten Kapitel repräsentiert.
	
\paragraph{Beispiel:}
	Jeder Körper $ K $ ist ein $ K $-VR (Vektorraum über sich selbst): das ist ein (trivialer) Spezialfall des folgenden...
	
\paragraph{Beispiel und Definition:}
	Ist $ I $ eine Menge und $ K $ ein Körper, so bilden die $ K $-wertigen Abbildungen
	\begin{equation*}
		v: I \to K: i \mapsto v_i
	\end{equation*}

	einen Vektorraum mit der punktweise definierten Addition und Skalarmultiplikation:
	\begin{gather*}
		I\ni i \mapsto (v+w)_i := v_i+w_i\in K_i\\
		i \mapsto (vx)_i := v_ix \in K_i
	\end{gather*}

	Dieser Vektorraum wird mit $K^{I}$ bezeichnet und Standardvektorraum (über I und K) genannt. Im Falle $ I=\{1,...,n\} $ schreibt man auch $K^{n} := K^{\{1,...,n\}}$

\paragraph{Bemerkung und Definition:}
	Anstelle der normalen Schreibweise
	\begin{equation*}
		I\ni i \mapsto v(i) \in K
	\end{equation*}

	für die Auswertung einer Abbildung  $v: I \to K$ um einen Punkt $i\in I$ haben wir die Indexschreibweise verwendet.
	\begin{equation*}
		I\ni i \mapsto v_i \in K_i
	\end{equation*}

	Wir haben damit eine Abbildung $v: I \to K$ als Familie von
	\begin{equation*}
		(v_i)_{i\in I}
	\end{equation*}

	über der Indexmenge I aufgefasst -- der Begriff Familie ist ein \glqq alternativer\grqq{} Begriff für Abbildungen.
	
\paragraph{Beispiel:}
	Sei i eine \glqq Zahl\grqq{} mit $i^2=-1$ ($i$ entspricht nicht dem Element der Indexmenge aus dem vorherigen Abschnitt). Die komplexen Zahlen
	\begin{equation*}
		\mathbb{C}:=\{{x+iy\mid x,y\in \mathbb{R}}\}
	\end{equation*}
 
	bilden mit der Addition und Multiplikation einen Körper:
	\begin{align*}
		+&:\mathbb{C}\times \mathbb{C} \to \mathbb{C}: ((x+y),(x'+y')) \mapsto ((x+y)+(x'+y')) := (x+x')+i(y+y')\\
		\cdot &:\mathbb{C}\times \mathbb{C} \to \mathbb{C}: ((x+iy),(x'+iy'))\mapsto (x+iy)\cdot (x'+iy') :=(xx'-yy')+i(xy'+x'y)
	\end{align*}

	Die komplexen Zahlen $\mathbb{C}$ bilden einen $\mathbb{R}$-VR mit
	\begin{equation*}
		+:\mathbb{C}\times\mathbb{C}\to\mathbb{C}
	\end{equation*}

	wie oben und der Skalarmultiplikation
	\begin{equation*}
		\cdot:\mathbb{R}\times\mathbb{C}\to\mathbb{C}:(x',(x+iy))\mapsto(x+iy)x':=xx'+iyx'
	\end{equation*}

	Diese Skalarmultiplikation ist also gerade die Einschränkung der komplexen Multiplikation auf $\mathbb{R}\times\mathbb{C}$ wobei die Identifikation
	\begin{equation*}
		\mathbb{R}\cong \{{x+iy\in\mathbb{C}\mid y=0}\}
	\end{equation*}

	verwendet wird.
% % % %VO5 % % % %
\subsection{Untervektorräume \& lineare Hülle}
\paragraph{Definition:}
	Eine Teilmenge $U\subset V$ eines $K$-VR $V$ heißt Unter(vektor)raum (UVR), falls $U$ mit der eingeschränkten Addition und Skalarmultiplikation
	\begin{align*}
		 ^+    & \mid_{U\times U}: U\times U \to V,(v,w) \mapsto v+w \\
		 \cdot & \mid_{K\times U}: K\times U \to V,(x,v) \mapsto vx
	\end{align*}

	selbst ein Vektorraum ist, d.h. wenn insbesondere
	\begin{gather*}
		\forall v,w \in U: v+w\in U \text{ und}\\
		\forall x\in K\forall v\in U: vx\in U.
	\end{gather*}

\paragraph{Bemerkung}
	Eine nicht-leere Teilmenge $U\subset V, U\neq\emptyset$, ist genau dann ein UVR, wenn die auf U eingeschränkten Operationen wohldefiniert sind, d.h. wenn $ U $ bzgl. $ + $ und $ \cdot $ abgeschlossen ist.

	Dies kann zum Unterraumkriterium zusammengefasst werden:
	\begin{equation*}
		U\subset V \text{ ist UVR }\Leftrightarrow 
 		 \begin{cases}
 		 	U\neq\emptyset\\
 		 	\forall v,w\in U\forall x\in K: vx+w\in U
 		 \end{cases}
	\end{equation*}

\paragraph{Beispiel}
	Sei $I=\{1,...,n\}$. Für jedes (feste) $i\in I$ ist
	\begin{equation*}
		U_i := \{v:I\to K\mid v_i =0\}
	\end{equation*}

	ein UVR von $K^n$, denn
	\begin{enumerate}
		\item $v = 0 \in U_i\text{, also } U_i \neq \emptyset$
		\item Seien $v,w\in U_i$, d.h. $v,w\in K^n$ mit $v_i =w_i =0$, und $x\in K$; dann gilt $(vx+w)_i = v_ix+ w_i = 0\cdot x + 0 = 0$, also: $vx+w\in U_i$ und damit ist $U_i$ UVR nach Unterraumkriterium.  
	\end{enumerate}
	
	Kein UVR von $K^n, n\geq 2$, ist jedoch die Menge
	\begin{equation*}
		N:=\{v:I\to K\mid v_1\cdot v_2 = 0\},
	\end{equation*}
  
	denn 
	\begin{enumerate}
		\item $N$ ist zwar nicht-leer, $N\neq \emptyset$, aber
		\item $^+\mid_{N\times N}: N\times N\to N$ nicht wohldefiniert: seien $v,w\in N$, so dass
			\begin{gather*}
				v_1=0, v_2=1\text{ }(v_3 ... v_n \text{ irrelevant})\\
				w_1=1, w_2 = 0\text{ }(w_3 ... w_n \text{ irrelevant})
			\end{gather*}
	\end{enumerate}
	
	dann gilt:
	\begin{gather*}
		(v+w)_1 = v_1 + w_1 = 0+1=1\\
		(v+w)_2 = v_2 + w_2 = 1+0 = 1
	\end{gather*}

	und damit
	\begin{equation*}
		(v+w)_1(v+w)_2 = 1 \Rightarrow v+w\notin N
	\end{equation*}

\paragraph{Bemerkung und Beispiel:}
	In analoger Weise definiert man die Begriffe
	\begin{itemize}
		\item einer Untergruppe $H\subset G$ einer Gruppe $(G,\cdot)$, bzw.
		\item eines Unter- oder Teilkörpers $T\subset K$ eines Körpers $(K,+,\cdot )$
	\end{itemize}
	
	Z.B.: Jeder UVR $U\subset V$ eines $K$-VR $V$ bildet (mit der Addition) eine Untergruppe der Gruppe $(V,+)$.
    Und: In gleicher Weise ist eine nicht-leere (!) Teilmenge ein/e Unterkörper/-gruppe, falls die eingeschränkten Operationen wohldefiniert sind.
    
    Z.B.: ist $H\subset G$ eine Untergruppe, falls (Untergruppenkriterium):
    \begin{enumerate}
        \item $H\neq \emptyset$
        \item $\forall g,h\in H: g\circ h^{-1} \in H$
    \end{enumerate}
            
	Achtung: Inversenbildung muss im Kriterium explizit formuliert werden, sonst würde z.B.: $\mathbb{N}\subset\mathbb{Z}$ als Teilmenge von $(\mathbb{Z}, +)$ als Gruppe ein Gegenbeispiel liefern.
            
     Z.B.: 
     \begin{itemize}
        \item die Translationen bilden eine Untergruppe der Bewegungsgruppe
        \item $\mathbb{Q}\subset\mathbb{R}$ und $\mathbb{R}\cong \{x+iy\mid y=0\}\subset\mathbb{C}$ bilden Teilkörper von $\mathbb{R}$ bzw. $\mathbb{C}$.
     \end{itemize}

\paragraph{Lemma:}
    Ist $(U_i)_{i\in I}$ eine Familie von UVR $U_i\subset V$ eines $K$-VR $V$, so ist ihr Schnitt
    \begin{equation*}
        U:= \bigcap_{i\in I}U_i =\{ u\in V\mid \forall i\in I: u\in U_i\}
    \end{equation*}
        
    ein UVR von $V$. (Beweis in Aufgabe 17)
    
\paragraph{Definition:}
	Die lineare Hülle $[S]$ einer Teilmenge $S\subset V$ eines $ K $-VR $ V $ ist der Schnitt aller S enthaltenden UVR $U\subset V$:
	\begin{equation}
		[S] := \bigcap_{S\subset U \text{UVR}} U
	\end{equation}

	Die lineare Hülle einer Familie $(v_i)_{i\in I}$ von Vektoren $v_i\in V$ in einem $ K $-VR $ V $ ist:
        \begin{equation}
        	[(v_i)_{i\in I}] := [\{v_i\mid i\in I\}]
        \end{equation}

\paragraph{Bemerkung:}
    $[S]$ ist ein UVR (nach Lemma) -- der \glqq kleinste\grqq{} UVR, der S enthält, d.h. ist $U\subset V$ UVR mit $S\subset U$, so gilt $[S]\subset U$; da aber $[S] = \bigcap_{S\subset \tilde{U}  \text{UVR}}\tilde{U}\subset U$,
    da $S\subset U$, also $U$ am Schnitt beteiligt ist.
    
\paragraph{Bemerkung:}
	$[\emptyset ] = \{o\}$ und $[V] = V$.
	
\paragraph{Beispiel:}
	Ist $U\subset V$ UVR, so gilt $[U] = U$
	
\paragraph{Beispiel:}
	$N=\{v:I\to K\mid v_1v_2=0\} \subset K^n,I=\{1,...,n\},n\geq 2$, hat lineare Hülle $[N]=K^n$
	
\paragraph{Beispiel:}
	Für $I=\{1,...,n\}$ und $i\in I$ definiere
	$e_i:I\to K , j\mapsto e_i(j):= \delta_{ij}$, wobei 
	\begin{equation*}
		\delta_{ij} :=
		\begin{cases}
			1,& \text{falls }i=j\\
			0,& \text{sonst}
		\end{cases}
	\end{equation*}
	das Kroneckersymbol bezeichnet.
	
	Dann ist die lineare Hülle der Familie $(e_i)_{i\in I}$
	\begin{equation*}
		[(e_i)_{i\in I}] = K^n
	\end{equation*}
	
	Nämlich: Da $[(e_i)_{i\in I}]\subset K^n$ ist, gilt für beliebige $x_1,...,x_n\in K$
	\begin{gather*}
		e_1x_1+...+e_nx_n,\in [(e_i)_{i\in I}]\\ \text{denn nach Unterraumkriterium: } (e_1x_1+...+(e_{n-1}x_{n-1}+(e_nx_n + 0))...) \in [(e_i)_{i\in I}]
	\end{gather*}
	
	da $[(e_i)_{i\in I}] \subset K^n$ UVR ist. Andererseits gilt für beliebiges $v\in K^n$:
	\begin{equation*}
		v=\sum^n_{i=1}e_iv(i): I\to K,
	\end{equation*}
	
	denn
	\begin{equation*}
		\forall j\in I: \left(\sum^n_{i=1} e_iv(i)\right)(j) = \sum^n_{i=1}e_i(j)v(i) = (\delta_{ij}) v(j) = v(j)
	\end{equation*}
	
	damit ist gezeigt, dass die beiden Abbildungen übereinstimmen; da $v\in K^n$ beliebig war, folgt $K^n \subset [(e_i)_{i\in I}]$
	
\paragraph{Definition:}
	Seien $(v_i)_{i\in I}$ und $(x_i)_{i\in I}$ Familien in einem $ K $-VR bzw. dem Körper $ K $, wobei
	\begin{gather*}
		\# \{i\in I\mid x_i \neq 0\} < \infty\text{ ,also}\\
		\{ i\in I \mid x_i \neq 0\} = \{i_1,...,i_n\}\text{ für ein geeignetes } n\in \mathbb{N}
	\end{gather*}
	
	
	Dann heißt die endliche Summe
	\begin{equation*}
    	\sum_{i\in I} v_ix_i:= \sum^n_{j=1}v_{ij}x_{ij}\text{ eine Linearkombination.}
	\end{equation*}

\paragraph{Bemerkung:}
	Die Bedingung
	\begin{equation*}
		\#\{i\in I \mid x_i\neq 0\} <\infty
	\end{equation*}
	
	garantiert, dass die Summe wohldefiniert ist $\rightarrow$ vgl. Reihen in der Analysis.
