% % 2015-11-19 % %
\chapter{Affine Geometrie}
\begin{tikzpicture}[scale=1.5,>=triangle 45]
	\draw[->,color=black] (-0.1,0) -- (10,0);
	\draw[->,color=black] (0,-0.1) -- (0.,4);
	
	\coordinate[label=left:$x$] (x) at (1,1);
	\coordinate[label=below:$\tau_v(x)$] (y) at (5,1.5);
	\coordinate[label=above:$\tau_w(x)$] (y') at (2,2.5);
	\coordinate (z) at (6,3);
	
	\draw [fill] (x) circle (.5pt);
	\draw [fill] (y') circle (.5pt);
	\draw [fill] (y) circle (.5pt);
	\draw [fill] (z) circle (.5pt);
	
	\draw [->] (x) to node[below left]{$ v $} (y);
	\draw [->] (x) --node[above left]{$ w $} (y');
	\draw [->] (y) --node[below right]{$ w $} (z);
	\draw [->] (y') --node[above right]{$ v $} (z);
	
	\draw (z) node[above right] {$\tau_w(\tau_v(x))=(\tau_w\circ\tau_v)(x)=\tau_{w+v}(x)$};
	\draw (z) node[below right] {$\tau_v(\tau_w(x))=(\tau_v\circ\tau_w)(x) = \tau_{v+w}(x)$};
\end{tikzpicture}
\paragraph{Definition (nach Klein): }
\begin{Definition}[Geometrie]
	Eine Geometrie besteht aus einer Menge $ A $ (z.B. Punktmenge) und einer darauf operierenden Gruppe $ (G,*) $, d.h.,
	es gibt eine Gruppenoperation
		\[ \rho: G\times A\to A,(g,a)\mapsto \rho_g(a)  \]
	wobei gilt
		\begin{enumerate}[(i)]
			\item $ \forall a\in A\forall g,h,\in G:(\rho_g\circ \rho_h)(a) = \rho_{g*h}(a) $
			\item $ \forall a\in A:\rho_e(a) = a $ für das neutrale Element $ e \in G $
		\end{enumerate}
\end{Definition}

\paragraph{Definition: }
\begin{Definition}[Affiner Raum (AR)]
	Sei $ K $ ein Körper. Ein affiner Raum (AR) $ (A,V,\tau) $ über $ K $ besteht aus einer Menge $ A $, einem $ K $-Vektorraum $ V $ und einer Gruppenoperation
		\[ \tau:V\times A\to A,(v,a)\mapsto \tau_v(a) \]
	von $ V $ (als additive Gruppe $ (V,+) $) auf $ A $, die einfach transitiv ist, d.h.
		\[ \forall a,b\in A\exists!v\in V:b=\tau_v(a) \]
\end{Definition}

\begin{figure}\centering
\begin{tikzpicture}[scale=1.5,>=triangle 45]
	\draw[->,color=black] (-0.1,0) -- (5,0);
	\draw[->,color=black] (0,-0.1) -- (0.,2);
	
	\coordinate[label=left:$a$] (x) at (1,0.5);
	\coordinate[label=right:${b=\tau_v(a)}$] (y) at (3,1.5);

	\draw [fill] (x) circle (.5pt);
	\draw [fill] (y) circle (.5pt);
	
	\draw [->] (x) to node[below]{$ v $} (y);
	\draw (5,0.5) node[] {Der Verbindungsvektor ist eindeutig.};
	
\end{tikzpicture}
\end{figure}

	Weiters nennen wir
		\begin{itemize}
			\item Elemente von $ A $ Punkte,
			\item $ V $ den Richtungsvektorraum oder Tangentialraum von $ A $,
			\item $ v $ mit $ \tau_v(a)=b $ den Verbindungsvektor von $ a $ nach $ b $,
			\item $ \tau_v:A\to A, a\mapsto \tau_v(a) $ die Translation von $ v $
			\item und $ \dim V $ die Dimension des affinen Raums $ A $
		\end{itemize}
		
\subparagraph{Bemerkung: }
	Die Translationen eines AR $ A $ bilden eine abelsche Gruppe.
	
	Alternative Notation:
		\[ a+v:=\tau_v(a) \text{ und } b-a:= v\text{, falls } b=\tau_v(a) \]
	Mit dieser alternativen Schreibweise für die Operation von $ (V,+) $ auf $ A $, erscheinen die Bedingungen, dass $ V=(V,+) $ einfach transitiv auf $ A $ operiert, "`offensichtlich"'.
	
	Gruppenoperation:
		\begin{enumerate}[(i)]
			\item $ \forall a\in A\forall v,w,\in V: (a+v)+w = a+(v+w) $ ist kurz für $ \tau_w(\tau_v(a)) = \tau_{v+w}(a) $, entspricht also nicht der Assoziativität.
			\item $ \forall a\in A:a+0=a $ entspricht $ \tau_0(a) = a $
		\end{enumerate}
	Transitivität:
		\[ \forall a,b\in A\exists v\in V: b=a+v \]
	Nämlich: sind $ a,b\in A $ gegeben, so liefert $ v:=b-a $ (weil $ V $ einfach transitiv operiert) eindeutig den gesuchten Vektor.

\paragraph{Beispiel \& Definition: }
	Jeder $ K $-VR liefert einen affinen Raum $ (V,V,\tau) $ mit der Operation
		\[ \tau: V\times V\to V, (v,a)\mapsto \tau_v(a):= a+v \]
	von $ V $ auf sich selbst -- die Unterscheidung zwischen Punkten und Vektoren wird dann etwas undurchsichtig.
	
	Der affine Standardraum $ (K^n,K^n,\tau) $ wird mit $ A^n $ bezeichnet.
\paragraph{Beispiel \& Definition}
	Sei $ (A,V,\tau) $ AR, für jede Wahl eines Ursprungs $ o\in A $ ist
		\[ \tau(o) :V\to A,v\mapsto \tau_v(o) \]
	eine Bijektion -- ein VR ist also ein "`AR mit Ursprung"'.
	
\subparagraph{Beispiel: }
	Auf einem Zylinder
	\[ Z^2 := S^1\times \mathbb{R}:= (\mathbb{R}/2\pi\mathbb{Z})\times \mathbb{R} \]
	liefert die Operation
		\[ \tau:\mathbb{R}^2\times Z^2\to Z^2,(v,a)\mapsto a+v \]
	keinen affinen Raum, da diese Operation nicht einfach transitiv ist: zu je zwei Punkten gibt es unendlich viele "`Verbindungsvektoren"'.

%-----------------------------------------------------------------------
\tdplotsetmaincoords{340}{0} 
\begin{tikzpicture}[scale=2,tdplot_main_coords]
\def\cyradius{1.2}
\def\cyhight{2.5}
\def\xstart{-6}
\def\ystart{1.2}
%\def\xstart{-\cyradius}
%\def\ystart{-2}

\def\xpostext{\xstart-0.1}
\def\ypostext{\ystart-1}

\coordinate[color=blue,label={[xshift=-10, yshift=5]:$\mathbb{R}^2$}] (Nullpunkt) at (\xstart,\ystart,0);
%x,Z,y
\foreach \t in {0,5,...,180}{%
\draw[line width=1pt,color=red] ({\cyradius*cos(\t)},{0},{\cyradius*sin(\t)})--({\cyradius*cos(\t+5)},{0},{\cyradius*sin(\t+5)});
}

\foreach \t in {0,5,...,360}{%
\draw[line width=1pt,color=red] ({\cyradius*cos(\t)},{\cyhight},{\cyradius*sin(\t)})--({\cyradius*cos(\t+5)},{\cyhight},{\cyradius*sin(\t+5)});
}

\draw[line width=1pt,color=red] ({\cyradius},{0},{0})--({\cyradius},{\cyhight},{0});
\draw[line width=1pt,color=red] ({-\cyradius},{0},{0})--({-\cyradius},{\cyhight},{0});

\foreach \t in {0,-5,...,-180}{%
\draw[line width=1pt,color=blue] ({\cyradius*cos(\t)},{1-\t/360},{sin(\t)})--({\cyradius*cos(\t+2)},{1-\t/360},{sin(\t +2)});
}%for end

\foreach \t in {-180,-181,...,-260}{%
\ifthenelse{\t=-260}{\draw[-{>[scale=1,length=10,width=6]},line width=1pt,color=blue] ({\cyradius*cos(\t)},{1-\t/360},{sin(\t)})--({\cyradius*cos(\t-5)},{1-\t/360},{sin(\t -5)});}{%else Zweig
\ifthenelse{\t=-225}{\draw[line width=1pt,color=blue] ({\cyradius*cos(\t)},{1-\t/360},{sin(\t)})--({\cyradius*cos(\t-5)},{1-\t/360},{sin(\t -5)})node[below]{$v$};}{\draw[line width=1pt,color=blue] ({\cyradius*cos(\t)},{1-\t/360},{sin(\t)})--({\cyradius*cos(\t-5)},{1-\t/360},{sin(\t -5)});}
}
}%for end

\foreach \t in {85,84,...,0}{%
\draw[line width=1pt,color=blue] ({\cyradius*cos(\t)},{1-\t/360},{sin(\t)})--({\cyradius*cos(\t+5)},{1-\t/360},{sin(\t +5)});
}
%zwei Punkte x,y und u,z
\draw[fill,color=red] (0,0.75,1) circle [x=1cm,y=1cm,radius=0.05]node[below,label={[xshift=0, yshift=-34]$\text{Äquivalenzklasse } \mathbb{R}^2_{(x,y)}$}]{$(x,y)\sim (\tilde{x},\tilde{y})$};
\draw[fill,color=red] (0,1.75,1) circle [x=1cm,y=1cm,radius=0.05]node[ yshift=20,label={[xshift=0, yshift=22]$Z^2 =S^{1} \times \mathbb{R}^{1}$}]{$(u,z)\sim (\tilde{u},\tilde{z})$};
% Richtungsvektor
\draw[-{>[scale=1,length=10,width=6]},shorten >=6pt, shorten <=6pt,line width=1pt,color=blue] (0,0.75,1) -- (0,1.75,1) node[midway, right]{$v$} ;
%%blause Koordinatensystem
\draw[-{>[scale=1,length=10,width=8]},line width=1pt,color=blue] ({\xstart},{\ystart-1.5},{0})--({\xstart},{\ystart+1.5},{0});
\draw[-{>[scale=1,length=10,width=8]},line width=0.75pt,color=blue] ({\xstart-1},{\ystart},{0})--({\xstart+3*\cyradius},{\ystart},{0});
%rote Linien
\draw[line width=1pt,color=red] ({\xstart+2*\cyradius},{\ystart-0.6},{0})--({\xstart+2*\cyradius},{\ystart+1.5},{0});
\draw[line width=1pt,color=red] ({\xstart},{\ystart-1},{0})--({\xstart},{\ystart+1},{0});

%rote Punkte 2d x,y und x1,y1
\draw[fill,color=red] ({\xstart+0.5},{\ystart-0.5},0) circle [x=1cm,y=1cm,radius=0.05]node[above]{$(x,y)$};
\draw[fill,color=red] ({\xstart+0.5+2*\cyradius},{\ystart-0.5},0) circle [x=1cm,y=1cm,radius=0.05]node[above]{$(\tilde{x},\tilde{y})$};

%rote Punkte 2d u,z und u1,z1
\draw[fill,color=red] ({\xstart+0.5},{\ystart+0.5},0) circle [x=1cm,y=1cm,radius=0.05]node[above]{$(u,z)$};
\draw[fill,color=red] ({\xstart+0.5+2*\cyradius},{\ystart+0.5},0) circle [x=1cm,y=1cm,radius=0.05]node[above]{$(\tilde{u},\tilde{z})$};

%text node unterhalb der 3d graphik    
\node[text width=6cm, anchor=north west, text centered] at (\xpostext,\ypostext,0)
    {$(x,y)\sim (\tilde{x},\tilde{y})$ \\ $:\Leftrightarrow \begin{cases} \exists k \in \mathbb{Z}: &  \tilde{x} =x + 2k \pi, \\  & \tilde{y} = y \end{cases}$};
    
\draw[->,line width=1pt,color=red, dashed, shorten >=7pt, shorten <=7pt] (\xpostext+0.89,\ypostext-0.15,0) -- ({\xstart+0.5},{\ystart-0.5},0);
   
\draw[->,line width=1pt,color=red, dashed, shorten >=7pt, shorten <=7pt] (\xpostext+2.21,\ypostext-0.15,0) -- ({\xstart+0.5+2*\cyradius},{\ystart-0.5},0);
\end{tikzpicture}
%-----------------------------------------------------------------------

\paragraph{Beispiel \& Definition: }
	Ist $ U\subset V $ UVR eines $ K $-VR $ V $, so liefert jedes $ v\in V $ die Nebenklasse
		\[ A = v+U \]
	einen affinen Raum $ (A,U,\tau) $ mit 
		\[ \tau:U\times A\to A,(u,a)\mapsto \tau_u(a):= a+u; \]
	offensichtlich ist die Operation wohldefiniert (operiert auf der Nebenklasse) und einfach transitiv.
	
	Eine Nebenklasse $ A= v+U\subset V $ nennt man daher auch einen affinen Unterraum des VR $ V $.
\paragraph{Definition: }
\begin{Definition}[Affiner Unterraum (AUR)]
	$ A'\subset A $ ist affiner Unterraum (AUR) des affinen Raumes $ (A,V,\tau) $, falls
		\[ \exists a\in A\exists U\subset V \text{UVR}:A' = a+U = \{\tau_u(a)\mid u\in U\}.\]
	Ist $ \dim A' =1 $ oder $ \dim A' = 2 $, so heißt $ A' $ (affine) Gerade bzw. Ebene; ist $ \dim A' < \infty $ und $ \dim A' = \dim A-1 $, so heißt $ A' $ (affine) Hyperebene.
\end{Definition}
\subparagraph{Bemerkung: }
	Jeder AUR ist selbst AR mit der "`geerbten"' (eingeschränkten) Operation.
\subparagraph{Beispiel: }
	Ist $ f\in \hom(V,W) $ und $ w\in f(V) $, so erhält man einen affinen Raum
		\[ (f^{-1}(\{w\}),\ker f,\tau) \text{ mit }\tau_u(a):= a+u.\]
	Ist $ f\in V^*\setminus \{o\} $ (und $ \dim V<\infty $), so wird $ f^{-1}(\{x\})\subset V $ für jedes $ x\in K (=f(V)) $ eine affine Hyperebene in $ (V,V,\tau) $ -- nach Rangsatz.
	
% % % 2015-11-24

\subparagraph{Bemerkung: }
	Ist $ A' = a+U\subset A $ AUR des AR $ (A,V,\tau) $, so gilt
		\[ \forall b\in A'\exists u\in U:b=\tau_n(a)\]
	und damit
	\begin{align*}
		b+U&=\{\tau_{u'}(b)\mid u'\in U\}\\
		&=\{(\tau_{u'}\circ \tau_u)(a)=\tau_{u'+u}(a)\mid u'\in U\}\\
		&= \{\tau_{u''}(a)\mid u'' \in U\} = a+U = A'
	\end{align*}
	Damit zeigt man: Ist $ (A'_i)_{i\in I} $ eine Familie AUR $ A'_i\subset A $ eines AR $ A $, so ist der Schnitt leer oder ein affiner Unterraum. Ist nämlich der Schnitt nicht leer, d.h.,
	\[ \exists a\in A\forall i\in I:a\in A'_i, \]
	so erhält man
	\begin{gather*}
		\forall i\in I:A'_i = a+U_i \text{ mit einem geeigneten UVR } U_i\subset V\\
		\Rightarrow \bigcap_{i\in I}A'_i = a+\bigcap_{i\in I}U_i \text{ und } U:= \bigcap_{i\in I}U_i\subset V \text{ ist UVR.}
	\end{gather*}
\paragraph{Definition: }
	\begin{Definition}[Affine Hülle]
		Die affine Hülle $ [S] $ einer Teilmenge eines affinen Raumes $ A $ ist der Schnitt aller $ S $ enthaltenden AUR $ A'\subset A $,
		\[ [S] = \bigcap_{S\subset A' \text{AUR}}A'. \]
	\end{Definition}
\subparagraph{Bemerkung: }
	Die affine Hülle einer Teilmenge $ S\subset A $ ist also der kleinste $ S $ enthaltende affine Unterraum von $ A $.
	
	Achtung: In einem $ K $-VR $ V $ (den kann man auch als AR auffassen, siehe Beispiel vorher) sind die lineare Hülle und die affine Hülle (in $ V $ aufgefasst als AR) im Allgemeinen verschieden:
		\[ [S]_{\text{lin}} = \bigcap_{S\subset U\text{ UVR}}U \neq \bigcap_{S\subset A \text{ AUR}}A = [S]_{\text{aff}} \]
\subparagraph{Beispiel: }
	Für $ S=\{a\}\subset V $ mit $ a\neq 0 $ gilt
		\[ [S]_{\text{lin}} = \{ax\in A = V\mid x\in K\} \neq \{a\} = [S]_{\text{aff}} \]
	allgemein gilt:
		\[ [S]_{\text{aff}}\subset [S\cup \{0\}]_{\text{aff}}=[S]_{\text{lin}} \]
	Beweis in Aufgabe 45.
\paragraph{Lemma \& Definition (baryzentrischer Kalkül): }
	\begin{Definition}[Affinkombination / Baryzentrum]
		Seien $(a_i)_{i\in I}$ und $(x_i)_{i\in I}$ Familien in einem AR $ A $ über $ K $ bzw. in $ K $, wobei
		\[ \#\{i\in I\mid x_i\neq 0\}<\infty \text{ und } \sum_{i\in I}x_i=1; \]
		dann ist die mit einem beliebigen Ursprung $ o\in A $ definierte Affinkombination	
		\[ \sum_{i\in I}a_ix_i := o+\sum_{i\in I} (a_i-o)x_i \]
		wohldefiniert, d.h. unabhängig von der Wahl des Ursprungs $ o\in A $.
		Dann heißt 
		\[ s:= \sum_{i\in I} a_ix_i \]
		Schwerpunkt oder Baryzentrum der Punkte $ a_i $ mit Gewichten $ x_i $.
	\end{Definition}
%-------------------------Begin Grafik Affinkombination---------------------------------    
\begin{figure}\centering
\tdplotsetmaincoords{0}{-27} %-27
\begin{tikzpicture}[scale=1,tdplot_main_coords]
 
\def\xstart{0}
\def\ystart{0}

\def\xstartdraw{(\xstart + 2)}
\def\ystartdraw{(\ystart + 1)}

\def\xlength{3.5}
\def\ylength{1.7}

%---------Begin Balken----------
\def\drehwinkel{-27}
\def\balkenbreite{0.4}
\def\balkenhoehe{(\ylength*2+2)}
\def\balkenlaenge{(\xlength*2+3)}

\node (VekV) at ({\xstart+0.5*cos(\drehwinkel)-\balkenbreite*sin(\drehwinkel)},{\ystart+0.5*sin(\drehwinkel)+\balkenbreite*cos(\drehwinkel)})[color=blue] {$V$};
\node (AffA) at ({\xstart+(\balkenlaenge-1)*cos(\drehwinkel)},{\ystart+(\balkenlaenge-1)*sin(\drehwinkel)+\balkenbreite*cos(\drehwinkel)})[color=red] {$A$};

\path[ shade, top color=white, bottom color=blue, opacity=.6] 
    ({\xstart},{\ystart},0)  -- ({\xstart - \balkenbreite * cos(\drehwinkel)- (-\balkenbreite+0)*sin(\drehwinkel)},{\ystart - \balkenbreite * sin(\drehwinkel)+ (-\balkenbreite+0)*cos(\drehwinkel)},0)  -- ({\xstart - \balkenbreite * cos(\drehwinkel)- (\balkenhoehe+0.5)*sin(\drehwinkel)},{\ystart - \balkenbreite * sin(\drehwinkel)+ (\balkenhoehe+0.5)*cos(\drehwinkel)},0) -- ({\xstart - 0 * cos(\drehwinkel)- (\balkenhoehe+0)*sin(\drehwinkel)},{\ystart - 0 * sin(\drehwinkel)+ (\balkenhoehe+0)*cos(\drehwinkel)},0) -- cycle;
        
\path[ shade, right color=white, left color=blue, opacity=.6] 
	({\xstart},{\ystart},0)  -- ({\xstart - \balkenbreite * cos(\drehwinkel)- (-\balkenbreite+0)*sin(\drehwinkel)},{\ystart - \balkenbreite * sin(\drehwinkel)+ (-\balkenbreite+0)*cos(\drehwinkel)},0) --
    ({\xstart + (\balkenlaenge+0.5) * cos(\drehwinkel)- (-\balkenbreite+0)*sin(\drehwinkel)},{\ystart + (\balkenlaenge+0.5) * sin(\drehwinkel)+ (-\balkenbreite+0)*cos(\drehwinkel)},0) --   
    ({\xstart + \balkenlaenge * cos(\drehwinkel)},{\ystart + \balkenlaenge * sin(\drehwinkel)},0)--
    cycle;       
%---------End Balken----------
%Punkte Definition
\node (pointo) at ({\xstartdraw},{\ystartdraw}) {};
\node (pointostrich) at ({\xstartdraw+2*\xlength},{\ystartdraw}) {};
\node (pointmiddle) at ({\xstartdraw+\xlength},{\ystartdraw}) {};
\node (pointa1) at ({\xstartdraw+\xlength},{\ystartdraw+\ylength}) {};
\node (pointa2) at ({\xstartdraw+\xlength},{\ystartdraw-\ylength}) {};

%Vektoren
\draw[-{>[scale=1,length=10,width=6]},shorten >=4pt, shorten <=4pt,line width=1pt,color=blue] (pointo) -- (pointostrich) node[xshift=5, yshift=-30]{$(a_{1}-o)+(a_{2}-o)$} ;
\draw[-{>[scale=1,length=10,width=6]},shorten >=4pt, shorten <=4pt,line width=1pt,color=blue] (pointo) -- (pointa1) node[midway, left]{$a_{1}-o$} ;
\draw[-{>[scale=1,length=10,width=6]},shorten >=4pt, shorten <=4pt,line width=1pt,color=blue] (pointo) -- (pointa2) node[midway, below]{$a_{2}-o$} ;
\draw[-{>[scale=1,length=10,width=6]},shorten >=4pt, shorten <=4pt,line width=1pt,color=blue] (pointo) -- (pointmiddle) node[xshift=10, yshift=-28]{$(a_{1}-o)\frac{1}{2}+(a_{2}-o)\frac{1}{2}$};
%Hilfslinien
\draw[-,shorten >=3pt, shorten <=3pt,line width=0.3pt,color=blue] (pointa1) -- (pointostrich) ;
\draw[-,shorten >=3pt, shorten <=3pt,line width=0.3pt,color=blue] (pointa2) -- (pointostrich) ;

%Punkte malen
\draw[fill,color=red] (pointo) circle [x=1cm,y=1cm,radius=0.08]node[ xshift=-10]{$o$};
\draw[fill,color=red] (pointostrich) circle [x=1cm,y=1cm,radius=0.08]node[xshift=10]{$o'$};
\draw[fill,color=red] (pointa1) circle [x=1cm,y=1cm,radius=0.08]node[ xshift=-10]{$a_{1}$};
\draw[fill,color=red] (pointa2) circle [x=1cm,y=1cm,radius=0.08]node[ yshift=-10]{$a_{2}$};
\draw[fill,color=red] (pointmiddle) circle [x=1cm,y=1cm,radius=0.08]node[xshift=-8, yshift=15]{$a_{1}\frac{1}{2}+a_{2}\frac{1}{2}$};
\end{tikzpicture}
\end{figure}
%-------------------------End Grafik Affinkombination-------------------------------------- 

\subparagraph{Beispiel: }
	Sind etwa $ K=\mathbb{R} $ und $ I = \{1,...,n\} $, so erhält man mit $ x_i = \frac{1}{n} $ für $ {i\in I} $ den üblichen geometrischen Schwerpunkt der (endlichen) Punktmenge,
		\[ s =\sum_{i=1}^{n}a_i\frac{1}{n}. \]
	Achtung: Die Ausdrücke
		\[ \frac{\sum_{i =1}^{n}a_i}{n}\text{ oder } \frac{1}{n}\sum_{i=1}^{n}a_i \]
	sind sinnlos, da nicht definiert.
\subparagraph{Beweis: }
	Zu zeigen: Sind $ o,o'\in A $, so gilt
	\[ o'+\sum_{i\in I} v'_ix_i = o+\sum_{i\in I} v_ix_i \text{, wobei }
		\begin{cases}
		v_i := a_i-o\\
		v'_i := a_i-o'
		\end{cases}\]
	Zunächst bemerken wir, dass mit $ w:= o'-o $ für $ {i\in I} $ gilt: $ v'_i+w=v_i $, denn:
	\begin{gather*}
		\tau_{v'_i+w}(o) = \tau_{v'_i}(\tau_w(o)) = \tau_{v'_i}(o')\\
		= a_i = \tau_{v_i}(o),
	\end{gather*}
	also
	\begin{gather*}
		o+\sum_{i\in I}v_ix_i = o+\sum_{i\in I} (w+v'_i)_{x_i} = o+ \sum_{i\in I}wx_i + \sum_{i\in I}v'_ix_i\\
		= o+ w\cdot \sum_{i\in I}x_i + \sum_{i\in I}v'_i x_i = o+w + \sum_{i\in I}v'_ix_i = o' + \sum_{i\in I}v'_i x_i
	\end{gather*}
\paragraph{Lemma: }
	\begin{Lemma}{}
		Ist $ S\subset A $ Teilmenge eines AR $ A $, so ist ihre affine Hülle
		\[ [S] = \{\sum_{a\in S}ax_a\mid \#\{a\in S\mid x_a\neq 0\}<\infty \land \sum_{a\in S}x_a = 1 \}. \]
	\end{Lemma}
\subparagraph{Beweis: }
	Wir setzen $ S\neq \emptyset $ voraus und wählen $ o\in S $, dann ist
	\[ [S] = o+[\{a-o\mid a\in S\}] \]
	und die Behauptung folgt aus der entsprechenden für die lineare Hülle.
\subparagraph{Beispiel: }
	Die affine Hülle zweier Punkte $ a,b\in A, a\neq b $ ist die (affine) Gerade
	\[ [ab] := [\{a,b\}] = \{a(1-t)+bt\mid t\in K\}. \]
	Die affine Hülle von drei verschiedenen Punkten $ a,b,c\in A $ ist eine Gerade oder Ebene, je nachdem, ob $ \dim[\{a,b,c\}] $ gleich 1 oder 2 ist. Im zweiten Fall sagen wir: das Dreieck $ \{a,b,c\} $ sei nicht-degeneriert.
\paragraph{Definition: }
	\begin{Definition}[Allgemeine Lage]
		Eine Familie $ (a_i)_{i\in I} $ von Punkten $ a_i\in A $ eines AR $ A $ ist affin unabhängig, bzw. in allgemeiner Lage, falls
		\[ \forall i\in I:a_i\notin [\{a_j\mid j\in I\setminus \{i\}\}], \]
		und sonst affin abhängig; Punkte heißen kolinear bzw. koplanar, falls sie in einer Geraden oder einer Ebene liegen.
	\end{Definition}
\paragraph{Lemma}
	\begin{Lemma}[Affine und lineare (Un-)Abhängigkeit]
		Eine Familie $ (a_i)_{i\in I} $ ist genau dann affin unabhängig, wenn für jedes $ i\in I $ die Familie $ (a_j-a_i)_{j\in I\setminus \{i\}} $ linear unabhängig ist.
	\end{Lemma}
\subparagraph{Beweis: }
	Die Familie $ (a_i)_{i\in I} $ ist genau dann affin abhängig, wenn
	\begin{gather*}
		\exists i\in I:a_i\in [\{a_j\mid j\in I\setminus \{i\}\}] \Leftrightarrow \exists i\in I\exists(x_j)_{j\in I\setminus \{i\}}:a_i=\sum_{j\neq i}a_jx_j\land 1=\sum_{j\neq i}x_j\\
		\Leftrightarrow \exists i\in I\exists (x_j)_(j\in I\setminus \{i\}):0=\sum_{j\neq i}(a_j-a_i)x_j \land 1=\sum_{j\neq i}x_j,
	\end{gather*}
	d.h., wenn die Familie $ (a_j-a_i)_{j\in I\setminus \{i\}} $ eine nicht-triviale Linearkombination von 0 erlaubt, also linear abhängig ist.
\paragraph{Lemma: }
	\begin{Lemma}
		Eine Familie $ (a_i)_{i\in I} $ ist genau dann affin unabhängig, wenn jeder Punkt ihrer affinen Hülle eine eindeutige Affinkombination hat:
		\[ \forall a\in [\{a_i\mid i\in I\}]\exists!(x_i)_{i\in I}:
			\begin{cases}
			1 = \sum_{i\in I}x_i\\
			a = \sum_{i\in I}a_ix_i
			\end{cases}\]
	\end{Lemma}
	\subparagraph{Beweis: }
	Hat jeder Punkt $ a\in [\{a_i\mid i\in I\}] $ eine eindeutige Affinkombination, so gilt insbesondere
		\[ \forall i\in I: a_i = a_i\cdot 1 \notin [\{a_j\mid j\neq i\}]. \]
	Hat andererseits der Punkt $ a $ zwei Affindarstellungen,
		\[ a = \sum_{i\in I} a_ix_i = \sum_{i\in I}a_iy_i, \]
	so folgt mit einem Ursprung $ o\in A $ und $ v_i = a_i-o $
		\[ a=o+\sum_{i\in I}v_ix_i=o+\sum_{i\in I}v_iy_i \Rightarrow 0 = \sum_{i\in I}v_i(y_i-x_i). \]
	Ist $ (a_i)_{i\in I} $ affin unabhängig, so ist $ (v_j)_{j\in I\setminus \{i\}} $ linear unabhängig für ein beliebiges $ i\in I $ und $ o:= a_i $. Es folgt:
	\begin{gather*}
        \forall j\in I\setminus \{i\}:x_j=y_j \Rightarrow x_i = 1-\sum_{j\neq i}x_j = 1-\sum_{j\neq i}y_j = y_i 
        \\ \text{ also } (x_{i})_{i \in I} = (y_{i})_{i \in I}
	\end{gather*}
       