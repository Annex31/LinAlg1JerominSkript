\section{Dreiecke in der Affinen Geometrie}
\subsection{Beispiel \& Definition}
	\begin{Definition}[Mittelpunkt]
	Der (geometrische) Schwerpunkt zweier Punkte $ a,b\in A $ eines affinen Raumes über dem Körper $ K $ ist ihr Mittelpunkt
		\[ s_{a,b} = a\cdot \frac{1}{2}+b\cdot \frac{1}{2}. \]
	\end{Definition}
	
	Dies ist sinnlos, falls $ \Char K = 2 $ ist, was wir also ausschließen müssen.
	Ist etwa $ A $ AR über $ K=\mathbb{Z}_2 $, so enthält jede Gerade genau zwei Punkte,
		\[ \forall a,b\in A: [ab] = 
			\begin{cases}
				\{a,b\},& \text{falls }a\neq b\\
				\{a\},& \text{falls } a=b
			\end{cases} \]
	Für den Rest des Kapitels wird $ \Char K \neq 0 $ ausgeschlossen.
\paragraph{Bemerkung}
	Ist $ K $ ein geordneter Körper, e.g. $ K=\mathbb{R} $, so kann man die \emph{Strecke}
		\[ \overline{ab}:= \{a(1-s)+bs\mid 0\leq s\leq 1\} \]
	zwischen zwei Punkten $ a,b\in A $ definieren. Jeder Punkt $ c\in \overline{ab} $ auf der Strecke liegt \emph{zwischen} ihren \emph{Endpunkten} $ a $ und $ b $; $ s_{ab} $ ist dann auch Mittelpunkt der Strecke $ \overline{ab} $.
	
	Offenbar ist das sinnlos, wenn der Körper $ K $ nicht angeordnet ist.
	
	
\subsection{Schwerpunktsatz}
	\begin{Satz}[Schwerpunktsatz]
	Der Schwerpunkt eines nicht-degenerierten Dreieck $ \{a,b,c\}\subset A $ ist der Schnittpunkt der Seitenhalbierenden, die er im Verhältnis $ -\frac{1}{2} $ teilt.
	\end{Satz}
% % % Grafik Schwerpunktsatz
	\begin{figure}[H]\centering
	\definecolor{uququq}{rgb}{0.25,0.25,0.25}
	\definecolor{zzttqq}{rgb}{0.6,0.2,0}
	\definecolor{qqqqff}{rgb}{0,0,1}
		\begin{tikzpicture}[line cap=round,line join=round,>=triangle 45,scale=1.5] %,x=1.0cm,y=1.0cm]
		\clip(0.71,0.52) rectangle (5.68,4.35);
		\coordinate (a) at (1.36,2.12);
		\coordinate (b) at (3,4);
		\coordinate (c) at (4.56,0.82);
		\fill[color=zzttqq,fill=zzttqq,fill opacity=0.1] (a) -- (b) -- (c) -- cycle;
		\draw [color=zzttqq] (a)-- (b);
		\draw [color=zzttqq] (b)-- (c);
		\draw [color=zzttqq] (c)-- (a);
		
		% Halbierungspunkte:
		\coordinate (Sab) at (2.18,3.06);
		\coordinate (Sbc) at (3.78,2.41);
		\coordinate (Sac) at (2.96,1.47);
		\coordinate (S) at (2.97,2.31);
		
		\fill [color=qqqqff] (a) circle (1.5pt);
		\draw[color=qqqqff] (a) node[left] {$a$};
		\fill [color=qqqqff] (b) circle (1.5pt);
		\draw[color=qqqqff] (b) node[above] {$b$};
		\fill [color=qqqqff] (c) circle (1.5pt);
		\draw[color=qqqqff] (c) node[right] {$c$};
		\fill [color=uququq] (Sab) circle (1.5pt);
		\draw[color=uququq] (Sab) node[left] {$S_{ab}$};
		\fill [color=uququq] (Sbc) circle (1.5pt);
		\draw[color=uququq] (Sbc) node[right] {$S_{bc}$};
		\fill [color=uququq] (Sac) circle (1.5pt);
		\draw[color=uququq] (Sac) node[below] {$S_{ac}$};
		\fill [color=uququq] (S) circle (1.5pt);
		\draw[color=uququq] (S) node[below left] {$S$};
		\draw [dash pattern=on 2pt off 2pt] (Sbc)-- (a);
		\draw [dash pattern=on 2pt off 2pt] (Sac)-- (b);
		\draw [dash pattern=on 2pt off 2pt] (Sab)-- (c);
		\end{tikzpicture}
	\end{figure}
\paragraph{Beweis}
	Der (geometrische) Schwerpunkt des Dreiecks $ \{a,b,c\}\subset A $ ist
		\[ s = a\cdot \frac{1}{3}+ b\cdot \frac{1}{3}+ c\cdot \frac{1}{3} = (a\cdot \frac{1}{2}+b\frac{1}{2})\frac{2}{3}+c\cdot \frac{1}{3} = s_{ab}\cdot\frac{2}{3}+c\cdot \frac{1}{3} \in [s_{ab}c];\]
	weiters gilt
			\[ (s_{ab}s:cs)= -\frac{\frac{1}{3}}{1-\frac{1}{3}} = -\frac{1}{2}, \]
	$ s $ teilt die Strecke $ \overline{s_{ab}c} $ im Verhältnis $ -\frac{1}{2} $. Aus Symmetriegründen gelten diese Resultate genauso für die anderen Seitenhalbierenden.
\paragraph{Bemerkung}
	Andere bekannte Schnittsätze im Dreieck machen in der affinen Geometrie keinen Sinn. Sätze wie der Höhensatz oder über den Umkreismittelpunkt können gar nicht erst formuliert werden: in der affinen Geometrie kennt man weder Längen- noch Winkelmessung.
	
	Dem gegenüber sind die Sätze von Menelaos und Ceva "`affine Sätze"', d.h. sie können rein affin formuliert werden und beschreiben unter affinen Transformationen \emph{invariante} Sachverhalte.
\subsection{Bemerkung \& Definition}
	Sind $ \alpha:A\to A' $ und $ \beta:A'\to A'' $ affine Abbildungen und bezeichnen $ \lambda:V\to V' $ bzw. $ \mu:V'\to V'' $ ihre linearen Anteile,
		\[ \forall a\in A\forall v\in V:\alpha(a+v) = \alpha(a)+\lambda(v)\text{ und }\forall a'\in A'\forall v'\in V':\beta(a'+v') = \beta(a')+\mu(v'), \]
	so gilt für ihre Komposition
		\[ (\beta\circ\alpha)(a+v) = \beta(\alpha(a)+\lambda(v)) = \beta(\alpha(a))+\mu(\lambda(v)) = (\beta\circ\alpha)(a)+(\mu\circ\lambda)(v), \]
	d.h. der lineare Anteil einer Komposition von affinen Abbildungen ist die Komposition der linearen Anteile.
	
	\begin{Definition}[Dilationsgruppe]
	Da eine affine Transformation, deren linearer Anteil Vielfaches der Identität ist, eine Translation oder eine Streckung ist, bilden die Translationen und Streckungen eines affinen Raumes eine Gruppe, die \emph{Dilationsgruppe}.
	\end{Definition}
	
\subsection{Satz von Menelaos}
	\begin{Satz}[Satz von Menelaos]
		Seien $ \{a,b,c\}\subset A $ ein nicht-degeneriertes Dreieck und $ g\subset A $ eine Gerade, die die drei Seiten des Dreiecks außerhalb der Ecken des Dreiecks schneidet;
			\[ a'\in g\cap [bc],b'\in g\cap [ca] \text{ und }c'\in g\cap [ab] \]
		bezeichne die Schnittpunkte. Dann gilt:
			\[ (ac':bc')(ba':ca')(cb':ab') = 1 \]
		Umgekehrt garantiert die TV-Bedingung, dass drei Punkte $ a'\in [bc],b'\in [ca] $ und $ c'\in [ab] $ auf den Seiten des Dreiecks kollinear sind.
	\end{Satz}
	
	\begin{figure}[H]\centering
	\definecolor{zzttqq}{rgb}{0.6,0.2,0}
	\definecolor{qqqqff}{rgb}{0,0,1}
	\begin{tikzpicture}[line cap=round,line join=round,>=triangle 45,x=1.0cm,y=1.0cm]
	\clip(1.05,-0.85) rectangle (9.76,4.01);
	\fill[color=zzttqq,fill=zzttqq,fill opacity=0.1] (2.1,0.26) -- (6.42,0.81) -- (4.5,3) -- cycle;
	\draw [color=zzttqq] (2.1,0.26)-- (6.42,0.81);
	\draw [color=zzttqq] (6.42,0.81)-- (4.5,3);
	\draw [color=zzttqq] (4.5,3)-- (2.1,0.26);
	\draw [domain=1.05:9.76] plot(\x,{(-0.02--0.55*\x)/4.32});
	\draw [domain=1.05:9.76] plot(\x,{(-13.58--1.45*\x)/-3.33});

	\fill [color=qqqqff] (2.1,0.26) circle (1.5pt);
	\draw[color=qqqqff] (2.14,0.33) node {$A$};
	\fill [color=qqqqff] (6.42,0.81) circle (1.5pt);
	\draw[color=qqqqff] (6.46,0.88) node {$B$};
	\fill [color=qqqqff] (4.5,3) circle (1.5pt);
	\draw[color=qqqqff] (4.54,3.07) node {$C$};
	\fill (3.94,2.36) circle (1.5pt);
	\draw (3.99,2.43) node {$b'$};
	\fill (7.28,0.92) circle (1.5pt);
	\draw (7.33,0.99) node {$c'$};
	\draw[color=black] (1.33,3.19) node {$g$};
	\fill (5.74,1.59) circle (1.5pt);
	\draw (5.78,1.66) node {$a'$};
	\end{tikzpicture}
	\end{figure}
	
\paragraph{Beweis}
	Betrachte Streckung $ \gamma $ mit Zentrum $ c' $ und Faktor $ s_{ab}\in K^\times $,
		\[ \gamma:A\to A, c'+ v\mapsto \gamma(c'+v) := c'+vs_{ab}; \]
	insbesondere ist für $ s_{ab}=\frac{1}{(ac':bc')} $
		\[ \gamma(a)=c'+(a-c')\frac{1}{(ac':bc')}=a'+(b-c') = b. \]
	Definiert man Streckungen $ \alpha $ und $ \beta $ entsprechend, mit Zentren $ a' $ bzw. $ b' $ und Faktoren $ s_{bc} = \frac{1}{(ba':ca')} $ bzw. $ s_{ca}=\frac{1}{cb':ab'} $, so liefert die Komposition eine affine Transformation
		\[ \delta:= \beta\circ\alpha\circ\gamma:A\to A:,a+v \mapsto \delta(a+v) := a+vs_{ab}s_{bc}s_{ca}, \]
	da
		\[ a\overset{\gamma}{\mapsto}b\overset{\alpha}{\mapsto}c\overset{\beta}{\mapsto}a. \]
	Damit gilt
		\[ (ac':bc')(ba':ca')(cb':ab') = 1 \Leftrightarrow \delta = \id_A \]
	Wegen $ a\notin [c'a'] $ ist andererseits
		\[ \delta = \id_A \Leftrightarrow [c'a'] = \delta([c'a']) = \beta([c'a']), \]
	da $ \gamma([c'a']) = [c'a'] $ und $ \alpha([c'a']) = [c'a'], $ was die letzte Gleichung liefert, damit ist
		\[ \delta = \id_A \Leftrightarrow [c'a']=\beta([c'a'])\Leftrightarrow b'\in [c'a'], \]
	da $ \beta $ Streckung mit Zentrum $ b' $ ist. Damit ist die Behauptung bewiesen.
\subsection{Satz von Ceva}
	\begin{Satz}[Satz von Ceva]
		Seien $ \{a,b,c\}\subset A $ ein nicht-degeneriertes Dreieck und
			\[ a'\in [bc]\setminus \{b,c\},b'\in [ac]\setminus \{a,c\},c'\in [ab]\setminus \{a,b\}. \]
		Schneiden sich die drei Transversalen $ [aa'],[bb'] $ und $ [cc'] $ in einem Punkt, so gilt
			\[ (ac':bc')(ba':ca')(cb':ab')=-1. \]
	\end{Satz}
	\begin{figure}[H]\centering
	\definecolor{zzttqq}{rgb}{0.6,0.2,0}
	\definecolor{qqqqff}{rgb}{0,0,1}
	\begin{tikzpicture}[line cap=round,line join=round,>=triangle 45,x=1.0cm,y=1.0cm]
	\clip(0.25,0.33) rectangle (9.57,6.25);
	\fill[color=zzttqq,fill=zzttqq,fill opacity=0.1] (1.5,1.52) -- (6.8,1.78) -- (4.08,5.58) -- cycle;
	\draw [color=zzttqq] (1.5,1.52)-- (6.8,1.78);
	\draw [color=zzttqq] (6.8,1.78)-- (4.08,5.58);
	\draw [color=zzttqq] (4.08,5.58)-- (1.5,1.52);
	\draw [domain=0.25:9.57] plot(\x,{(--15.39-1.1*\x)/4.43});
	\draw [domain=0.25:9.57] plot(\x,{(-13.46--3.96*\x)/0.48});
	\draw [domain=0.25:9.57] plot(\x,{(--1.82--1.03*\x)/2.21});
	\fill [color=qqqqff] (1.5,1.52) circle (1.5pt);
	\draw[color=qqqqff] (1.58,1.64) node {$a$};
	\fill [color=qqqqff] (6.8,1.78) circle (1.5pt);
	\draw[color=qqqqff] (6.88,1.9) node {$b$};
	\fill [color=qqqqff] (4.08,5.58) circle (1.5pt);
	\draw[color=qqqqff] (4.14,5.7) node {$c$};
	\fill (2.37,2.88) circle (1.5pt);
	\draw (2.45,3.01) node {$b'$};
	\fill (3.6,1.62) circle (1.5pt);
	\draw (3.68,1.75) node {$c'$};
	\fill (5.62,3.43) circle (1.5pt);
	\draw (5.7,3.56) node {$a'$};
	\end{tikzpicture}
	\end{figure}
	Beweis in Aufgabe 59.
\paragraph{Bemerkung}
	Für die Seitenmitten gilt der Satz (Schwerpunktsatz).