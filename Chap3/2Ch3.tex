\section{Lineare Gleichungssysteme}
	Mission: Viele Probleme in Anwendungen oder Naturwissenschaften werden zu "`linearen Problemen"' reduziert, d.h. auf lineare Gleichungssysteme unterschiedlicher Komplexität.
	Diese Reduktion ist etwa eine wichtige Aufgabe der Analysis; Aufgabe der linearen Algebra ist dann die Lösung bzw. Strukturanalyse.
\subsection{Definition}
	\begin{Definition}[Lineares Gleichungssystem]
	Ein lineares Gleichungssystem ist ein System von $ m $ Gleichungen
		\[ \begin{array}{cccc}
		a_{11}x_1+&\dots &+ a_{1n}x_n &=y_1\\
		\vdots & &\vdots & \vdots\\
		a_{m1}x_1 +& \dots &+a_{mn}x_n &= y_m
		\end{array} \]
	für $ n $ Unbekannte $ x_1,\dots,x_n\in K $, wobei die Parameter $ a_{ij},y_i\in K $ gegeben sind. Ist $ y_1 = \dots = y_m = 0 $, so heißt das System homogen, anderenfalls inhomogen.
	\end{Definition}
\paragraph{Bemerkung}
	Mit Matrizen $ A\in K^{m\times n},X\in K^{n\times 1} $ und $ Y\in K^{m\times 1} $ lässt sich ein lineares Gleichungssystem kompakter schreiben als
		\[ AX = Y \]
	Die Standardbasen $ E $ und $ E' $ von $ K^n $ bzw. $ K^m $ liefern den Isomorphismus
		\[ K^{m\times n}\ni A\mapsto f_a\in \hom(K^n,K^m)\text{, wobei }f_A(E) = E'A, \]
	damit lässt sich die Gleichung umformulieren als Gleichung eines affinen Unterraumes von $ K^n: $
		\[ f_A(x) = y \text{ mit } x=EX \text{ und } y=E'Y. \]
	Nämlich: Existiert eine Lösung $ x\in f_A^{-1}(\{y\})\neq \emptyset $, so ist der Lösungsraum
		\[ f_A^{-1}(\{y\}) = x+\ker f_A\subset K^n \]
	ein affiner Unterraum.
	
	Das nächste Lemma folgt dann mit dem Basisisomorphismus:
		\[ K^{n\times 1} \ni X \mapsto EX =: x\in K^n \]
\subsection{Definition \& Lemma}
	\begin{Lemma}[Lösungsraum]
	Der Lösungsraum $ L_{A,Y} $ eines linearen Gleichungssystems,
		\[ L_{A,Y}:=\{X\in K^{n\times 1}\mid AX=Y\}\subset K^{n\times 1} \]
	ist leer oder ein affiner Unterraum der Dimension $ k = n-\rg A $.
	
	Ist $ Y = 0 $, so gilt $ 0\in L_{A,Y} $ und $ L_{A,Y}\subset K^{n\times 1} $ ist ein linearer Unterraum (UVR).
	\end{Lemma}