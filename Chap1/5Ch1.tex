%VO_10.11.15
\section{Summen, Produkte und Quotienten}
\subsection{Definition (Summe von UVR)}
	\begin{Definition}[Summe von UVR]
		Die Summe einer Familie $ (U_i)_{i\in I} $ von UVR $ U\subset V $ eines $ K $-VR ist die Menge
		\begin{equation*}
		\sum_{i\in I} U_i := \{\sum_{i \in I}u_i\mid \forall i\in I: u_i\in U_i \land \# \{i\in I\mid u_i \neq 0\}<\infty\}.
		\end{equation*}
	\end{Definition}
		
\paragraph{Bemerkung}
	Offenbar ist $ \sum_{i\in I} U_i\subset V $ UVR mit
	\[  \bigcup_{i\in I}U_i \subset \sum_{i\in I} U_i \Rightarrow [\bigcup_{i\in I}U_i]\subset \sum_{i\in I} U_i; \]
	andererseits gilt:
	\[ \sum_{i\in I}U_i \subset \{\sum_{j\in J}v_jx_j\mid \forall j\in Jv_j\in \bigcup_{i\in I}U_i \land \#\{j\in J\mid x_j\neq 0\}<\infty\}\subset [\bigcup_{i\in I}U_i]. \]
	
	Damit ist die Summe einer Familie $ (U_i)_{i\in I} $ gerade die lineare Hülle ihrer Vereinigung $ \bigcup_{i\in I}U_i $,
	\[ \sum_{i\in I}U_i= [\bigcup_{i\in I}U_i]. \]
		
\paragraph{Beispiel}
	Sei $ V=\mathbb{R}^\mathbb{N} $ der Raum der reellen Folgen. Für $ n\in \mathbb{N} $ setze
		\begin{equation*}
		U_n := \{v\in \mathbb{R}^\mathbb{N}\mid \forall j\in \mathbb{N}: j>n\Rightarrow v_j = 0 \} \subset \mathbb{R}^\mathbb{N};
		\end{equation*}
		
	dann gilt $ \forall n\in \mathbb{N}: U_n\subset U_{n+1} $, und damit auch
		\begin{gather*}
		\sum_{i\leq n} U_i = U_n = \bigcup_{i\leq n}U_i \text{, aber}\\
		\sum_{i\in \mathbb{N}}U_i = \bigcup_{i\in \mathbb{N}}U_i \neq V.
		\end{gather*}
		
	Nun setze für $ i\in \{0,1\} $
		\begin{equation*}
		\tilde{U}_i := \{v\in \mathbb{R}^\mathbb{N}\mid \forall j\in \mathbb{N}: j=i\operatorname{mod} 2\Rightarrow v_j = 0\}
		\end{equation*}
		
	dann ist 
		\begin{equation*}
		\bigcup_{i\in \{0,1\}}\tilde{U}_i \neq \sum_{i\in \{0,1\}}\tilde{U}_i = V.
		\end{equation*}
		
\subsection{Dimensionssatz}
	\begin{Satz}[Dimensionssatz]
	Sind $ U_i \subset V $ UVR mit $ \dim U_i < \infty $ für $ i\in \{1,2\} $, so ist
		\[ \dim (U_1+U_2) + \dim (U_1\cap U_2) = \dim U_1 + \dim U_2. \]	
	Ist $ \dim U_1 = \infty$ oder $ \dim U_2=\infty $, so ist auch $ \dim (U_1+U_2)=\infty $.
	\end{Satz}
	
\paragraph{Beweis}
	Seien
		\begin{itemize}
		\item $ B_0 \subset U_1\cap U_2 $ eine Basis von $ U_0 := U_1\cap U_2 $;
		\item $ S_i \subset U_i $ lin. unabh., sodass $ B_i = B_0 \cup S_i $ Basen von $ U_i $ sind ($ i = 1,2; $ BES).
		\end{itemize}
	
	Offenbar gilt dann, da $ B_i = B_0\cup S_i $ lin. unabh. sind,
		\begin{equation*}
		B_0\cap S_1 = \emptyset \text{ und } B_0\cap S_2 = \emptyset
		\end{equation*}
	und 
		\begin{equation*}
		S_1\cap S_2 \subset U_1\cap U_2 = [B_0] \Rightarrow S_1\cap S_2 = \emptyset.
		\end{equation*}
		
	Wir zeigen, dass $ B:= B_0\cup S_1\cup S_2 $ Basis von $ U_1 + U_2 =: U $ ist.
	
	$ B\subset U $ ist Erz. Syst. nach Konstruktion:
		\begin{gather*}
		\forall i\in \{1,2\} : U_i=[B_i]\subset [B]\\
		\Rightarrow U_1+U_2 = [U_1\cup U_2]\subset [B]
		\end{gather*}
	
\subparagraph{$ B $ ist linear unabhängig:}
	Gegeben sei eine Linearkombination von $ 0\in U $,
		\begin{gather*}
		0 = \sum_{b\in B}bx_b = \sum_{b\in B_0}bx_b + \sum_{b\in S_1}bx_b + \sum_{b\in S_2}bx_b =: b_0 + s_1+ s_2\\
		\text{mit } b_0\in [B_0] = U_0 \text{ und } s_i\in [S_i] \text{ für } i= 1,2;
		\end{gather*}
		
	dann gilt etwa, $ B_1 = B_0 \cup S_1 $ lin. unabh.,
		\begin{equation*}
		b_0+s_1 = -s_2 \in U_1\cap [S_2]\subset U_0 \Rightarrow s_1 = 0
		\end{equation*}
		
	und damit, da $ B_2 = B_0 \cup S_2 $ lin. unabh. ist,
		\begin{equation*}
		0 = b_0 + s_1 + s_2 \Rightarrow b_0=s_2 = 0.
		\end{equation*}
	
	Mit der linearen Unabhängigkeit von $ B_0, S_1 $ und $ S_2 $ folgt dann
		\begin{equation*}
		0 = \sum_{b\in B_0}bx_b = \sum_{b\in S_1}bx_b = \sum_{b\in S_2}bx_b \Rightarrow \forall b\in B: x_b = 0.
		\end{equation*}
	
	Mit
		\begin{gather*}
		\#B + \#B_0 = (\#B_0 + \#S_1 + \#S_2) + \#B_0\\
		= (\#B_0+\#S_1)+(\#B_0 + \#S_2) = \#B_1+\#B_2
		\end{gather*}
	
	folgt dann die Behauptung.

\paragraph{Bemerkung}
	Im Beweis haben wir benutzt: 
	
	Ist z.B. $ B_1 = B_0\cup S_1 $ lin. unabh., und $ b_0\in [B_0] $ und $ s_1\in [S_1] $ mit $ b_0 + s_1 = 0 $, so folgt $ b_0 = s_1 = 0 $:
	sind nämlich $ b_0 = \sum_{b\in B_0} bx_b $ und $ s_1 = \sum_{b\in S_1}bx_b $, so gilt 
		\begin{equation*}
		0 = b_0+s_1 = \sum_{b\in B_0} bx_b+\sum_{b\in S_1} bx_b = \sum_{b\in B_1} bx_b \Rightarrow \forall b\in B_1: x_b = 0\Rightarrow b_0 = s_1 = 0.
		\end{equation*}

\paragraph{Bemerkung}
	Ist $ U_1\cap U_2 = \{o\} $ bzw. $ \dim (U_1\cap U_2)  = 0 $, so zeigt der Beweis auch:
		\begin{equation*}
		\forall v\in U_1+U_2\exists ! u_1 \in U_1\exists ! u_2\in U_2: v= u_1+u_2
		\end{equation*}

\subsection{Definition (Komplementäre UVR)}
	\begin{Definition}[Komplementäre UVR]
		Zwei UVR $ U_1,U_2 \subset V $ heißen komplementär in $ V $, falls
		\begin{equation*}
		U_1+U_2=V\text{ und } U_1\cap U_2 = \{o\}.
		\end{equation*}
	\end{Definition}
		
\subsection{Lemma (Komplementäre UVR)}
	\begin{Lemma}[Komplementäre UVR]
		Zu jedem UVR $ U\subset V $ existiert ein (in $ V $) komplementärer UVR.
	\end{Lemma}
	
\paragraph{Beweis}
	Sei $ U\subset V $ UVR eines $ K $-VR $ V $.
	Seien 
		\begin{itemize}
		\item $ B\subset U $ eine Basis von $ U $;
		\item $ S\subset V $ lin. unahb., sodass $ C=B\cup S $ Basis von $ V $ ist (BES).
		\end{itemize}
	
	Definiere $ U':= [S] $. Dann ist $ U'\subset V $ UVR mit
		\begin{enumerate}[(i)]
		% TODO: wir zeigen, dass U+U' übermenge ist, wollen aber Gleichheit zeigen ... wo ist die andere inklusion?
		\item $ U+U' \supset [C] = V $, da $ C\subset U\cup U' $ Erz. Syst. von $ V $ ist;
		\item $ U\cap U' = [B]\cap [S] = \{o\}$, da $ C=B\cup S $ linear unabhängig ist.
		\end{enumerate}
		
\paragraph{Bemerkung}
	Zu einem UVR $ U\subset V $ gibt es normalerweise viele komplementäre UVR $ U'\subset V $.
	Z.B.: Zu
		\begin{equation*}
		U:= \{v\in K^2\mid v_2 = 0\}
		\end{equation*}
	
	ist jeder UVR $ U' = [u']$ mit $u'_2\neq 0 $ komplementär in $ K^2 $.
	
\subsection{Lemma \& Definition (direkte Summe)}
	\begin{Definition}[Direkte Summe]
	Sei $ U= \sum_{i\in I}U_i\subset V $ Summe einer Familie von UVR $ U_i\in V $; dann besitzt jeder Vektor $ u\in U $ eine eindeutige Zerlegung als Summe von $ u_i $, genau dann, wenn
		\begin{equation*}
		\forall i\in I: U_i\cap \sum_{j\in I\setminus \{i\}}U_j = \{o\}.
		\end{equation*}
		
	In diesem Falle heißt die Summe "`direkt"' und man schreibt
	\[ U = \bigoplus_{i\in I} U_i. \]
	\end{Definition}
		
\paragraph{Bemerkung}
	Eine Summe $ V = \sum_{i\in I} U_i $ ist genau dann direkt, wenn
		\begin{equation*}
		\forall i\in I: U_i, \sum_{j\in I\setminus\{i\}}U_j \subset V
		\end{equation*}
		
	komplementäre UVR in $ V $ sind.

\paragraph{Beweis}
	Zu zeigen ist die Eindeutigkeitsaussage. Sei also $ u \in \bigoplus_{i\in I}U_i $,
		\begin{gather*}
		u = \sum_{i\in I} u_i = \sum_{i\in I} u_i' \text{ mit } \forall i\in I: u_i,u'_i\in U_i;
		\end{gather*}
	dann gilt für jedes $ i\in I$:
		\begin{equation*}
		u_i-u'_i = \sum _{j\neq i}u_j-\sum_{j\neq i} u'_j = \sum_{j\neq i}u_j-u'_j \in U_i\cap \sum_{j\neq i} U_j = \{o\},
		\end{equation*}
	da die Summe als direkt angenommen wurde; damit folgt $ \forall i \in I: u_i = u'_i $, d.h. die Zerlegung ist eindeutig.
	
	Die Umkehrung ist trivial:
		\begin{gather*}
		\exists i\in I:U_i\cap \sum_{j\neq i} U_j \neq \{o\} \Rightarrow \exists i\in I\exists u_i\in U_i\setminus\{o\}\exists (u_j)_{j\in I\setminus\{i\}}:\\
		(\forall j\in I\setminus\{i\}:u_j \in U_j)\land u_i = \sum_{j\neq i} u_j,
		\end{gather*}
		
	d.h., die Zerlegung von $ u_i\in \sum_{i\in I}U_i $ ist nicht eindeutig.

\paragraph{Bemerkung}
	Sind $ \dim V <\infty $ und $ \# I < \infty $ so gilt
		\begin{equation*}
		\forall i\in I: \dim U_i < \infty
		\end{equation*}
		
	und es gilt die Dimensionsformel für direkte Summen (Beweis in Aufgabe 35):
		\begin{equation*}
		\dim \bigoplus_{i\in I}U_i = \sum_{i\in I} \dim U_i.
		\end{equation*}
	
	Ist insbesondere $ B=(b_1,...,b_n) $ eine Basis von $ V $, so gilt
		\begin{equation*}
		\dim V = \dim \bigoplus_{i=1}^n [b_i]=\sum_{i=1}^{n}1 = n.
		\end{equation*}
		
\paragraph{Bemerkung}
	Seien $ U,U'\subset V $ komplementäre UVR, also $ V = U \oplus U' $, dann werden durch
		\begin{equation*}
		v = u+u' \mapsto
			\begin{cases}
				p(v):=u\\
				p':= u'
			\end{cases}
		\end{equation*}
	
	Endomorphismen $ p,p'\in \operatorname{End}(V) $ (wohl-)definiert, da $ u,u' $ durch $ v $ eindeutig bestimmt sind (Linearität von $ p,p' $ ist klar).
	Offenbar ist 
		\[ p(V) = U \text{ und } \ker p = U'\]
	und es gilt
		\[ p^2 := p\circ p = p \]
	und analog für $ p' $; außerdem gilt ($ \circ $ ausgelassen)
		\[ p+p' = id_V \text{ und } p'p = 0 = pp'.\]
		
\subsection{Definition (Projektion)}
	\begin{Definition}[Projektion]
		$ p\in \End(V) $ heißt Projektion, falls $ p^2 = p $ (d.h. falls $ p $ idempotent ist).
	\end{Definition}
	
\subsection{Satz (Projektionen)}
	\begin{Satz}[Projektionen]
	Sei $ p\in \End(V) $ Projektion, dann ist $ p'= \id_V-p $ Projektion mit $ pp' = p'p = 0 $. Gilt andererseits $ p+p' = id_V $ und $ pp' = 0 $ für $ p,p' \in \End(V) $, so sind $ p,p' $ Projektionen mit
		\[ V = p(V)\oplus p'(V) = \ker p' \oplus \ker p. \]
	\end{Satz}

% % % VO_12.11.15
\paragraph{Beweis}
	Seien $p\in \operatorname{End}(V)$ Projektion und $p' := id_V -p$; dann gilt:
		\begin{gather*}
		p\circ p' = p(id_V-p)=p-p^{2} = 0; (p^{2} = p\circ p)\\
		p'\circ p = (\id_V-p)\circ p = p - p^{2} = 0
		\end{gather*}
	und
		\[p' \circ p' = p'^2 = p' \circ(\id_V-p)=p'-p'\circ p = p',\]
	d.h., $p'\in\End(V)$ ist Projektion.
	
	Anderseits: Seien $p+p' = \id_V \text{ und } pp' = 0$.
		
	Dann gilt:
		\[p-p^2 = p(id_V-p) = pp' = 0\]
	d.h., $p\in\operatorname{End}(V)$ ist Projektion, damit ist auch $p'$ Projektion (erster Teil) und $p' p = 0 $. Weiters liefert
		\[\forall v\in V: v=id_v(v) = p(v) + p'(v) \Rightarrow V = p(V)+p'(V),\]

	und ist $w = p(v)= p'(v')$ für geeignete $v,v'\in V$ (d.h., $w \in p(V)\cap p'(V)$), so gilt
		\[ w = p(v) = p^2(v) = p(p(v)) = p(w) = p(p'(v')) = pp'(v') = 0, \]
	also $p(V)\cap p'(V) = {0}$ und damit $V = p(V)\oplus p'(V)$. Weiters gilt
		\[0 = p \circ p' \rightarrow p'(V)\subset \ker p\]
	und ist $v\in \ker p$, so folgt
		\[v = p(v) + p'(v) = 0 + p'(v)\in p'(V) \Rightarrow \ker p \subset p'(V).\]
	Für $p'$ gilt das Gleiche und wir haben $\ker p = p'(V)$ und $\ker p' = p(V)$.
	Damit folgt die letzte Behauptung 
		\[V = \ker p \oplus\ker p'.\]
	
\paragraph{Bemerkung}
	Im Beweis haben wir etwas mehr bewiesen als behauptet - nämlich:
		\[ \ker p = p'(V)\text{ und }\ker p' = p(V) \]
		
\subsection{Beispiel und Definition (Involution)}	
	\begin{Definition}[Definition]
		Sei $s\in \operatorname{End}(V)$ eine Involution d.h., 
		\[ s^2 = \id_V \text{ und } p_\pm := \frac{1}{2}(\id_v\pm s). \]
	Offenbar gilt dann
		\[ p_{+} + p_{-} = \id_V \text{ und } p_+ p_{-} = \frac{1}{4}(\id_V +s)(\id_V -s) =  \frac{1}{4}(\id_v^2-s^2)=0 \]
	also (Satz) sind $p_\pm\in\End(V)$ Projektionen mit komplementären Bildern bzw. Kernen.
	\end{Definition}
		
\subsection{Lemma und Definition (Produkt von VR)}
	\begin{Definition}[Produkt von VR]
		Ist $(V_i)_{i\in I}$ eine Familie von $ K $-VR $V_i$, so wird das (mengenthoretische) Produkt:
		\[V:= \prod_{i\in I}V_i=\{(v_i)_{i\in I}\mid\forall i\in I:v_i\in V_i\}\]
	mit den komponentenweise definierten VR-Operationen zu einem $ K $-VR. Dies ist der Produktraum der Familie	$(V_i)_{i\in I}$.
	\end{Definition}
		
\paragraph{Beweis} Aufgabe!

\paragraph{Bemerkung} 
	Ist $V = \prod_{i\in I} V_i$ ein Produktraum, so erhält man kanonische UVR
		\[U_i:=\{v=(v_i)_{i\in I}\in V\mid\forall j \neq i:v_j = 0\}\subset V,\]
	die isomorph zu den $V_i$ sind vermittels der Faktorprojektionen
		\[\pi_i:V\to V_i:(v_j)_{j\in I} \mapsto v_i,\]
	bzw.  Faktor-Injektionen
		\[\iota_i:V_i\to V: v_i \mapsto(v_j)_{j\in I}\text{, wobei }v_j :=
			\begin{cases}
				v_i &\text{ falls } j=i\\
				0 &\text{ sonst.}
			\end{cases}
		\]
	Ist dann $\# I < \infty$, so erhält man
		\[\prod_{i\in I} V_i\cong \bigoplus_{i\in I}U_i (=: \bigoplus_{i\in I}V_i);\]
	ist $\#I=\infty$, so ist diese Identifikation im Allgemeinen falsch!
	
\paragraph{Beispiel}
	Für einen Körper $ K $ liefert das $ n $-fache Produkt den Standardraum
		\[\prod_{i=1}^{n}K = \{(x_i)_{i = 1,...,n}\mid\forall i \in \{1,...,n\}: x_i \in K\} = K^n \cong \bigoplus_{i=1}^n\{(x_i)_{i\in {1,...,n}}\mid\forall j\neq i: x_j = 0\};\]
	für den Raum der K-wertigen Folgen ist jedoch
		\[\prod_{i\in \mathbb{N}}K=K^{\mathbb{N}}\neq\bigoplus_{i\in \mathbb{N}}\{(x_i)_{i\in \mathbb{N}}\mid\forall j\neq i: x_j=0\}.\]
			
\subsection{Lemma und Definition (Nebenklassen)}
	\begin{Definition}[Nebenklassen]
		Sei $U\subset V$ UVR. Die Menge der Nebenklassen 
		\[V/U := \{v+U\mid v\in V\},\]
	wobei
		\[v+U:=\{v+u\mid u\in U\}\]
	die Nebenklasse zu $v\in V$ bezeichnet, wird mit den durch
		\[(v+U)+(w+U):=(v+w)+U \text{ und }(v+U)x := vx + U\]
	bzw.
		\begin{align*}
			+&: V/U \times V/U \to ((v+U),(w+U))\mapsto (v+w)+U,\\
			\cdot &: K\times V/U \to V/U, (x,(v+U))\mapsto (vx)+U,
		\end{align*}
	definierten Operationen ein Vektorraum: der Quotientenraum $V/U$.
	\end{Definition}
			
\paragraph{Beweis}
	Zu zeigen: Wohldefiniertheit der Operationen und VR-Axiome (werden übergangen).
	
	Wohldefiniertheit der Skalarmultiplikation:
	
	Ist $x\in K$ und sind $(v+U),(v'+U)\in V/U$ gleich, also $v+U = v'+U$, so gilt
		\begin{gather*}
		v+U = v'+U \Leftrightarrow v - v'\in U\\
		\Rightarrow (v-v')x \in U \Rightarrow vx+U=v' x+U
		\end{gather*}
	Das Resultat der Skalarmultiplikation bringt also nicht von dem Repräsentanten $v$ einer Nebenklasse $v+U$ ab, sondern nur von der Nebenklasse.
	
	Die Wohldefiniertheit der Addition ist analog zu beweisen.

%VO_17.11.15
	
\subsection{Bemerkung \& Definition (Äquivalenzrelation)}
	\begin{Definition}[Äquivalenzrelation]
		Der Definition von $ V/U $ liegt ein allgemeineres Prinzip zugrunde:	
		\[ v\sim w :\Leftrightarrow (v+U)= (w+U) \Leftrightarrow w-v \in U \]
	definiert für jeden UVR $ U\subset V $ eines $ K $-VR $ V $ eine Äquivalenzrelation auf $ V $, d.h.:
		\begin{enumerate}[(i)]
			\item $ \forall v\in V: v\sim v $ (Reflexivität;)
			\item $ \forall v,w\in V: v\sim w\Leftrightarrow w\sim v $ (Symmetrie);
			\item $ \forall u,v,w\in V: u\sim v\land v\sim w\Rightarrow u\sim w $ (Transitivität).
		\end{enumerate}
	\end{Definition}

\subparagraph{Beweis ($ v\sim w:\Leftrightarrow w-v\in U $ definiert Äquivalenzrelation)}
	
	\begin{itemize}
		\item Reflexivität: Sei $ v\in V $ beliebig, dann gilt $ v-v=0\in U $, da $ U $ UVR.
		\item Symmetrie: Seien $ v,w\in V $ beliebig, dann gilt
			\begin{align*}
			v\sim w 	&\Leftrightarrow w-v\in U\\
						&\Leftrightarrow v-w\in U \Leftrightarrow w\sim v.
			\end{align*}
		\item Transivitität: Seien $ u,v,w\in V $ beliebig; gilt nun
			\begin{gather*}
			u\sim v \text{, d.h. } v-u\in U \text{, und: } v\sim w\text{, d.h. } w-v\in U,\\
			\text{so gilt auch: }(w-v)+(v-u)= w-u\in U \Leftrightarrow u\sim w.
			\end{gather*}
	\end{itemize}
	
	Ist dann $ \sim $ eine Äquivalenzrelation auf einer Menge $ X $, so zerfällt $ X $ in Äquivalenzklassen
		\[ X_a = \{y\in X\mid y\sim a\} \]
	d.h., $ X $ ist disjunkte Vereinigung der Äquivalenzklassen:
		\[ X = \dot{\bigcup}_{a\in X}X_a \text{ und } X_a \cap X_b \neq \emptyset \Rightarrow X_a = X_b. \]
		
	Insbesondere zerfällt also ein VR $ V $ in Äquivalenzklassen ("`Nebenklassen"')
		\[ v+U (= V_v)\subset V, \]
	wenn $ U $ ein UVR von $ V $ ist -- nach Lemma wird die Menge der Neben- bzw. Äquivalenzklassen wird dann wieder ein VR. Ähnlich wie den Quotientenvektorraum definiert man (z.B.) die Faktorgruppe [Havlicek 1.11.11].
	
\paragraph{Bemerkung}
	Allgemeiner definiert man eine (binäre) Relation $ (X,Y,\Gamma) $ zwischen Mengen $ X, Y $ durch den Graph den Relation, eine Teilmenge
		\[ \Gamma \subset X\times Y = \{(x,y)\mid x\in X \land y\in Y\}. \]
	Zum Beispiel ist eine Abbildung $ f:X\to Y $ eine Relation $ (X,Y,\Gamma_f) $, sodass
		\[ \forall x\in X\exists ! y\in Y:(x,y)\in \Gamma_f. \]
	Ein anderes Beispiel ist die Ordnungsrelation auf $ \mathbb{Z} $, definiert durch
		\[ x\leq y :\Leftrightarrow \{(x,y)\in \mathbb{Z}^2\mid y-x\in \mathbb{N}\}, \]
	eine Ordnungsrelation ist reflexiv, transitiv und antisymmetrisch (also $ \forall x,y\in \mathbb{Z}: x\leq y\land y\leq x\Rightarrow x=y $).
	Die Teilmengenrelation
		\[ Y\subset\tilde{Y} \text{ für } Y,\tilde{Y}\in \mathcal{P}(X):= \{Y\mid Y\subset X\} \]
	liefert auch eine Ordnungsrelation auf der Potenzmenge $ \mathcal{P}(X) $ von $ X $, jedoch nur eine Halbordnung -- im Gegensatz zur Totalordnung auf $ \mathbb{Z} $, wo je zwei Elemente vergleichbar sind, d.h.,
		\[ \forall x,y\in \mathbb{Z}: x\leq y\lor y\leq x. \]
	Auf $ \mathcal{P}(X) $ gilt dies im Allgemeinen nicht, denn z.B. in $ \mathcal{P}(\{0,1\}) $ sind $ \{0\},\{1\} $ nicht vergleichbar, denn
		\[ \{0\}\not\subset\{1\}\land \{1\}\not\subset \{0\}. \]
		
%VO 2015-11-17 Teil 2

\subsection{Lemma (Dimensionen von komplementären UVR)}
	\begin{Lemma}[Dimensionen von komplementären UVR]
		Ist $ U\subset V $ UVR und $ U' $ komplementärer UVR zu $ U $ in $ V $, so gilt
		\[ U'\cong V/U \]
	vermöge
		\[ U'\ni u' \overset{\phi}{\mapsto} u'+U\in V/U. \]
	Ist $ \dim V<\infty $, so gilt $ \dim U+\dim V/U = \dim V $.
	\end{Lemma}
	
\paragraph{Beweis}
	Zu zeigen: $ \phi $ ist Isomorphismus. Dass $ \phi $ Homomorphismus ist, folgt aus der Definition der VR-Operationen auf $ V/U $.

	Injektivität: Sei $ u'\in \ker \phi $, d.h.
		\begin{gather*}
		\phi(u')=u'+U=0+U\in V/U\\
		\Rightarrow u'\in U'\cap U = \{0\} \Rightarrow u'=0
		\end{gather*}
		
	Surjektivität: Sei $ v+U\in V/U $ mit $ v\in V = U \oplus U' $; mit der Projektion
	\begin{gather*}
		p':V\to V, v=u+u' \mapsto p'(v) := u'\\
		\text{ist } v + U = u' + u + U = u' + U = \phi(p'(v)),\\
		\text{also }V/U = \phi(U').
	\end{gather*}
	
	Die Dimensionsformel folgt dann aus der für direkte Summen:
		\[ \dim U' = \dim V/U \Rightarrow \dim V/U = \dim V-\dim U \]
		
\paragraph{Beispiel}
	Ist $ p\in \End(V) $ eine Projektion, so auch $ p':= \id_V-p $ und es gilt
		\[ V= \ker p \oplus \ker p' = \ker p \oplus p(V), \]
	also gilt
		\[ p(V)\cong V/\ker p. \]

\subsection{Homomorphiesatz für lineare Abbildungen: }
	\begin{Satz}[Homomorphiesatz für lineare Abbildungen]
		Für $ f\in \hom(V,W) $ ist $ f(V)\cong V/\ker f $ vermöge
		\[ V/\ker f\ni v+\ker f \overset{\phi}{\mapsto} f(v)\in f(V) \]
	\end{Satz}

\paragraph{Beweis}
	Wohldefiniertheit von $ \phi $: 
	
	Sind $ v+\ker f, v'+\ker f \in V/\ker f$ und $ v+\ker f = v'+\ker f $, so gilt 
		\begin{gather*}
		v'-v \in \ker f \Rightarrow f(v')-f(v) = f(v'-v) = 0 \\
			\Rightarrow f(v') = f(v)
		\end{gather*}
	
	d.h. $ \phi $ ist wohldefiniert.
	
	Linearität:
		\begin{itemize}
		\item Für $ x\in K $ und $ v+\ker f \in V/\ker f $ ist $ \phi( (v+\ker f)x) =f(vx) = f(v)x = \phi (v+\ker f)\cdot x $
		\item Für $ v+\ker f, w+\ker f\in V/\ker f $ ist 
			\begin{gather*}
			\phi ((v+\ker f)+(w+\ker f) ) = \phi((v+w)+\ker f) =\\
			f(v+w) = f(v)+f(w)=\phi(v+\ker f)+\phi(w+\ker f)
			\end{gather*}
		\end{itemize}
	
	Injektivität:
	Sei $ v+\ker f\in \ker \phi $, also 
		\[ \phi(v+\ker f)= f(v) = 0; \]
	dann folgt
		\[ v\in \ker f \Rightarrow v+\ker f = \ker f = 0\in V/\ker f. \]
	
	Surjektivität:
	folgt direkt aus der Definition.		