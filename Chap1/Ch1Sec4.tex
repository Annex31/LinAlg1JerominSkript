%VO_29.10.15
\section{Homomorphismen}
\paragraph{Definition:}
	Sind $ V $ und $ W $ $ K $-VR, so heißt eine Abbildung $f: V \rightarrow W$ ($ K $-)Linear oder ein (Vektorraum-)Homomorphismus $f\in \hom(V,W)$, falls gilt:

\begin{enumerate}[(i)]
	\item $\forall v,w \in V: f(v+w) = f(v) + f(w)$;
	\item $\forall v\in V\ \forall x\in K: f(vx) = f(v)x$
\end{enumerate}

    das heißt, f ist verträglich mit den Vektorraumoperationen in V und W.
    
\paragraph{Bemerkung:}
	Damit die Verträglichkeit mit der Skalarmultiplikation sinnvoll ist, müssen $ V $ und $ W $ Vektorräume über demselben Körper $ K $ sein.

\paragraph{Bemerkung:}
	Für $f\in \hom(V,W)$ gilt stets $f(0_V) = f(0_V0_K) = f(0_V)0_K = 0_W$.
  
  Ebenso erklärt man zum Beispiel Gruppenhomomorphismen und Körperhomomorphismen. Sind etwa $(G,\circ)$ und $(H,*)$ Gruppen, so ist eine Abbildung $f: G \to H$ ein Gruppenhomomorphismus, falls $\forall g,h \in G: f(g\circ h) = f(g) * f(h)$
  
\paragraph{Beispiel:}
	Ist $f\in \hom(V,W)$ ein Vektorraumhomomorphismus so ist $ f $ nach (i) Gruppenhomomorphismus von $ (V,+) $ in $ (W,+) $.
  
\paragraph{Beispiel:}
	Sei $ V $ ein $ K $-VR und $y\in K$ fest, dann ist die Streckung um $y: \eta_y:V\to V: v\mapsto \eta_y(v) := vy$ ein Homomorphismus von $ V $ in sich, $\eta_y\in \hom(V,V)$. Eine Streckung nennt man auch Homothetie.
  	
\paragraph{Beispiel:}
	Sei $V = \mathbb{C} = \{z = x+iy\mid x,y\in \mathbb{R}\}$, dann ist die komplexe Konjugation $\mathbb{C}\ni z = x+iy \mapsto x-iy =: \bar{z} \in \mathbb{C}$ kein Homomorphismus von $\mathbb{C}$ in sich, wenn man $\mathbb{C}$ als $\mathbb{C}$-VR auffasst. Hingegen ist sie ein Homomorphismus von $\mathbb{C}$ in sich, wenn man $\mathbb{C}$ als $ \mathbb{R} $-VR auffasst.
\paragraph{Lemma:}
	$f:V\to W$ ist genau dann ein Homomorphismus, wenn für jede beliebige Linearkombination gilt: $f(\sum_{i\in I}v_ix_i) = \sum_{i\in I}f(v_i)x_i$

\paragraph{Beweis:}
	Eine Richtung ist trivial, die andere mit vollständiger Induktion zu zeigen.

\paragraph{Fortsetzungssatz:} 
	Seien $ V $ und $ W $ $K$-VR, $(b_i)_{i\in I}$ eine Basis von $ V $ und $(c_i)_{i\in I}$ eine Familie in $ W $. Dann gilt: $\exists!f\in \hom(V,W), \forall i\in I: f(b_i) = c_i$.
    
\paragraph{Bemerkung:}
        Anders ausgedrückt: ist $B\subset V$ eine Basis von $ V $, so kann jede Abbildung $f: B\to C\subset W$ Eindeutig zu einem Homomorphismus $f: V\to W$ fortgesetzt werden.
    
\paragraph{Beweis:}
	Wir beweisen die Existenz und die Eindeutigkeit getrennt. 
	\begin{enumerate}
		\item Eindeutigkeit: Sei $f\in \hom(V,W)$ so, dass $\forall i\in I: f(b_i)=c_i$. Sei $v\in V$ beliebig. Da $ B $ Erzeugendensystem ist, lässt sich $ v $ als Linearkombination in $(b_i)_{i\in I}$ mit geeigneten Koeffizienten $(x_i)_{i\in I}$ in $ K $ darstellen.
			\begin{gather*}
    				v=\sum_{i\in I}b_ix_i \Rightarrow f(v) = \sum_{i\in I} f(b_i)x_i = \sum_{i\in I}c_ix_i
    			\end{gather*}
    
                        Damit ist $ f(v) $ eindeutig durch $ v $ und die $c_i = f(b_i)x_i$ bestimmt.
    
    		\item Existenz: Da $(b_i)_{i\in I}$ auch linear unabhängig ist, ist jedes $v\in V$ eindeutig als Linearkombination in $(b_i)_{i\in I}$ dargestellt, damit ist durch $f:V\to W: v=\sum_{i\in I}b_ix_i \mapsto f(v):=\sum_{i\in I}c_iv_i$ eine Abbildung wohldefiniert.
    
                        Weiters ist $f\in\hom(V,W)$ wegen
                        \begin{gather*}
                                f(v+w) =\sum_{i\in I}c_i(x_i+y_i)=\sum_{i\in I}c_ix_i+ \sum_{i\in I}c_iy_i\Rightarrow f(x) + f(y) \text{ für alle }\left\{
                                        \begin{array}{l}
                                                v=\sum_{i\in I}b_ix_i \in V\\
                                                w=\sum_{i\in I}b_iy_i \in V
                                        \end{array}
                                \right.
                        \end{gather*}
    
                        und
                        \begin{gather*}
                            f(vx) =\sum_{i\in I}c_i(x_ix)\Rightarrow\sum_{i\in I}(c_ix_i)x = (\sum_{i\in I}c_ix_i)x= f(v)x \text{ für }  x\in K\text{ und }v= \sum_{i\in I}b_ix_i \in V_i
                        \end{gather*}
                        Damit ist die Linearität von $ f $ gezeigt.
        \end{enumerate}
    
\paragraph{Beispiel und Definition:}
	Der Dualraum $V^\ast := \hom(V,K)$ eines $K$-VRs $V$ ist ein $ K $-VR $(\subset K^V)$. Ist $\dim V=:n<\infty$ so ist $\dim V^\ast=n$.
	Ist $B=(b_i, ... ,b_n)$ eine Basis von $ V (\dim V < \infty)$, so definieren wir für $ i = \{1, ... ,n\} $ die Linearform (nach Fortsetzungssatz):
	\begin{equation*}
		b_i^\ast\in V^*:V\to K, \forall j\in \{1,...,n\}:b_i^*(b_j)=\delta_{ij}
	\end{equation*} die zu $ B $ duale Basis $ B^* $ von $V^\ast$.

%ADB_VO_03.11.15
        Nämlich:
        \begin{itemize}
        \item $ V^* $ ist $ K $-VR. Wir zeigen $ V^*\subset K^V $ ist UVR.
        \begin{enumerate}
                \item $ 0: V\to K $ ist linear, d.h. $ 0 \in V^* \Rightarrow V^* \neq \emptyset $
                \item Seien $ f,g \in V^* $ und $ x\in K $; dann gilt
			\begin{align*}
				\forall v,w\in V: (fx+g)(v+w) &= f(v+w)x+g(v+w)\\
                                                              &= (f(v)+ f(w))x+(g(v)+g(w))\\
                                                              &= (f(v)x+g(v))+(f(w)x+g(w))\\
                                                              &= (fx+g)(v)+(fx+g)(w)
			\intertext{genauso:}
                                \forall v\in V, y\in K: (fx+g)(vy) &= f(vy)x+g(vy)\\
                                                                   &= f(v)yx + g(v)y\\ 
                                                                   &= (f(v)x +g)y = ((fx+g)y)(v)
                        \end{align*}
                        Damit gilt: $ fx+g\in \hom (V,K) = V^* $
        \end{enumerate}
	
	Da $ f,g\in V^* $ und $ x\in K $ beliebig waren, zeigt das UR-Kriterium, dass $ V^*\subset K^V $ ein UVR ist und damit selbst $ K $-VR ist.
	\item $ B^* $ ist linear unabhängig: Seien $ x_1,...,x_n $ so, dass
		\begin{equation*}
		0 = \sum_{i=1}^{n}b_i^*x_i \Rightarrow \forall j=\{1,...,n\}:0=(\sum_{i=1}^{n}b_i^*x_i)(b_j) = \sum_{i=1}^{n}b_i^*(b_j)x_i = \sum_{i=1}^{n}\delta_{ij}x_i = x_j
		\end{equation*}
	Also $ x_1 = ... = x_n = 0 $ und damit ist $ B^* $ linear unabhängig.
	\item $ B^* $ ist Erzeugendensystem: Sei $ f\in V^* $ beliebig, dann gilt:
	\begin{equation*}
		\forall j = \{1,...,n\}:f(b_i) = \sum_{i=1}^{n}b_i^*(b_j)f(b_i) = (\sum_{i=1}^{n}b_i^*f(b_i))b_j \Rightarrow f = \sum_{i=1}^{n}b_i^*f(b_i)\in [B^*]
	\end{equation*}
	
	Da $ f\in V^* $ beliebig war, ist also $ V^* = [B^*]$.
	\end{itemize}
	
	Damit ist $ B^* = \{b_1^*,...,b_n^*\}$ eine Basis von $ V^* $ -- insbesondere also $ \dim V^* = n = \dim V = \dim K\cdot \dim V $.
%ADE_VO_03.11.15

\paragraph{Bemerkung:}
	Ist $\dim V = \infty$ und $B=(b_i)_{i\in I}$ eine Basis von $V$, so liefert $B\ast=(b_i^\ast)_{i\in I}$ mit $\forall j\in I:b_i^\ast(b_j)=\delta_{ij}$ eine lineare unabhängige Familie. Diese ist jedoch kein Erzeugendensystem von $V^\ast: f\in\hom(V,K)=V^\ast$ mit $\forall j\in I:f(b_j)=1$ lässt sich nicht in $B^\ast$ linear kombinieren. Wäre $f=\sum_{i\in I}b_i^\ast x_i$, so gälte $\forall j\in I: x_j =\sum_{i\in I}b_i^\ast(b_j)x_j= \sum_{i\in I} \delta_{ij}x_j = f(b_j) = 1$; 

	Das heißt, $(x_i)_{i\in I}$ wäre eine Familie in $ K $ mit $\#\{i\in I\mid x_i\neq 0\}=\infty$.

%VO_03.11.15
\paragraph{Satz:}
	$ \hom (V,W) $ ist ein VR. $\dim\hom (V,W) = m\cdot n$, falls $m:=\dim W<\infty, n:=\dim V< \infty$.
	
\paragraph{Beweis:}
	Addition und Skalarmultiplikation in $\hom (V,W)$ werde (wie für $K$-wertige Abbildungen oder in $V^*$) punktweise definiert:
	\begin{itemize}
		\item für $f,g \in \hom (V,W)$ setzt man $(f+g)(v) := f(v) + g(v)$ für alle $v\in V$,
		\item für $f\in \hom (V,W)$ und $x\in K$ setzt man $(fx)(v) := f(v)x$ für alle $v\in V$.
	\end{itemize}
	Die so definierten Abbildungen $f+g,fx: V\to W$ sind linear, $f+g, fx\in \hom (V,W)$, aufgrund der VR-Eigenschaften von $V$.
	
	Damit zeigt man: $\hom (V,W)$ ist $K$-VR (siehe Aufgabe 27).
	
	Seien nun $\dim V = n < \infty$ und $\dim W = m < \infty$.
	
	Wir wählen (nach BES) Basen $B = (b_1,...,b_n)$ von $V$ und $C=(c_1,...,c_m)$ von $W$ und definieren
		\begin{equation*}
			f_{ij}\in \hom (V,W) mit f_{ij}:= c_i\cdot b_j^* \text{ für } 
				\begin{cases}
					i\in I := \{1,...,m\}\\
					j\in J := \{1,...,n\}
				\end{cases}
		\end{equation*}
	Behauptung: $F=(f_{ij})_{I,J}$ ist Basis von $\hom (V,W)$.
	
	Da $(c_i)_{i\in I}$ linear unabhängig in $W$ ist, gilt für jede Famlilie $(x_{ij})_{I,J}$ in $K$:
		\begin{gather*}
			0 = \sum_{I,J} f_{ij}x_{ij} \Rightarrow \forall k \in J: 0 = \sum (f_{ij}x_{ij})(b_k)\\
			= \sum c_i b_j^* (b_k) x_{ij} = \sum_{i\in I} c_ix_{ik} \Rightarrow \forall k\in J\forall i\in I:x_{ik} = 0
		\end{gather*}
	Also ist $F$ linear unabhängig.
	
	Da $(c_i)_{i\in I}$ Erzeugendensystem von $W$ ist, existiert zu jedem (fest gegebenen) $f\in\hom (V,W)$ eine Familie $(x_{ij})_{IJ}$ in $K$, sodass
		\begin{gather*}
		\forall k\in J: f(b_k) = \sum_{i\in I} c_i x_{ik} \text{ (da $(c_i)_{i\in I}$ Erzeugendensystem)}\\
		= \sum_{I,J}c_ib_j^*(b_k)x_{ij} = \left(\sum_{IJ} f_{ij}x_{ij}\right)(b_k)\\
		\text{also (Fortsetzungssatz): } f=\sum_{I,J}f_{ij}x_{ij} \in [F]
		\end{gather*}
	
	Da $f\in\hom (V,W)$ beliebig war, gilt also $\hom (V,W) = [F]$. Damit ist $F$ Basis von $\hom (V,W)$ und $\dim\hom (V,W) = \# F = m\cdot n$.
	
\paragraph{Lemma und Definition}
	Sei $f\in \hom (V,W)$. Dann sind Bild und Kern von f:
		\begin{equation*}
			f(V) = \{f(v)\in W\mid v\in V \}\subset W \text{ bzw. } \ker (f) := \{v\in V\mid f(v) = 0 \} \subset V
		\end{equation*}
	
	UVR von $W$ bzw. $V$. Ihre Dimensionen heißen Rang und Defekt von $f$:
		\begin{equation*}
			\rg f := \dim f(V) \text{ bzw. } \dfkt f := \dim \ker f
		\end{equation*}

\paragraph{Bemerkung: }
	Da $f(0)=0$ für  $f\in \hom (V,W)$ gilt $\{o_V \}\in \ker f$ und $\{o_W \}\in f(V)$.

\paragraph{Beweis: }
	Zu zeigen: Das Bild $f(V)\subset W$ und $\ker f\subset V$ sind UVR. Nach Bemerkung gilt $f(V)\neq \emptyset$ und $\ker f \neq \emptyset$ -- wir verwenden dann das UR-Kriterium.
	
	Das Bild $f(V)$ ist UVR: $f(V) \neq \emptyset$. Es bleibt zu zeigen:
		\begin{equation*}
			\forall w_1,w_2\in f(V), \forall x\in K: w_1x+w_2 \in f(V).
		\end{equation*}
	
	Seien also $w_1 = f(v_1), w_2 = f(v_2) \in f(V)$ und $x\in K$; dann gilt:
		\begin{equation*}
			w_1x+w_2 = f(v_1)x+f(v_2) = f(v_1x+v_2)\in f(V)
		\end{equation*}
		
	Der Kern $\ker f$ ist UVR: $\ker f\neq \emptyset$; seien $v_1,v_2\in \ker f$ und $x\in K$, dann gilt:
		\begin{equation*}
			f(v_1x+v_2) = f(v_1)x+f(v_2) = 0\cdot x + 0 = 0 \Rightarrow v_1x+v_2\in \ker f
		\end{equation*}

\paragraph{Bemerkung: }
	Allgemeiner kann man für $f\in \hom (V,W)$ zeigen:
		\begin{enumerate}
			\item ist $U\subset V$ UVR, so ist $f(U)\subset W$ UVR
			\item ist $U\subset V$ UVR, so ist $f^{-1}(U) = \{v\in V\mid f(v) \in U \}\subset V$ ein UVR
		\end{enumerate}

\paragraph{Bemerkung: }
	Die Funktion $f\in \hom (V,W)$ ist genau dann injektiv, wenn $\ker f = \{o\}$. Nämlich:
		\begin{itemize}
			\item ist $f$ injektiv und $v\in \ker f$, so gilt $f(v) = 0 = f(0) \Rightarrow v=0$
			\item ist $\ker f = \{ o \}$ und sind $v,w \in V$ mit $f(v) = f(w)$, so folgt\\
				$0=f(v)-f(w) = f(v-w) \Rightarrow v-w\in \ker f = \{o\} \Rightarrow v = w$
		\end{itemize}

\paragraph{Bemerkung: }
	Eine lineare Abbildung $ f\in \hom (V,W) $ ist genau dann
		\begin{enumerate}[(i)]
			\item injektiv, wenn $ \forall S\subset V: S$ lin. unabh. $ \Rightarrow f(S) $ lin. unabh.
			\item surjektiv, wenn $ \forall E \subset V:E $ Erz. Syst. $ \Rightarrow f(E)$ Erz. Syst.
			\item bijektiv, wenn $ \forall B\subset V: B$ Basis $ \Rightarrow f(B)$ Basis
		\end{enumerate}

	Ist $ f\in \hom (V,W) $ bijektiv, so ist $ f^{-1}\in \hom (W,V) $.

\paragraph{Rangsatz: }
	Sei $ f\in \hom (V,W) $. Ist $ \dim V = n < \infty $,  so gilt $\rg f + \dfkt f = \dim V$.  Ist $ \dim V = \infty $, so gilt $ \rg f = \infty $ oder $ \dfkt f = \infty $.

%VO_05.11.15