% % % % Chapter 1 Section 3 % % % %
\section{Basis und Dimensionen}

\paragraph{Definition}
	Eine Teilmenge $S\subset V$ oder eine Familie $(v_i| i\in I)$ in $ V $ heißt:
	\begin{itemize}
		\item Erzeugendensystem von $ V $, falls $[S] = V$ bzw. $[(v_i)_{i\in I}] = V$
		\item linear unabhängig, falls $\forall v\in S: v \notin [S\setminus\{{v\}}]$ bzw. $\forall i\in I: v_i \notin [(v_j)_{j\in I\setminus\{{i\}}}]$
	\end{itemize}

	und sonst linear abhängig. Eine Basis ist ein linear unabhängiges Erzeugendensystem.

\paragraph{Bemerkung}
	Man kann jede (Teil-) Menge $S\subset V$ als Familie in V auffassen mit
	\begin{equation*}
		v: S \to V: v\mapsto id_S(v) = v.
	\end{equation*}
	
	Andererseits gilt für eine Familie $(v_i)_{i\in I} $:
	\begin{equation*}
		(v_i)_{i\in I} \text{ linear unabhängig } \Rightarrow \{v_i| i\in I\} \text{ linear unabhängig.}
	\end{equation*}
	
	Die Umkehrung gilt im Allgemeinen nicht.

	Eine Familie (in $ V $) enthält mehr Information als eine Teilmenge von $ V $.
	
\paragraph{Beispiel und Definition}
	Für $V = K^n$ ist $(e_i, ... , e_n)$,
	\begin{equation*}
		e_i:\{{1, ... ,n\}} =: I\to K: j\mapsto e_i(j)= \delta_{ij}=
		\begin{cases}
			1,& \text{falls } i=j\\
			0,& \text{sonst}
		\end{cases}
	\end{equation*}

	eine Basis -- die Standardbasis des (Standard-)Vektorraumes $K^n$.

\paragraph{Beweis}
	Z.z.: $ (e_i)_{i\in I} $ ist ein linear unabhängiges Erzeugendensystem. Bekannt ist: $ [(e_i)_{i\in I}] = K^n $. Andererseits gilt für jedes $i\in I$ und jede Familie $(x_j| j\in I)$ in K
	\begin{align*}
		\left(\sum_{j= I\setminus\{i\}}e_jx_j\right)(i) = \sum_{j=I\setminus\{i\}}e_j(i)x_j = 0\\
		\neq 1 = e_i(i) \Rightarrow \sum_{j\in I\setminus\{i\}} e_jx_j \neq e_i,
	\end{align*}
	
	also gilt:
	\begin{equation*}
		\forall i\in I: e_i \notin [(e_j)_{j\in I\setminus\{i\}}] = \left\{\sum_{j=I\setminus\{i\}} e_jx_j\mid (x_j)_{ j\in I}\right\} \text{ mit } \#\{j\in I| x_j \neq 0\}<\infty
	\end{equation*}
	
\paragraph{Lemma}
	Eine Familie $(v_i)_{i\in I}$ ist linear unabhängig gdw. für jede Linearkombination
	\begin{equation*}
		0 = \sum_{i\in I} v_ix_i \Rightarrow \forall i\in I: x_i = 0.
	\end{equation*}

\paragraph{Beweis}
	Wir zeigen zwei Richtungen der Äquivalenz der Negationen: 
	\begin{equation*}
		(v_i)_{i\in I} \text{ linear abhängig } \Leftrightarrow \exists(x_i)_{i\in I} \neq (0)_{i\in I}: \sum_{i\in I} v_ix_i = 0.
	\end{equation*}

	$\Leftarrow$: Wir nehmen an, es gäbe eine nicht-triviale Linearkombination der Null,
	\begin{equation*}
		0 = \sum_{i\in I} v_ix_i, \text{ wobei } \exists j\in I: x_j \neq 0.
	\end{equation*}

	Für $(y_i)_{i\in I}, y_i := - \frac{xi}{y_i}$ ist dann
	\begin{equation*}
		0 = v_jx_j + \sum_{i\in I\setminus\{j\}} v_ix_i \Rightarrow v_j = -\left(\sum_{i\in I\setminus\{j\}}v_ix_i\right)x_j^{-1} = \sum_{i\in I\setminus\{j\}} v_iy_i \in [(v_i)_{i\in I\setminus\{j\}}],
	\end{equation*}

	insbesondere ist also $(v_i)_{i\in I}$ linear abhängig.

	$\Rightarrow$: siehe Aufgabe.
	
\paragraph{Korollar}
	Ist $(v_i)_{i\in I}$ Basis von $ V $, so ist jeder Vektor $v\in V$ eindeutig in den $v_i$ darstellbar:
	\begin{equation*}
		\forall v\in V \exists! (x_i)_{i\in I}: v = \sum_{i\in I} v_ix_i
	\end{equation*}

\paragraph{Beweis}
	Sei $v\in V$ beliebig, dann gilt:
	\begin{equation*}
		V = [(v_i)_{i\in I}] \Rightarrow \exists (x_i)_{i\in I}: v = \sum_{i\in I} v_ix_i
	\end{equation*}

	liefern $(x_i)_{i\in I}$ und $(y_i)_{i\in I}$
	\begin{equation*}
		v = \sum_{i\in I} v_ix_i = \sum_{i\in I}v_iy_i \Rightarrow 0 = \sum_{i\in I} v_i(x_i-y_i)
		\begin{array}{l}
			\Rightarrow \forall i\in I: x_i = y_i\\
			\Rightarrow (x_i)_{i\in I} = (y_i)_{i\in I}
		\end{array}
	\end{equation*}

	Damit ist die Basisdarstellung $v = \sum_{i\in I} v_ix_i$ von $ v $ auch eindeutig.


\paragraph{Basislemma}
    Sei $S\subset V$ lin. unabh. und $E\subset V$ ein Erzeugendensystem mit $S\subset E$. Dann existiert eine Basis $B$ von $V$ mit $S\subset B\subset E$.

\paragraph{Beweis}
    Wir gehen für den Beweis davon aus, dass $\#E<\infty$. Betrachte alle Teilmengen $X\subset V$ mit $S\subset X\subset E$ und $X$ lin. unabh. Sei $B$ eine solche Menge, die maximal ist, d.h.
    \begin{equation*}
        \forall X\subset E: ((B\subset X\land X\text{ lin. unabh.}) \Rightarrow X= B)
    \end{equation*}
    
    Nach Konstruktion ist $B=\{b_1,...,b_n\}$ lin. unabh. Zu zeigen: $V=[B]$.\\
    Ist $B=E$, so folgt $[B]=[E]=V$.\\
    Ist $B\neq E$, so ist $B\cup \{v\} $ für (jedes) $v\in E\setminus B$ lin. abh., da $B$ maximal (Existenz einer maximalen Menge ist problematisch!) und lin. unabh. ist; also existiert eine nicht-triviale Linearkombination des Nullvektors.
    \begin{equation*}
    \exists x,x_1,...,x_n \in K: o=vx+\sum^n_{i=1}b_ix_i
    \end{equation*}

    Wäre $x=0$, so würde folgen $x_1=...=x_n=0$, da $B$ lin. unabh. ist. 
    Also ist $x\neq 0$ und 
    \begin{equation*}
    	v=-\sum^n_{i=1} b_i\frac{x_i}{x} \in [B].
    \end{equation*}
    
    Da dies für beliebiges $v\in E\setminus B$ gilt, folgt
    \begin{equation*}
    	E\subset [B] \Rightarrow V=[E]\subset [[B]] = [B],
    \end{equation*}
    
    d.h., $ B $ ist Erzeugendensystem und damit eine Basis mit $S\subset B\subset E$.

\paragraph{Bemerkung}
    Ist $\#E = \infty$, so kann man einen analogen Beweis führen, falls man an die Existenz einer maximalen Menge glaubt: Dies garantiert das Zornsche Lemma bzw. Auswahlaxiom.
    Wir werden das Lemma auch im Falle $\#E = \infty$ benutzen!

\paragraph{Beispiel}
    Für $V=K^3=K^I$ mit $I=\{1,2,3\}$ betrachte die Standardbasisvektoren 
    \begin{align*}
        e_i &:I\to K, j\mapsto e_i(j) = \delta_{ij}\text{, und}\\
        f_i &: I\to K, j\mapsto f_i(j):= 1-\delta_{ij};
    \end{align*}

    dann sind $S:= \{e_1,f_1\}$ und $E:= \{e_i,f_i\mid i\in I\}$ lin. unabh. bzw. Erzeugendensystem von $K^3$. Ergänzung von $S$ durch einen Vektor $e_i$ oder $f_i, i = 2,3$ liefert eine Basis $B$ mit $S\subset B\subset E$.
    
    Zum Beispiel: $B=\{e_1,f_1,f_2\}$ eine Basis, da sich jede Funktion $v\in K^3$ aus den Funktionen $e_1,f_1$ und $f_2$ linear kombinieren lässt.
    \begin{gather*}
        v=e_1x_1+f_1y_1 + f_2y_2\Leftrightarrow \left\{
            \begin{array}{l}
                v(2)=y_1\\
                v(3) - v(2) = y_1 + y_2 - y_1 = y_2\\
                v(1) + v(2) - v(3) = x_1 + y_2 - y_2 = x_1
            \end{array}
    	\right.
    \end{gather*}

    Dass $B$ lin. unabh. folgt dann; Wäre $B$ lin. abh., so würde folgen $f_2\in [\{e_1,f_1\}]\Rightarrow [B] \subset [\{e_1,f_1\}] \neq K^3$, was nicht der Fall ist.

\paragraph{Basisergänzungssatz}
    Jede lin. unabh. Menge $S\subset V$ kann zu einer Basis $B$ von $V$ ergänzt werden: Es existiert eine Basis $B$ von $V$ mit $S\subset B$.

\paragraph{Beweis}
    Sei $E\subset V$ ein Erzeugendensystem von $V$ (z.B. $E=V$). Dann ist $S\cup E$ ein Erzeugendensystem von $V$ mit $S\subset S\cup E$, das Basislemma liefert dann die gesuchte Basis.

\paragraph{Bemerkung und Definition}
    Strikt genommen haben wird den Basisergänzungssatz (BES) nur unter der Annahme bewiesen, dass $V$ endlich erzeugt sei, d.h. $V$ ein endliches Erz. Syst. $E$ besitzt, $V=[E]$ und $\#E<\infty$.
\paragraph{Bemerkung}
    Wir haben für den BES die (in diesem Falle einfachere) Mengenschreibweise (anstelle der Familienschreibweise) verwendet.
\paragraph{Bemerkung}
    Ähnlich kann man einen Verkürzungssatz beweisen: Jedes Erzeugendensystem eines Vektorraums $V$ kann zu einer Basis verkürzt werden.

\paragraph{Austauschlemma}
    Seien $B,B' \subset V$ Basen von $V$. Dann gilt:
    \begin{equation*}
        \forall b\in B \exists b' \in B': (B\setminus\{b\})\cup\{b'\} \text{ ist Basis}
    \end{equation*}
    
\paragraph{Beweis}
    Sei $b\in B$ beliebig gewählt und $S:= B\setminus \{b\}$. Da $B$ lin. unabh. ist, gilt $b\notin [S] \Rightarrow \emptyset \neq V\setminus [S] = [B']\setminus [S] \Rightarrow B' \not\subset [S]$, d.h. es existiert $b' \in B'$ mit $b' \notin [S]$. Wir zeigen, dass $B'' := S\cup \{b'\} = (B\setminus\{b\})\cup \{b'\}$ Basis ist. $B''$ ist Erzeugendensystem: Da $b'\in [B]$ existiert $(x_j)_{j\in B}$ mit $$b' = \sum_{j\in B} jx_j $$ mit $x_b \neq 0$, da $b' \notin [S]$.

    Damit ist $b=(b'-\sum_{j\in B} jx_j)\frac{1}{x_b} \in [B'] \Rightarrow V = [B] \subset [B'' \cup \{b\}] \subset [B'']$.
    $B''$ ist lin. unabh.. $B''$ ist Erz. Syst. und $S\subset B' = S \cup \{b'\}$ lin unabh., kann also (nach Basislemma) erg"anzt werden zu einer Basis $\tilde{B}$ mit $S\subset \tilde{B}\subset B''$.
    Da $[S] \neq V$ gilt $\tilde{B} \neq S$ und damit $\tilde{B} = B''$ Basis, insbesondere lin. unabh.
    
\paragraph{Bemerkung}
    Hier haben wir die Familienschreibweise (mit $B$ bzw. $S$ als Indexmenge) verwendet, um Linearkombinationen darzustellen.
    
\paragraph{Basissatz}
    Sei $V$ ein endlich erzeugter $K$-VR, $V=[E]$ mit $\#E < \infty$. Dann gilt:
    \begin{enumerate}[(i)]
   		\item $V$ besitzt eine endliche Basis $B$ mit $n:= \#B \leq \#E$.
    	\item Ist $B'\subset V$ eine Basis von $V$, so ist $\#B' = \#B = n$.
    \end{enumerate}
    
\paragraph{Beweis}
    \begin{enumerate}[(i)]
        \item  Dies folgt direkt aus dem Basislemma (mit $S=\emptyset$).
        \item Seien $B,B'$ Basen von V, $B = (b_1,...,b_n)$.\\
        Annahme: $\#B' < n, B' = (b'_1,...,b'_k)$ mit $k < n$. Wiederholte Anwendung des Austauschlemmas auf die Basen $B$ und $B'$ liefert nach (spätestens) $k+1\leq n$ Schritten einen Widerspruch zur linearen Unabhängigkeit der neuen Basis $B''$, da Vektoren $b'_i$ doppelt vorkommen müssen.\\
        Annahme: $\#B' < n, B' = (b'_1,...,b'_n,b'_{n+1})$: Das gleiche Argument mit vertauschten Rollen der Basen führt wieder zum Widerspruch.
     \end{enumerate}

\paragraph{Definition}
    Sei $V$ ein $K$-VR, die Dimension von V ist dann:
    \begin{itemize}
        \item $\dim V:= \#B$, falls V endlich erzeugt und $B$ eine Basis von $V$ ist;
        \item $\dim V:= \infty$, falls $V$ nicht endlich erzeugt ist.
    \end{itemize}
\paragraph{Bemerkung}
    Nach dem Basissatz hängt $\dim V = \#B$ (falls $V$ endlich erz.) nicht von der Basis $B$ ab, d.h. $\dim V$ ist wohldefiniert.
\paragraph{Beispiel}
    $\dim K^n = \#\{e_1,...,e_n\} = n$ (Standardbasis).

\paragraph{Korollar:} Ist $ V $ ein $ K $-VR mit $\dim V =: n\in \mathbb{N}$. Dann gilt:
    \begin{enumerate}[(i)]
    	\item Ist $S \subset V$ linear unabhängig, so ist $\# S \subseteq n$ und $\# S = n$ genau dann, wenn $ S $ Basis ist
    	\item Ist $E \subset V$ Erzeugendensystem, so ist $\#E \geq n$, bzw. $\#E = n$ genau dann, wenn $ E $ eine Basis ist.
    \end{enumerate}
    
\paragraph{Bemerkung:}
	Insbesondere: Ist $U\subset V$ UVR mit $\dim U=\dim V < \infty$, so gilt $ U=V $.
   
\paragraph{Beweis (Korollar):}
    \begin{enumerate}[(i)]
    	\item Ist $ S $ linear unabhängig, so existiert (nach BES) eine Basis $ B $ von $ V $ mit 
			 \begin{gather*}
			    S\subset B\Leftrightarrow \left\{
				    \begin{array}{l}
					    \#S \leq \#B\\
						\#S = \#B \Leftrightarrow S = B
					\end{array}
			    \right.
		    \end{gather*}
		    \item Analog (mit Basislemma), siehe Aufgabe 23.
	 \end{enumerate} 
