\documentclass[12pt,a4paper,parskip=half-,DIV=15]{scrreprt}
\usepackage[utf8]{inputenc}
\usepackage[T1]{fontenc}
\usepackage{lmodern}%schoeneres Schriftbild
\usepackage[ngerman]{babel}%deutsche Silbentrennung
\usepackage[onehalfspacing]{setspace}
\usepackage{tikz}%Zeichnungen
\usetikzlibrary{arrows,positioning}
\newcommand{\equal}{=}
\usepackage{amsmath,amsfonts,amssymb}%Mathematik-Pakete
\usepackage{graphicx}
\usepackage{paralist}%enumerate mit roemischen Zahlen
\usepackage{float}%fuer H Positionierung

\author{Christoph Fritz, djangonightfall, puenka}
\title{Skript Lineare Algebra \& Geometrie 1, Hertrich-Jeromin}

\setcounter{chapter}{-1}%Grundlagen bei 0

\let\hom\relax
\DeclareMathOperator{\Char}{Char}
\DeclareMathOperator{\hom}{Hom}
\DeclareMathOperator{\rg}{rg}
\DeclareMathOperator{\dfkt}{def}

\begin{document}
\maketitle
%\tableofcontents
% % % %Kapitel 0 - Grundlagen % % % %
\chapter{Grundlagen}

\section*{Vorbereitungen, logische Symbolik}
	Es existieren zwei Methoden zur präzisen Formulierung:
	\begin{itemize}
	\item Funktion einer Formulierung wird präzisiert durch:
		\begin{itemize}
			\item Definition: Begriffsklärung
			\item Satz (Lemma, Proposition, Korollar): Aussage über einen (mathematischen) Sachverhalt
			\item Beweis: eine (logische) Argumentationskette, die erklärt, warum ein Satz/Lemma wahr ist
			\item Bemerkung, Beispiel: zusätzliche Information/Illustration, die oft Eigenarbeit (Beweis) erfordert
		\end{itemize}
	\item Formeln und (logische) Symbole werden verwendet:
		\begin{itemize}
			\item $\forall$ -- All-Quantor: \glqq für alle\grqq
			\item $\exists(!)$ -- Existenz-Quantor: \glqq es existiert (genau) ein\grqq
			\item $\lnot$ -- logische Verneinung: $\lnot A$ ist wahr, wenn $A$ falsch ist
			\item $\land ,\lor$ -- logisches \glqq und\grqq{} und \glqq oder\grqq
			\item $\Rightarrow ,\Leftrightarrow$ -- Implikation und Äquivalenz
		\end{itemize}
	\end{itemize}

	\begin{figure}[H]\centering
		\begin{tabular}{c|c|c|c|c|c|c}
			$A$ & $B$ & $\lnot A$ & $A\land B$ &$A\lor B$&$A \Rightarrow B$ & $A\Leftrightarrow B$\\\hline
			w & w & f & w & w & w & w\\
			w & f & f & f & w & f & f\\
			f & w & w & f & w & w & f\\
			f & f & w & f & f & w & w\\
		\end{tabular}
	\caption{Wahrheitstafel}
	\end{figure}

	Beispiele:
	\begin{itemize}
		\item Implikation: Für $x,y\in\mathbb{R}: xy = 0 \Rightarrow (x = 0\lor y = 0)$
		\item Für Aussagen $ A $ und $ B $ gilt: $(A\Rightarrow B)\Leftrightarrow (\lnot A \lor B)$, Beweis durch Wahrheitstafel
	\end{itemize}
	
	\begin{figure}[H]\centering
		\begin{tabular}{c|c|c|c|c|c}
			$A$ & $B$ & $\lnot A$ & $\lnot A\lor B$ & $A \Rightarrow B$ & $(A\Rightarrow B)\Leftrightarrow (\lnot A \lor B)$\\\hline
			w & w & f & w & w & w \\
			w & f & f & f & f & w \\
			f & w & w & w & w & w \\
			f & f & w & w & w & w \\
		\end{tabular}
	\caption{Beweis durch Wahrheitstafel}
	\end{figure}

\paragraph{Bemerkung (Kommutativität):}
	$\land$, $\lor$, und $\Leftrightarrow$ sind kommutativ (symmetisch), $\Rightarrow$ jedoch nicht, d.h.:
	\begin{gather*}
		(A\land B)\Leftrightarrow (B\land A)\\
		(A\lor B)\Leftrightarrow (B\lor A)\\
		(A\Leftrightarrow B)\Leftrightarrow (B\Leftrightarrow A)\\
		(A\Rightarrow B)\nLeftrightarrow (B\Rightarrow A)\\
	\end{gather*}
	
	weil beispielsweise formal gilt: $x,y\in\mathbb{R}: x = 0 \Rightarrow xy = 0$, aber nicht $xy = 0 \Rightarrow x = 0$.

\paragraph{Bemerkung (Beweisformen der Implikation):}
	Um eine Implikation $A\Rightarrow B$ zu zeigen, bedient man sich häufig auch folgender Äquivalenzen:
	\begin{equation*}
		(A\Rightarrow B)\Leftrightarrow
		\begin{cases}
			\lnot B\Rightarrow \lnot A&\text{(Indirekter Schluss)}\\
			\lnot (A\land \lnot B)&\text{(Widerspruchsbeweis)}
		\end{cases}
	\end{equation*}

\paragraph{Beispiel:}
	Für reelle Zahlen $x,y\in\mathbb{R}$ gilt:
	\begin{equation*}
		\left((xy = 0)\Rightarrow (x=0 \lor y=0)\right) \Leftrightarrow \left((xy=0 \land x \neq 0)\Rightarrow (y =0)\right)
	\end{equation*}
	
	bzw. allgemein:
	\begin{equation*}
		(A\Rightarrow (B\lor C))\Leftrightarrow ((A\land\lnot B)\Rightarrow C)
	\end{equation*}

\paragraph{Bemerkung (Mengenlehre):}
	Die Ähnlichkeit mit der Mengensymbolik ist nicht zufällig, z.B. Mengen $X, Y$:
	\begin{gather*}
		(x\in X\cap Y)\Leftrightarrow (x\in X\land x\in Y)\\
		(x\in X\cup Y)\Leftrightarrow (x\in X\lor x\in Y)\\
		(X\subset Y) \Leftrightarrow \{\forall x : (x\in X \Rightarrow x\in Y)\}
	\end{gather*}

\section*{Abbildungen}
\paragraph{Definition:}
	Eine Zuordnung $f: X\to Y$ zwischen zwei Mengen $X$ und $Y$ heißt eine Abbildung, falls $\forall x\in X: \exists ! y\in Y: y=f(x)$.

	X heißt der Definitionsbereich der Abbildung und $f(X):=\{f(x)\mid x\in X \}\subseteq Y$ das Bild.

	Eine Abbildung $f: X\to Y$ heißt
	\begin{itemize}
		\item injektiv, falls $\forall x,x'\in X:f(x) = f(x') \Rightarrow x=x'$
		\item surjektiv, falls $\forall y\in Y:\exists x\in X: y = f(x)$
		\item bijektiv, falls $\forall y\in Y:\exists !x\in X: y = f(x)$
	\end{itemize}

\paragraph{Beispiel:}
	Mit $X=Y=\mathbb{R}$ definiert
	\begin{itemize}
		\item die Relation $x^2 = y$ eine Abbildung $f:X\to Y, x\mapsto f(x)=x^2$
		\item die Relation $x=y^2$ keine Abbildung $f:X\to Y$, denn
		\begin{itemize}
			\item für ein $x$ gibt es zwei $y$-Werte
			\item $x < 0$ ist nicht definiert
		\end{itemize}
	\end{itemize}

\paragraph{Beispiel:}
	Die Identität $id_X :X\to X, x\mapsto id_X(x):= x$ ist eine bijektive Abbildung.
	
\paragraph{Bemerkung:}
	Eine Abbildung ist genau dann bijektiv, wenn sie injektiv und surjektiv ist.
	
\paragraph{Definition:}
	Sind $ f:X\to Y $ und $ g:Y\to Z$ Abbildungen, so ist ihre Komposition/Verkettung die Abbildung $ g\circ f:X\to Z, x\mapsto (g\circ f)(x):= g(f(x)) $.
	
\paragraph{Beispiel:}
	Seien $ X = Y = Z = \mathbb{R} $ und $ f:X\to Y, x\mapsto f(x) :=x^2 $, $ g:Y\to Z, y\mapsto g(x):=y^3 + y $, so ist die Verkettung $ g\circ f: X\to Z, x\mapsto (g\circ f)(x) = (x^2)^3+x^2 = x^6 + x^2 $.

\section*{Inverse}
\paragraph{Lemma:}
	Seien $ f:X\to Y $ und $ g:Y\to X $ Abbildungen. Dann gilt:
	\begin{enumerate}[i)]
		\item ist $ g $ Linksinverse von $ f $, d.h. $ g\circ f = id_X $, so ist f injektiv
		\item ist $ g $ Rechtsinverse von $ f $, d.h. $ f\circ g = id_Y$, so ist f surjektiv
		\item ist $ g $ Links- und Rechtsinverse von $ f $, so heißt $ g =f^{-1}$ Inverse von $ f $
	\end{enumerate}

\paragraph{Beispiel:}
	$ f:\mathbb{N}\to \mathbb{N}, n\mapsto f(n):= n+1 $ hat Linksinverse
	\begin{equation*}
		g:\mathbb{N} \to \mathbb{N}, n\mapsto g(n):=
		\begin{cases}
			15700, & \text{falls } n=0\\
			n-1, & \text{falls } n\neq 0
		\end{cases}
	\end{equation*}

	Tatsächlich ist $ f $ injektiv, da
	\begin{equation*}
		\forall n,n'\in \mathbb{N} : n+1 = f(n) = f(n') = n'+1 \Rightarrow n=n'
	\end{equation*}
	
	jedoch $ f(\mathbb{N}) = \mathbb{N}\setminus \{0\} $, daher kann keine Rechtsinverse existieren.

\paragraph{Beweis (Lemma):}
	Zwei Aussagen sind zu beweisen:
	\begin{enumerate}[i)]
		\item Sei $ g $ Linksinverse von $ f $. Dann gilt für $ x,x'\in X $ mit \\$ f(x) = f(x'): x = g(f(x)) = g(f(x')) = x' $, also ist $ f $ injektiv.
		\item Sei $g $ Rechtsinverse von $ f $ und $ y\in Y $. Setze $ x:= g(y)\in X $, dann gilt f(x) = f(g(y)) = y. Damit existiert zu jedem $ y\in Y $ (mindestens) ein $ x = g(y) $, sodass  $ y=f(x) $.
	\end{enumerate}

% % % %Kapitel 1 - Lineare Räume und Abbildungen % % % %
\chapter{Lineare Räume und Abbildungen}
\section{Von Geometrie zu Algebra}
	Euklids führte in den \glqq Elementen\grqq{} (ca. 300 v. Chr.) das bis heute gültige Schema ein:
	\begin{itemize}
		\item Definition
		\item Axiom/Postulat
		\item Lehrsatz
		\item Beweis
	\end{itemize}

\subsection{Parallelenaxiom/-problem (Euklid, Formulierung nach Playfair)}
	Es existiert genau eine Parallele $ g' $ zum Punkt $ P \notin g $ zur Geraden $ g $.

	Kann das Axiom aus den anderen Axiomen hergeleitet/bewiesen werden? Nein, denn es existieren nichteuklidische, hyperbolische Geometrien (18. Jh.) in denen es mehrere derartige Parallelen gibt. Als Beispiel lässt sich eine Geometrie anführen, die nicht auf einer Ebene sondern auf einem Kreis operiert. Dort lassen sich zu einer Sekante mehrere parallele Sekanten betrachten (also Sekanten, die die ursprüngliche nicht schneiden).

	\begin{figure}[H]
		\begin{minipage}{.45\textwidth}
			\begin{tikzpicture}[line cap=round,line join=round,>=triangle 45,x=1.0cm,y=1.0cm]
				\clip(-1.69,-0.64) rectangle (4.14,2.83);
				\draw [domain=-1.69:4.14] plot(\x,{(-1--1*\x)/1});
				\draw [domain=-1.69:4.14] plot(\x,{(-0--1*\x)/1});
				\draw (0.6,1) node[] {P};
				\draw (1.58,0.16) node[] {g};
				\draw (1.52,1.78) node[] {g'};
				\begin{scriptsize}
				\fill [color=blue] (1,1) circle (2pt);
				\end{scriptsize}
			\end{tikzpicture}
		\end{minipage}
		\begin{minipage}{.45\textwidth}
			\begin{tikzpicture}[line cap=round,line join=round,>=triangle 45,x=1.0cm,y=1.0cm]
				\clip(-2.24,-3.38) rectangle (3.15,1.76);
				\draw(0,0) circle (1cm);
				\draw (-0.94,0.35)-- (0.66,0.75);
				\draw (-0.13,0.74) node[] {g};
				\draw (0.32,-0.56) node[] {P};
				\draw (-1,-0.02)-- (0.88,-0.48);
				\draw (-0.35,-0.94)-- (0.91,0.42);
				\begin{scriptsize}
				\fill [color=blue] (0.22,-0.32) circle (1.5pt);
				\end{scriptsize}
			\end{tikzpicture}
		\end{minipage}
	\end{figure}

\paragraph{Was ist eine Geometrie?}
	Eine Geometrie ist durch eine Menge X und eine auf X operierende Transformationsgruppe gegeben.
%%%%%%%%%%%%%%%%% BEGINN VO3-20151013 %%%%%%%%%%%%%%%%%%%%%

\subsection{Definition (Gruppe)}
	\begin{Definition}[Gruppe]
		Ein Paar $(G,\circ)$ bestehend aus einer Menge $G$ und einer Verknüpfung $(\circ : G\times G \to G) : (g,h) \mapsto g \circ h$ heißt Gruppe, falls:

	\begin{enumerate}[(i)]
		\item $\forall f,g,h\in G : f\circ (g\circ h) = (f\circ g)\circ h$ (Assoziativität)
		\item $\exists e\in G\forall g\in G : e\circ g = g$ (Existenz eines neutralen Elements)
		\item $\forall g \in G \exists g^{-1} \in G : g^{-1}\circ g = e$ (Existenz eines inversen Elements)
	\end{enumerate}
	
	Die Gruppe heißt kommutativ oder abelsch, falls zusätzlich gilt:
	\begin{equation*}
		\forall g,h\in G: g\circ h = h\circ g \text{ (Kommutativität)}
	\end{equation*}
	\end{Definition}

\paragraph{Bemerkung}
	Das ist eine axiomatische Definition, d.h. der Begriff \glqq Gruppe\grqq{} wird durch (aus vielen (!) Beispielen abstrahierten) \glqq Rechenregeln\grqq{} definiert.
\paragraph{Beispiel}
	Die rationalen Zahlen $\mathbb{Q}$ bilden mit der Addition eine Gruppe $(\mathbb{Q} ,+)$.
	Die rationalen Zahlen ohne $0$, $\mathbb{Q}^{\times} := \mathbb{Q}\setminus \{0\}$, bilden mit der Multiplikation eine Gruppe $(\mathbb{Q}^\times ,\cdot)$.

\subsection{Definition (Gruppenoperation)}
	\begin{Definition}[Gruppenoperation]
		Sind $(G,\circ )$ eine Gruppe und $X$ eine Menge, so heißt eine Abbildung
		\[ \cdot : G\times X\to X, (g,x)\mapsto g\cdot x \]
	
	eine Gruppenoperation (von $(G,\circ )$ auf $X$), falls

	\begin{enumerate}[(i)]
		\item $\forall g,h\in G :\forall x\in X: g\cdot (h\cdot x) = (g\circ h)\cdot x$ (entspricht nicht der Assoziativität!)
		\item $\forall x\in X: e\cdot x = x$ für das neutrale Element $e$ der Gruppe $(G,\circ )$
	\end{enumerate}
	$(G,\circ )$ heißt dann Transformationsgruppe von X.
	\end{Definition}

\paragraph{Bemerkung}
	Operiert $G$ (kurz für $(G,\circ )$, aus dem Zusammenhang ersichtlich) auf $X$, so ist für jedes $g\in G$ die Abbildung
		\[ g:X\to X, x\mapsto g\cdot x \]
	eine bijektive Abbildung von $X$ auf sich. Wegen der Axiome (i) und (ii) aus der Definition erhält man $g^{-1}: X\to X$ als Inverse der Abbildung.
	
\subsection{Beispiel und Definition (Permutationsgruppe)}
	\begin{Definition}[Permutationsgruppe]
		Die bijektiven Abbildungen einer Menge $X$ auf sich, 
		\[ G:= \{g:X\to X\mid g \text{ bij}\}, \]
	bilden (mit der Komposition $\circ$) eine (Transformations-)Gruppe $(G,\circ )$ (die auf $X$ operiert): die Permutationsgruppe oder symmetrische Gruppe $S_X$ von $X$. Für $X=\{1,2,...,n\}$ schreibt man auch $S_n$ statt $S_{\{1,...,n\}}$.
	\end{Definition}
\paragraph{Bemerkung}
	Im Gegensatz zu allgemeinen Abbildungen stimmen in (Permutations-)Gruppen Links- und Rechtsinverse stets überein.
\subsection{Lemma (Eindeutigkeit des neutralen Elements)}
	\begin{Lemma}[Eindeutigkeit des neutralen Elements]
		Das neutrale Element einer Gruppe $(G,\circ )$ ist eindeutig und $\forall g\in G: g\circ e = g$. Weiters: 
	\begin{equation*}
		\forall g\in G \exists ! g^{-1} \in G: g^{-1}\circ g = g \circ g^{-1} = e
	\end{equation*}
	\end{Lemma}

\paragraph{Beweis}
	Sei $g\in G$ gegeben und (gemäß Gruppenaxiom (iii)):
	\begin{itemize}
		\item $h:= g^{-1}$ (Linksinverse von $g$)
		\item $k:= h^{-1}$ (Linksinverse von $h$)
	\end{itemize}
	
	Damit berechnen wir (multiplikative Schreibweise: $a\circ b = ab$):
	\begin{gather*}
		hg = e = kh = k((hg)h) = k(h(gh)) = (kh)(gh) = gh\\
	\text{und }	ge = g(hg) = (gh)g = eg
	\end{gather*}
	
	Jedes (links-)neutrale Element $e$ ist also auch rechtsneutral:\hfill
	$\forall g\in G: eg = ge = g$
	
	und ist $e'\in G$ auch neutrales Element, dann:\hfill
	$ e' = ee' = e'e = e $

	Weiters ist jedes (Links-)Inverse auch rechtsinvers:\hfill
	$ \forall g \in G: gg^{-1}=g^{-1}g = e $

	und sind $h,h'\in G$ Inverse von $g\in G$, so gilt:\hfill
	$ h' = h'(gh) = (h'g)h = h $

	d.h. Eindeutigkeit des Inversen.

\subsection{Definition (Körper)}
	\begin{Definition}[Körper]
		Ein Tripel $(K,+,\cdot)$, bestehend aus einer Menge $K$ und zwei Verknüpfungen
	\begin{align*}
		+:&K\times K\to K,(x,y)\mapsto x+y\\
		\cdot : &K\times K\to K, (x,y)\mapsto xy
	\end{align*}
	
	heißt Körper, falls:
	\begin{enumerate}[(i)]
		\item $(K,+)$ ist abelsche Gruppe (mit neutralem Element $0$ und inversem Element $-x$ von $x$)
		\item $(K^\times,\cdot)$ ist abelsche Gruppe (mit neutralem Element $1$ und inversem Element $\frac{1}{x} = x^{-1}$ von $x\in K^\times$)
		\item die Distributivgesetze gelten:
			\[ \forall x,y,z\in K :\begin{cases}x\cdot (y+z) = xy+xz\\ (x+y)\cdot z = xz+yz \end{cases} \]
	\end{enumerate}
	\end{Definition}

\paragraph{Bemerkung}
	In einem Körper gilt stets:
	\begin{gather*}
		0\cdot x = x\cdot 0 = 0 \Rightarrow\\
		0\cdot x = (0+0)\cdot x = 0\cdot x + 0\cdot x\\
		\Rightarrow 0 = 0\cdot x + (-(0\cdot x)) \Rightarrow 0 = 0\cdot x.
	\end{gather*}
	
	Insbesondere folgt damit: $\forall x,y\in K: xy = yx$ (nicht nur für $K^\times$ (Axiom)).
	
\paragraph{Beispiel}
	Die rationalen Zahlen $\mathbb{Q}$, die reellen Zahlen $\mathbb{R}$ und die komplexen Zahlen $\mathbb{C}$ bilden mit den üblichen Verknüpfungen Körper.

\paragraph{Bemerkung und Beispiel}
	Aufgrund der Axiome (i) und (ii) enthält $ K $ mindestens 2 Elemente, also $ \# K \geq 2 $, nämlich:
	\begin{itemize}
		\item $ 0 $, das neutrale Elemente bezüglich $+$
		\item $1 (\neq 0)$, das neutrale Elemente (in $K^\times$ = $K\setminus\{0\}$) bezüglich $\cdot$
	\end{itemize}
	
	Es gibt auch einen Körper mit genau 2 Elementen $(\{0,1\},+,\cdot)$, wobei
	
	\begin{minipage}{0.45\textwidth}
		\begin{equation*}
			\begin{tabular}{c|c|c}
				$+$ & 0 & 1\\\hline
				0 & 0 & 1\\
				1 & 1 & 0\\
			\end{tabular}
		\end{equation*}
	\end{minipage}
	\begin{minipage}{0.45\textwidth}
		\begin{equation*}
			\begin{tabular}{c|c|c}
				$\cdot$ & 0 & 1\\\hline
				0 & 0 & 1\\
				1 & 1 & 1\\
			\end{tabular}
		\end{equation*}
	\end{minipage}
	
	Dieser Körper wird auch $\mathbb{Z}_2$ bezeichnet.

\subsection{Bemerkung und Definition (Charakteristik)}
	\begin{Definition}[Charakteristik]
		In $\mathbb{Z}_2: 1 + 1 = 0$. Allgemeiner definiert man die Charakteristik eines Körpers $(K,+,\cdot)$ (mit neutralen Elementen 0 und 1 von + bzw. $\cdot$) durch

	\begin{equation*}
		\Char(K):=
		\begin{cases}
			0,\text{falls } \forall n \in \mathbb{N}^\times: \sum_{j = 1}^{n} 1 \neq 0\\
			\min\{n \in \mathbb{N}^\times\mid \sum_{j = 1}^{n} 1 = 1+ ... + 1 = 0\}
		\end{cases}
	\end{equation*}
	\end{Definition}
	
	z.B. $\Char(\mathbb{Z}_2) = 2$, da
	\begin{gather*}
		\{n\in\mathbb{N}^\times\mid 1+...+1=0\}=\\
		=\{n\in\mathbb{N}^\times\mid n=0 \text{ mod } 2\}=\\
		=\{n\in\mathbb{N}^\times\mid n \text{ gerade}\}\\
		\text{und damit: }	\min\{n\in\mathbb{N}^\times\mid 1+...+1=0\}=2
	\end{gather*}
	
	Wir werden mitunter $\Char(K,+,\cdot)\neq 0$ oder (öfter) $\Char(K,+,\cdot)=2$ ausschließen (müssen).

% % % % Chapter 1 Section 2 % % % %
\section{Translationen und Vektoren}

\begin{tikzpicture}[scale=1.5,>=triangle 45]
	\draw[->,color=black] (-0.1,0) -- (10,0);
	\draw[->,color=black] (0,-0.1) -- (0.,4);
	
	\coordinate[label=left:$x$] (x) at (1,2);
	\coordinate[label=below:$y\equal\tau_v(x)$] (y) at (5,1.5);
	\coordinate[label=above:$y'\equal\tau_w(x)$] (y') at (2,3.5);
	\coordinate (z) at (6,3);
	
	\draw [fill] (x) circle (.5pt);
	\draw [fill] (y') circle (.5pt);
	\draw [fill] (y) circle (.5pt);
	\draw [fill] (z) circle (.5pt);
	
	\draw [->] (x) to node[below left]{$ v $} (y);
	\draw [->] (x) --node[above left]{$ w $} (y');
	\draw [->] (y) --node[below right]{$ w $} (z);
	\draw [->] (y') --node[above right]{$ v $} (z);
	\draw [->] (x) --node[above]{$ v+w $} node[below]{$ w+v $} (z);
	
	\draw (z) node[above right] {$z = \tau_w(y)=(\tau_w\circ\tau_v)(x)=\tau_{w+v}(x)$};
	\draw (z) node[below right] {$z' = \tau_v(y')=(\tau_v\circ\tau_w)(x) = \tau_{v+w}(x)$};
	\draw (4,0.5) node[] {Translationen \glqq der\grqq{} Ebene bilden eine abelsche Gruppe};
\end{tikzpicture}

\begin{tikzpicture}[scale=1.5, >=triangle 45]
	\draw[->,color=black] (-0.1,0) -- (9,0);
	\draw[->,color=black] (0,-0.1) -- (0,4);
	
	\coordinate[label=left:$ x $] (x) at (1,3);
	\coordinate[label=right:$ y \equal \tau_v(x) \equal (\tau_{\frac{v}{2}} \circ \tau_{\frac{v}{2}})(x) \equal \tau_{\frac{v}{2} + \frac{v}{2}}(x) $] (y) at (3,1);
	
	\draw [fill] (x) circle (.5pt);
	\draw [fill] (y) circle (.5pt);
	
	\draw [->] (1,2.8) --node[left]{$v$} (3,.8);
	\draw [->] (1,3.2) --node[above] {$\frac{v}{2}$} (2,2.2);
	\draw [->] (2,2.2) --node[above] {$\frac{v}{2}$} (3,1.2);
	
	\draw (5,3) node[text width = 7cm] {Translationen kann man \glqq strecken\grqq{}, sodass gewisse Rechengesetze gelten.};
\end{tikzpicture}

\paragraph{Definition}
	Sei $K$ ein Körper. Eine Menge $V$ mit zwei Abbildungen

	\begin{align*}
		 +&: V \times V \to V:(v,u)\mapsto v+w,\\
		 \cdot &: K \times V \to V:(x,v)\mapsto vx,
	\end{align*}
	
	heißt Vektorraum über $K$ ($K$-VR), falls gilt:
	\begin{enumerate}[(i)]
		\item $(V,+)$ ist eine abelsche Gruppe
		\item $\forall v\in V: v\cdot 1=v$ und\\
                      $\forall x,y \in K\ \forall v\in V: (vx)y = v(xy)$
		\item $\forall x,y \in K\ \forall v\in V: v(x+y) = vx + vy$\\
                      $\forall x\in K\ \forall v,w\in V: (v+w)x = vx + wx$
	\end{enumerate}

\paragraph{Bemerkung}
	Wir notieren die Skalarmultiplikation als Rechtsmultiplikation (Skalar steht rechts):
	\begin{equation*}
		\cdot: K \times V \to V : (x,v) \mapsto vx
	\end{equation*}

\paragraph{Beispiel}
	Die Translationen eines affinen Raumes bilden einen Vektorraum (vgl. mit der Skizze oben): Diese Beispiel wird im nächsten Kapitel repräsentiert.
	
\paragraph{Beispiel}
	Jeder Körper $ K $ ist ein $ K $-VR (Vektorraum über sich selbst): das ist ein (trivialer) Spezialfall des folgenden...
	
\paragraph{Beispiel und Definition}
	Ist $ I $ eine Menge und $ K $ ein Körper, so bilden die $ K $-wertigen Abbildungen
	\begin{equation*}
		v: I \to K: i \mapsto v_i
	\end{equation*}

	einen Vektorraum mit der punktweise definierten Addition und Skalarmultiplikation:
	\begin{gather*}
		I\ni i \mapsto (v+w)_i := v_i+w_i\in K\\
		i \mapsto (vx)_i := v_ix \in K
	\end{gather*}

	Dieser Vektorraum wird mit $K^{I}$ bezeichnet und Standardvektorraum (über I und K) genannt. Im Falle $ I=\{1,...,n\} $ schreibt man auch $K^{n} := K^{\{1,...,n\}}$

\paragraph{Bemerkung und Definition}
	Anstelle der normalen Schreibweise
	\begin{equation*}
		I\ni i \mapsto v(i) \in K
	\end{equation*}

	für die Auswertung einer Abbildung  $v: I \to K$ um einen Punkt $i\in I$ haben wir die Indexschreibweise verwendet.
	\begin{equation*}
		I\ni i \mapsto v_i \in K
	\end{equation*}

	Wir haben damit eine Abbildung $v: I \to K$ als Familie von
	\begin{equation*}
		(v_i)_{i\in I}
	\end{equation*}

	über der Indexmenge I aufgefasst -- der Begriff Familie ist ein \glqq alternativer\grqq{} Begriff für Abbildungen.
	
\paragraph{Beispiel}
	Sei $i$ eine \glqq Zahl\grqq{} mit $i^2=-1$ ($i$ entspricht nicht dem Element der Indexmenge aus dem vorherigen Abschnitt). Die komplexen Zahlen
	\begin{equation*}
		\mathbb{C}:=\{{x+iy\mid x,y\in \mathbb{R}}\}
	\end{equation*}
 
	bilden mit der Addition und Multiplikation einen Körper:
	\begin{align*}
		+&:\mathbb{C}\times \mathbb{C} \to \mathbb{C}: ((x+y),(x'+y')) \mapsto ((x+iy)+(x'+iy')) := (x+x')+i(y+y')\\
		\cdot &:\mathbb{C}\times \mathbb{C} \to \mathbb{C}: ((x+iy),(x'+iy'))\mapsto (x+iy)\cdot (x'+iy') :=(xx'-yy')+i(xy'+x'y)
	\end{align*}

	Die komplexen Zahlen $\mathbb{C}$ bilden einen $\mathbb{R}$-VR mit
	\begin{equation*}
		+:\mathbb{C}\times\mathbb{C}\to\mathbb{C}
	\end{equation*}

	wie oben und der Skalarmultiplikation
	\begin{equation*}
		\cdot:\mathbb{R}\times\mathbb{C}\to\mathbb{C}:(x',(x+iy))\mapsto(x+iy)x':=xx'+iyx'
	\end{equation*}

	Diese Skalarmultiplikation ist also gerade die Einschränkung der komplexen Multiplikation auf $\mathbb{R}\times\mathbb{C}$ wobei die Identifikation
	\begin{equation*}
		\mathbb{R}\cong \{{x+iy\in\mathbb{C}\mid y=0}\}
	\end{equation*}

	verwendet wird.
% % % %VO5 % % % %
\subsection{Untervektorräume \& lineare Hülle}
\paragraph{Definition}
	Eine Teilmenge $U\subset V$ eines $K$-VR $V$ heißt Unter(vektor)raum (UVR), falls $U$ mit der eingeschränkten Addition und Skalarmultiplikation
	\begin{align*}
		 ^+    & \mid_{U\times U}: U\times U \to V,(v,w) \mapsto v+w \\
		 \cdot & \mid_{K\times U}: K\times U \to V,(x,v) \mapsto vx
	\end{align*}

	selbst ein Vektorraum ist, d.h. wenn insbesondere
	\begin{gather*}
		\forall v,w \in U: v+w\in U \text{ und}\\
		\forall x\in K\forall v\in U: vx\in U.
	\end{gather*}

\paragraph{Bemerkung}
	Eine nicht-leere Teilmenge $U\subset V, U\neq\emptyset$, ist genau dann ein UVR, wenn die auf U eingeschränkten Operationen wohldefiniert sind, d.h. wenn $ U $ bzgl. $ + $ und $ \cdot $ abgeschlossen ist.

	Dies kann zum Unterraumkriterium zusammengefasst werden:
	\begin{equation*}
		U\subset V \text{ ist UVR }\Leftrightarrow 
 		 \begin{cases}
 		 	U\neq\emptyset\\
 		 	\forall v,w\in U\forall x\in K: vx+w\in U
 		 \end{cases}
	\end{equation*}

\paragraph{Beispiel}
	Sei $I=\{1,...,n\}$. Für jedes (feste) $i\in I$ ist
	\begin{equation*}
		U_i := \{v:I\to K\mid v_i =0\}
	\end{equation*}

	ein UVR von $K^n$, denn
	\begin{enumerate}
		\item $v = 0 \in U_i\text{, also } U_i \neq \emptyset$
		\item Seien $v,w\in U_i$, d.h. $v,w\in K^n$ mit $v_i =w_i =0$, und $x\in K$; dann gilt $(vx+w)_i = v_ix+ w_i = 0\cdot x + 0 = 0$, also: $vx+w\in U_i$ und damit ist $U_i$ UVR nach Unterraumkriterium.  
	\end{enumerate}
	
	Kein UVR von $K^n, n\geq 2$, ist jedoch die Menge
	\begin{equation*}
		N:=\{v:I\to K\mid v_1\cdot v_2 = 0\},
	\end{equation*}
  
	denn 
	\begin{enumerate}
		\item $N$ ist zwar nicht-leer, $N\neq \emptyset$, aber
		\item $^+\mid_{N\times N}: N\times N\to N$ nicht wohldefiniert: seien $v,w\in N$, so dass
			\begin{gather*}
				v_1=0, v_2=1\text{ }(v_3 ... v_n \text{ irrelevant})\\
				w_1=1, w_2 = 0\text{ }(w_3 ... w_n \text{ irrelevant})
			\end{gather*}
	\end{enumerate}
	
	dann gilt:
	\begin{gather*}
		(v+w)_1 = v_1 + w_1 = 0+1=1\\
		(v+w)_2 = v_2 + w_2 = 1+0 = 1
	\end{gather*}

	und damit
	\begin{equation*}
		(v+w)_1(v+w)_2 = 1 \Rightarrow v+w\notin N
	\end{equation*}

\paragraph{Bemerkung und Beispiel}
	In analoger Weise definiert man die Begriffe
	\begin{itemize}
		\item einer Untergruppe $H\subset G$ einer Gruppe $(G,\cdot)$, bzw.
		\item eines Unter- oder Teilkörpers $T\subset K$ eines Körpers $(K,+,\cdot )$
	\end{itemize}
	
	Z.B.: Jeder UVR $U\subset V$ eines $K$-VR $V$ bildet (mit der Addition) eine Untergruppe der Gruppe $(V,+)$.
    Und: In gleicher Weise ist eine nicht-leere (!) Teilmenge ein/e Unterkörper/-gruppe, falls die eingeschränkten Operationen wohldefiniert sind.
    
    Z.B.: ist $H\subset G$ eine Untergruppe, falls (Untergruppenkriterium):
    \begin{enumerate}
        \item $H\neq \emptyset$
        \item $\forall g,h\in H: g\circ h^{-1} \in H$
    \end{enumerate}
            
	Achtung: Inversenbildung muss im Kriterium explizit formuliert werden, sonst würde z.B.: $\mathbb{N}\subset\mathbb{Z}$ als Teilmenge von $(\mathbb{Z}, +)$ als Gruppe ein Gegenbeispiel liefern.
            
     Z.B.: 
     \begin{itemize}
        \item die Translationen bilden eine Untergruppe der Bewegungsgruppe
        \item $\mathbb{Q}\subset\mathbb{R}$ und $\mathbb{R}\cong \{x+iy\mid y=0\}\subset\mathbb{C}$ bilden Teilkörper von $\mathbb{R}$ bzw. $\mathbb{C}$.
     \end{itemize}

\paragraph{Lemma}
    Ist $(U_i)_{i\in I}$ eine Familie von UVR $U_i\subset V$ eines $K$-VR $V$, so ist ihr Schnitt
    \begin{equation*}
        U:= \bigcap_{i\in I}U_i =\{ u\in V\mid \forall i\in I: u\in U_i\}
    \end{equation*}
        
    ein UVR von $V$. (Beweis in Aufgabe 17)
    
\paragraph{Definition}
	Die lineare Hülle $[S]$ einer Teilmenge $S\subset V$ eines $ K $-VR $ V $ ist der Schnitt aller S enthaltenden UVR $U\subset V$:
	\begin{equation*}
		[S] := \bigcap_{S\subset U \text{UVR}} U
	\end{equation*}

	Die lineare Hülle einer Familie $(v_i)_{i\in I}$ von Vektoren $v_i\in V$ in einem $ K $-VR $ V $ ist:
        \begin{equation*}
        	[(v_i)_{i\in I}] := [\{v_i\mid i\in I\}]
        \end{equation*}

\paragraph{Bemerkung}
    $[S]$ ist ein UVR (nach Lemma) -- der \glqq kleinste\grqq{} UVR, der S enthält, d.h. ist $U\subset V$ UVR mit $S\subset U$, so gilt $[S]\subset U$; da aber $[S] = \bigcap_{S\subset \tilde{U}  \text{UVR}}\tilde{U}\subset U$,
    da $S\subset U$, also $U$ am Schnitt beteiligt ist.

\paragraph{Bemerkung}
	$[\emptyset ] = \{o\}$ und $[V] = V$.

\paragraph{Beispiel}
	Ist $U\subset V$ UVR, so gilt $[U] = U$.

\paragraph{Beispiel}
	$N=\{v:I\to K\mid v_1v_2=0\} \subset K^n,I=\{1,...,n\},n\geq 2$, hat lineare Hülle $[N]=K^n$.

\paragraph{Beispiel}
	Für $I=\{1,...,n\}$ und $i\in I$ definiere
	$e_i:I\to K , j\mapsto e_i(j):= \delta_{ij}$, wobei 
	\begin{equation*}
		\delta_{ij} :=
		\begin{cases}
			1,& \text{falls }i=j\\
			0,& \text{sonst}
		\end{cases}
	\end{equation*}
	das Kroneckersymbol bezeichnet.
	
	Dann ist die lineare Hülle der Familie $(e_i)_{i\in I}$
	\begin{equation*}
		[(e_i)_{i\in I}] = K^n
	\end{equation*}
	
	Nämlich: Da $[(e_i)_{i\in I}]\subset K^n$ ist, gilt für beliebige $x_1,...,x_n\in K$
	\begin{gather*}
		e_1x_1+...+e_nx_n,\in [(e_i)_{i\in I}]\\ \text{denn nach Unterraumkriterium: } (e_1x_1+...+(e_{n-1}x_{n-1}+(e_nx_n + 0))...) \in [(e_i)_{i\in I}]
	\end{gather*}
	
	da $[(e_i)_{i\in I}] \subset K^n$ UVR ist. Andererseits gilt für beliebiges $v\in K^n$:
	\begin{equation*}
		v=\sum^n_{i=1}e_iv(i): I\to K,
	\end{equation*}
	
	denn
	\begin{equation*}
		\forall j\in I: \left(\sum^n_{i=1} e_iv(i)\right)(j) = \sum^n_{i=1}e_i(j)v(i) = (\delta_{ij}) v(j) = v(j)
	\end{equation*}

	damit ist gezeigt, dass die beiden Abbildungen übereinstimmen; da $v\in K^n$ beliebig war, folgt $K^n \subset [(e_i)_{i\in I}]$
	
\paragraph{Definition}
	Seien $(v_i)_{i\in I}$ und $(x_i)_{i\in I}$ Familien in einem $ K $-VR bzw. dem Körper $ K $, wobei
	\begin{gather*}
		\# \{i\in I\mid x_i \neq 0\} < \infty\text{ ,also}\\
		\{ i\in I \mid x_i \neq 0\} = \{i_1,...,i_n\}\text{ für ein geeignetes } n\in \mathbb{N}
	\end{gather*}
	
	
	Dann heißt die endliche Summe
	\begin{equation*}
    	\sum_{i\in I} v_ix_i:= \sum^n_{j=1}v_{i_j}x_{i_j}\text{ eine Linearkombination.}
	\end{equation*}

\paragraph{Bemerkung}
	Die Bedingung
	\begin{equation*}
		\#\{i\in I \mid x_i\neq 0\} <\infty
	\end{equation*}
	
	garantiert, dass die Summe wohldefiniert ist $\rightarrow$ vgl. Reihen in der Analysis.
		
\paragraph{Lemma}
	Ist $(v_i)_{i\in I}$ $I \neq \emptyset$, Familie in einem $K$-VR, so gilt: 
	\begin{equation*}
		[(v_i)_{i\in I}] = \left\{\sum_{i\in I} v_ix_i| x: I\to K: \# \{i\in I| x_i \neq 0\}< \infty\right\},
	\end{equation*}
	
	d.h. die lineare Hülle der Familie ist die Menge aller Linearkombinationen der Familie.

\paragraph{Beweis}
	Wir zeigen (wie üblich) zwei Inklusionen:	

	$\supseteq$: z.z.: jede Linearkombination liegt in der Linearen Hülle. Sei also $(x_i)_{i\in I}$ eine geeignete Familie in $ K $, dann gilt:
	\begin{equation*}
		\sum_{i\in I} v_i x_i = v_{i_1} x_{i_1} + ... + (v_{i_n}x_{i_n}+0)
	\end{equation*}

	\begin{enumerate}[(i)]
		\item $\text{für } (v_{i_n}x_{i_n}+0) \in [...] \text{ nach UR-Kriterium}$
		\item für $v_{i_1}x_{i_1} + ... + (v_{i_n}x_{i_n}+0) \in [...]$ nach UR-Kriterium (nach n-Schritten)
	\end{enumerate}

	$\subseteq$: Z.z.: Lineare Hülle der Familie ist Teilmenge von U. Setze die Menge der Linearkombinationen einer Familie U $:= \{{\sum_{i\in I} v_ix_i| x: I\to K \text{ mit } \#\{{i\in I| x_i \neq 0\}} < \infty\}}$, offenbar gilt:
	\begin{equation*}
		\forall i\in I: v_i\in U
	\end{equation*}

	Wir zeigen, dass U ein Untervektorraum ist. Das heißt,
	\begin{align*}
	^+    & \mid_{U\times U}: U\times U \to U \subset V;\\
	\cdot & \mid_{K\times U}: K\times U \to U \subset V;
	\end{align*}
	
	d.h. die Addition und Skalarmultiplikation vererben sich auf $ U $.

\paragraph{Zur Skalarmultiplikation}
	Sind $(x_i)_{i\in I}$ mit $\#\{i\in I| x_i \neq 0\}<\infty$ eine Familie in $ K $ und $x\in K$, so gilt für ein geeignetes $n\in \mathbb{N}$:
	\begin{equation*}
		\{i\in I| x_i \neq 0\} = \{i_1, ... , i_n\}
	\end{equation*}

	und damit 
	\begin{equation*}
		\{i\in I| x_ix\neq 0\} =
		\begin{cases}
			\{{i_1,...,i_n\}},& \text{falls }x \neq 0\\
			\emptyset,& \text{falls }x = 0
		\end{cases}
	\end{equation*}

	Also folgt
	\begin{equation*}
		(\sum_{i\in I}v_i x_i) x = (\sum_{j=1}^{n} v_{i_j}x_{i_j})x \Rightarrow \sum_{j=1}^{n} v_{i_j}(x_{i_j}x) = \sum_{i\in I} v_i(x_ix) \in U_i,
	\end{equation*}

	da $\sum_{i\in I} v_i(x_ix)$ Linearkombination (mit der Familie $(x_ix)_{i\in I}$ in K) ist.

\paragraph{Zur Addition}
	Ähnlich (Vereinigung zweier Mengen, ist endlich), siehe Aufgabe.
\paragraph{Bemerkung}
	Um triviale Diskussionen zu vermeiden, setzt man $\sum_{i\in \emptyset} ?:=0$.

% % % % Chapter 1 Section 3 % % % %
\section{Basis und Dimensionen}

\subsection{Definition (Basis)}
	\begin{Definition}[Basis]
		Eine Teilmenge $S\subset V$ oder eine Familie $(v_i| i\in I)$ in $ V $ heißt:
	\begin{itemize}
		\item Erzeugendensystem von $ V $, falls $[S] = V$ bzw. $[(v_i)_{i\in I}] = V$
		\item linear unabhängig, falls $\forall v\in S: v \notin [S\setminus\{{v\}}]$ bzw. $\forall i\in I: v_i \notin [(v_j)_{j\in I\setminus\{{i\}}}]$
	\end{itemize}

	und sonst linear abhängig. Eine Basis ist ein linear unabhängiges Erzeugendensystem.
	\end{Definition}

\paragraph{Bemerkung}
	Man kann jede (Teil-) Menge $S\subset V$ als Familie in V auffassen mit
	\begin{equation*}
		v: S \to V: v\mapsto id_S(v) = v.
	\end{equation*}
	
	Andererseits gilt für eine Familie $(v_i)_{i\in I} $:
	\begin{equation*}
		(v_i)_{i\in I} \text{ linear unabhängig } \Rightarrow \{v_i| i\in I\} \text{ linear unabhängig.}
	\end{equation*}
	
	Die Umkehrung gilt im Allgemeinen nicht.

	Eine Familie (in $ V $) enthält mehr Information als eine Teilmenge von $ V $.
	
\subsection{Beispiel und Definition (Standardbasis)}
	\begin{Definition}[Standardbasis]
		Für $V = K^n$ ist $(e_1, ... , e_n)$,
	\begin{equation*}
		e_i:\{{1, ... ,n\}} =: I\to K: j\mapsto e_i(j)= \delta_{ij}=
		\begin{cases}
			1,& \text{falls } i=j\\
			0,& \text{sonst}
		\end{cases}
	\end{equation*}

	eine Basis -- die Standardbasis des (Standard-)Vektorraumes $K^n$.
	\end{Definition}

\paragraph{Beweis}
	Z.z.: $ (e_i)_{i\in I} $ ist ein linear unabhängiges Erzeugendensystem. Bekannt ist: $ [(e_i)_{i\in I}] = K^n $. Andererseits gilt für jedes $i\in I$ und jede Familie $(x_j| j\in I)$ in $ K $
	\begin{gather*}
		\left(\sum_{j\in I\setminus\{i\}}e_jx_j\right)(i) = \sum_{j\in I\setminus\{i\}}e_j(i)x_j = 0 \neq 1 = e_i(i)\\
		\Rightarrow \sum_{j\in I\setminus\{i\}} e_jx_j \neq e_i,
	\end{gather*}
	
	also gilt:
	\begin{equation*}
		\forall i\in I: e_i \notin [(e_j)_{j\in I\setminus\{i\}}] = \left\{\sum_{j=I\setminus\{i\}} e_jx_j\mid (x_j)_{ j\in I}\right\} \text{ mit } \#\{j\in I| x_j \neq 0\}<\infty
	\end{equation*}
	
\subsection{Lemma}
	\begin{Lemma}
		Eine Familie $(v_i)_{i\in I}$ ist linear unabhängig gdw. für jede Linearkombination
	\begin{equation*}
		0 = \sum_{i\in I} v_ix_i \Rightarrow \forall i\in I: x_i = 0.
	\end{equation*}
	\end{Lemma}

\paragraph{Beweis}
	Wir zeigen zwei Richtungen der Äquivalenz der Negationen: 
	\begin{equation*}
		(v_i)_{i\in I} \text{ linear abhängig } \Leftrightarrow \exists(x_i)_{i\in I} \neq (0)_{i\in I}: \sum_{i\in I} v_ix_i = 0.
	\end{equation*}

	$\Leftarrow$: Wir nehmen an, es gäbe eine nicht-triviale Linearkombination der Null,
	\begin{equation*}
		0 = \sum_{i\in I} v_ix_i, \text{ wobei } \exists j\in I: x_j \neq 0.
	\end{equation*}

	Für $(y_i)_{i\in I}, y_i := - \frac{x_i}{x_j}$ ist dann
	\begin{equation*}
		0 = v_jx_j + \sum_{i\in I\setminus\{j\}} v_ix_i \Rightarrow v_j = -\left(\sum_{i\in I\setminus\{j\}}v_ix_i\right)x_j^{-1} = \sum_{i\in I\setminus\{j\}} v_iy_i \in [(v_i)_{i\in I\setminus\{j\}}],
	\end{equation*}

	insbesondere ist also $(v_i)_{i\in I}$ linear abhängig.

	$\Rightarrow$: siehe Aufgabe.
	
\subsection{Korollar}
	\begin{Korollar}
		Ist $(v_i)_{i\in I}$ Basis von $ V $, so ist jeder Vektor $v\in V$ eindeutig in den $v_i$ darstellbar:
	\begin{equation*}
		\forall v\in V \exists! (x_i)_{i\in I}: v = \sum_{i\in I} v_ix_i
	\end{equation*}
	\end{Korollar}

\paragraph{Beweis}
	Sei $v\in V$ beliebig, dann gilt:
	\begin{equation*}
		V = [(v_i)_{i\in I}] \Rightarrow \exists (x_i)_{i\in I}: v = \sum_{i\in I} v_ix_i
	\end{equation*}

	liefern $(x_i)_{i\in I}$ und $(y_i)_{i\in I}$
	\begin{equation*}
		v = \sum_{i\in I} v_ix_i = \sum_{i\in I}v_iy_i \Rightarrow 0 = \sum_{i\in I} v_i(x_i-y_i)
		\begin{array}{l}
			\Rightarrow \forall i\in I: x_i = y_i\\
			\Rightarrow (x_i)_{i\in I} = (y_i)_{i\in I}
		\end{array}
	\end{equation*}

	Damit ist die Basisdarstellung $v = \sum_{i\in I} v_ix_i$ von $ v $ auch eindeutig.


\subsection{Basislemma}
    \begin{Lemma}[Basislemma]
    	Sei $S\subset V$ lin. unabh. und $E\subset V$ ein Erzeugendensystem mit $S\subset E$. Dann existiert eine Basis $B$ von $V$ mit $S\subset B\subset E$.
    \end{Lemma}

\paragraph{Beweis}
    Wir gehen für den Beweis davon aus, dass $\#E<\infty$. Betrachte alle Teilmengen $X\subset V$ mit $S\subset X\subset E$ und $X$ lin. unabh. Sei $B$ eine solche Menge, die maximal ist, d.h.
    \begin{equation*}
        \forall X\subset E: ((B\subset X\land X\text{ lin. unabh.}) \Rightarrow X= B)
    \end{equation*}
    
    Nach Konstruktion ist $B=\{b_1,...,b_n\}$ lin. unabh. Zu zeigen: $V=[B]$.\\
    Ist $B=E$, so folgt $[B]=[E]=V$.\\
    Ist $B\neq E$, so ist $B\cup \{v\} $ für (jedes) $v\in E\setminus B$ lin. abh., da $B$ maximal (Existenz einer maximalen Menge ist problematisch!) und lin. unabh. ist; also existiert eine nicht-triviale Linearkombination des Nullvektors.
    \begin{equation*}
    \exists x,x_1,...,x_n \in K: o=vx+\sum^n_{i=1}b_ix_i
    \end{equation*}

    Wäre $x=0$, so würde folgen $x_1=...=x_n=0$, da $B$ lin. unabh. ist. 
    Also ist $x\neq 0$ und 
    \begin{equation*}
    	v=-\sum^n_{i=1} b_i\frac{x_i}{x} \in [B].
    \end{equation*}
    
    Da dies für beliebiges $v\in E\setminus B$ gilt, folgt
    \begin{equation*}
    	E\subset [B] \Rightarrow V=[E]\subset [[B]] = [B],
    \end{equation*}
    
    d.h., $ B $ ist Erzeugendensystem und damit eine Basis mit $S\subset B\subset E$.

\paragraph{Bemerkung}
    Ist $\#E = \infty$, so kann man einen analogen Beweis führen, falls man an die Existenz einer maximalen Menge glaubt: Dies garantiert das Zornsche Lemma bzw. Auswahlaxiom.
    Wir werden das Lemma auch im Falle $\#E = \infty$ benutzen!

\paragraph{Beispiel}
    Für $V=K^3=K^I$ mit $I=\{1,2,3\}$ betrachte die Standardbasisvektoren 
    \begin{align*}
        e_i &:I\to K, j\mapsto e_i(j) = \delta_{ij}\text{, und}\\
        f_i &: I\to K, j\mapsto f_i(j):= 1-\delta_{ij};
    \end{align*}

    dann sind $S:= \{e_1,f_1\}$ und $E:= \{e_i,f_i\mid i\in I\}$ lin. unabh. bzw. Erzeugendensystem von $K^3$. Ergänzung von $S$ durch einen Vektor $e_i$ oder $f_i, i = 2,3$ liefert eine Basis $B$ mit $S\subset B\subset E$.
    
    Zum Beispiel: $B=\{e_1,f_1,f_2\}$ eine Basis, da sich jede Funktion $v\in K^3$ aus den Funktionen $e_1,f_1$ und $f_2$ linear kombinieren lässt.
    \begin{gather*}
        v=e_1x_1+f_1y_1 + f_2y_2\Leftrightarrow \left\{
            \begin{array}{l}
                v(2)=y_1\\
                v(3) - v(2) = y_1 + y_2 - y_1 = y_2\\
                v(1) + v(2) - v(3) = x_1 + y_2 - y_2 = x_1
            \end{array}
    	\right.
    \end{gather*}

    Dass $B$ lin. unabh. folgt dann; Wäre $B$ lin. abh., so würde folgen $f_2\in [\{e_1,f_1\}]\Rightarrow [B] \subset [\{e_1,f_1\}] \neq K^3$, was nicht der Fall ist.

\subsection{Basisergänzungssatz}
    \begin{Satz}[Basisergänzungssatz]
    	Jede lin. unabh. Menge $S\subset V$ kann zu einer Basis $B$ von $V$ ergänzt werden: Es existiert eine Basis $B$ von $V$ mit $S\subset B$.
    \end{Satz}

\paragraph{Beweis}
    Sei $E\subset V$ ein Erzeugendensystem von $V$ (z.B. $E=V$). Dann ist $S\cup E$ ein Erzeugendensystem von $V$ mit $S\subset S\cup E$, das Basislemma liefert dann die gesuchte Basis.

\subsection{Bemerkung}
    Strikt genommen haben wir den Basisergänzungssatz (BES) nur unter der Annahme bewiesen, dass $V$ endlich erzeugt sei, d.h. $V$ ein endliches Erz. Syst. $E$ besitzt, $V=[E]$ und $\#E<\infty$.
    
\paragraph{Bemerkung}
    Wir haben für den BES die (in diesem Falle einfachere) Mengenschreibweise (anstelle der Familienschreibweise) verwendet.
    
\paragraph{Bemerkung}
    Ähnlich kann man einen Verkürzungssatz beweisen: Jedes Erzeugendensystem eines Vektorraums $V$ kann zu einer Basis verkürzt werden.

\subsection{Austauschlemma}
    \begin{Lemma}[Austauschlemma]
    	Seien $B,B' \subset V$ Basen von $V$. Dann gilt:
    \begin{equation*}
        \forall b\in B \exists b' \in B': (B\setminus\{b\})\cup\{b'\} \text{ ist Basis}
    \end{equation*}
    \end{Lemma}
    
\paragraph{Beweis}
    Sei $b\in B$ beliebig gewählt und $S:= B\setminus \{b\}$. Da $B$ lin. unabh. ist, gilt $b\notin [S] \Rightarrow \emptyset \neq V\setminus [S] = [B']\setminus [S] \Rightarrow B' \not\subset [S]$, d.h. es existiert $b' \in B'$ mit $b' \notin [S]$. Wir zeigen, dass $B'' := S\cup \{b'\} = (B\setminus\{b\})\cup \{b'\}$ Basis ist. $B''$ ist Erzeugendensystem: Da $b'\in [B]$ existiert $(x_j)_{j\in B}$ mit $$b' = \sum_{j\in B} jx_j $$ mit $x_b \neq 0$, da $b' \notin [S]$.

    Damit ist $b=(b'-\sum_{j\in S} jx_j)\frac{1}{x_b} \in [B''] \Rightarrow V = [B] \subset [B'' \cup \{b\}] \subset [B'']$.
    
    $B''$ ist linear unabhängig. $B''$ ist Erz. Syst. und $S\subset B' = S \cup \{b'\}$ lin unabh., kann also (nach Basislemma) erg"anzt werden zu einer Basis $\tilde{B}$ mit $S\subset \tilde{B}\subset B''$.
    Da $[S] \neq V$ gilt $\tilde{B} \neq S$ und damit $\tilde{B} = B''$ Basis, insbesondere linear unabhängig.
    
\paragraph{Bemerkung}
    Hier haben wir die Familienschreibweise (mit $B$ bzw. $S$ als Indexmenge) verwendet, um Linearkombinationen darzustellen.
    
\subsection{Basissatz}
	\begin{Satz}[Basissatz]
	Sei $V$ ein endlich erzeugter $K$-VR, $V=[E]$ mit $\#E < \infty$. Dann gilt:
	\begin{enumerate}[(i)]
		\item $V$ besitzt eine endliche Basis $B$ mit $n:= \#B \leq \#E$.
		\item Ist $B'\subset V$ eine Basis von $V$, so ist $\#B' = \#B = n$.
	\end{enumerate}
	\end{Satz}
    
\paragraph{Beweis}
    \begin{enumerate}[(i)]
        \item  Dies folgt direkt aus dem Basislemma (mit $S=\emptyset$).
        \item Seien $B,B'$ Basen von V, $B = (b_1,...,b_n)$.\\
        Annahme: $\#B' < n, B' = (b'_1,...,b'_k)$ mit $k < n$. Wiederholte Anwendung des Austauschlemmas auf die Basen $B$ und $B'$ liefert nach (spätestens) $k+1\leq n$ Schritten einen Widerspruch zur linearen Unabhängigkeit der neuen Basis $B''$, da Vektoren $b'_i$ doppelt vorkommen müssen.\\
        Annahme: $\#B' > n, B' = (b'_1,...,b'_n,b'_{n+1})$: Das gleiche Argument mit vertauschten Rollen der Basen führt wieder zum Widerspruch.
     \end{enumerate}

\subsection{Definition (Dimension)}
    \begin{Definition}[Dimension]
    	Sei $V$ ein $K$-VR, die Dimension von $ V $ ist dann:
    \begin{itemize}
        \item $\dim V:= \#B$, falls $ V $ endlich erzeugt und $B$ eine Basis von $V$ ist;
        \item $\dim V:= \infty$, falls $V$ nicht endlich erzeugt ist.
    \end{itemize}
    \end{Definition}
    
\paragraph{Bemerkung}
    Nach dem Basissatz hängt $\dim V = \#B$ (falls $V$ endlich erz.) nicht von der Basis $B$ ab, d.h. $\dim V$ ist wohldefiniert.
    
\paragraph{Beispiel}
    $\dim K^n = \#\{e_1,...,e_n\} = n$ (Standardbasis).

\subsection{Korollar (Dimension und Teilmengen)}
	\begin{Korollar}[Dimension und Teilmengen]
		Sei $ V $ ein $ K $-VR mit $\dim V =: n\in \mathbb{N}$. Dann gilt:
    \begin{enumerate}[(i)]
    	\item Ist $S \subset V$ linear unabhängig, so ist $\# S \leq n$ und $\# S = n$ genau dann, wenn $ S $ Basis ist.
    	\item Ist $E \subset V$ Erzeugendensystem, so ist $\#E \geq n$, bzw. $\#E = n$ genau dann, wenn $ E $ eine Basis ist.
    \end{enumerate}
	\end{Korollar}
    
\paragraph{Bemerkung}
	Insbesondere: Ist $U\subset V$ UVR mit $\dim U=\dim V < \infty$, so gilt $ U=V $.
   
\paragraph{Beweis}
    \begin{enumerate}[(i)]
    	\item Ist $ S $ linear unabhängig, so existiert (nach BES) eine Basis $ B $ von $ V $ mit 
			 \begin{gather*}
			    S\subset B\Leftrightarrow \left\{
				    \begin{array}{l}
					    \#S \leq \#B\\
						\#S = \#B \Leftrightarrow S = B
					\end{array}
			    \right.
		    \end{gather*}
		    \item Analog (mit Basislemma), siehe Aufgabe 23.
	 \end{enumerate} 

%VO_29.10.15
\section{Homomorphismen}
\paragraph{Definition:}
	Sind $ V $ und $ W $ $ K $-VR, so heißt eine Abbildung $f: V \rightarrow W$ ($ K $-)linear oder ein (Vektorraum-)Homomorphismus $f\in \hom(V,W)$, falls gilt:

\begin{enumerate}[(i)]
	\item $\forall v,w \in V: f(v+w) = f(v) + f(w)$;
	\item $\forall v\in V\ \forall x\in K: f(vx) = f(v)x$
\end{enumerate}

    das heißt, f ist verträglich mit den Vektorraumoperationen in V und W.
    
\paragraph{Bemerkung:}
	Damit die Verträglichkeit mit der Skalarmultiplikation sinnvoll ist, müssen $ V $ und $ W $ Vektorräume über demselben Körper $ K $ sein.

\paragraph{Bemerkung:}
	Für $f\in \hom(V,W)$ gilt stets $f(0_V) = f(0_V0_K) = f(0_V)0_K = 0_W$.
  
  Ebenso erklärt man zum Beispiel Gruppenhomomorphismen und Körperhomomorphismen. Sind etwa $(G,\circ)$ und $(H,*)$ Gruppen, so ist eine Abbildung $f: G \to H$ ein Gruppenhomomorphismus, falls $\forall g,h \in G: f(g\circ h) = f(g) * f(h)$
  
\paragraph{Beispiel:}
	Ist $f\in \hom(V,W)$ ein Vektorraumhomomorphismus so ist $ f $ nach (i) Gruppenhomomorphismus von $ (V,+) $ in $ (W,+) $.
  
\paragraph{Beispiel:}
	Sei $ V $ ein $ K $-VR und $y\in K$ fest, dann ist die Streckung um $y: \eta_y:V\to V: v\mapsto \eta_y(v) := vy$ ein Homomorphismus von $ V $ in sich, $\eta_y\in \hom(V,V)$. Eine Streckung nennt man auch Homothetie.
  	
\paragraph{Beispiel:}
	Sei $V = \mathbb{C} = \{z = x+iy\mid x,y\in \mathbb{R}\}$, dann ist die komplexe Konjugation $\mathbb{C}\ni z = x+iy \mapsto x-iy =: \bar{z} \in \mathbb{C}$ kein Homomorphismus von $\mathbb{C}$ in sich, wenn man $\mathbb{C}$ als $\mathbb{C}$-VR auffasst. Hingegen ist sie ein Homomorphismus von $\mathbb{C}$ in sich, wenn man $\mathbb{C}$ als $ \mathbb{R} $-VR auffasst.
\paragraph{Lemma:}
	$f:V\to W$ ist genau dann ein Homomorphismus, wenn für jede beliebige Linearkombination gilt: $f(\sum_{i\in I}v_ix_i) = \sum_{i\in I}f(v_i)x_i$

\paragraph{Beweis:}
	Eine Richtung ist trivial, die andere mit vollständiger Induktion zu zeigen.

\paragraph{Fortsetzungssatz:} 
	Seien $ V $ und $ W $ $K$-VR, $(b_i)_{i\in I}$ eine Basis von $ V $ und $(c_i)_{i\in I}$ eine Familie in $ W $.
	Dann gilt: $\exists!f\in \hom(V,W), \forall i\in I: f(b_i) = c_i$.
    
\paragraph{Bemerkung:}
        Anders ausgedrückt: ist $B\subset V$ eine Basis von $ V $, so kann jede Abbildung $f: B\to C\subset W$ eindeutig zu einem Homomorphismus $f: V\to W$ fortgesetzt werden.
    
\paragraph{Beweis:}
	Wir beweisen die Existenz und die Eindeutigkeit getrennt. 
	\begin{enumerate}
		\item Eindeutigkeit: Sei $f\in \hom(V,W)$ so, dass $\forall i\in I: f(b_i)=c_i$. Sei $v\in V$ beliebig. Da $ B $ Erzeugendensystem ist, lässt sich $ v $ als Linearkombination in $(b_i)_{i\in I}$ mit geeigneten Koeffizienten $(x_i)_{i\in I}$ in $ K $ darstellen.
			\begin{gather*}
    				v=\sum_{i\in I}b_ix_i \Rightarrow f(v) = \sum_{i\in I} f(b_i)x_i = \sum_{i\in I}c_ix_i
    			\end{gather*}
    
                        Damit ist $ f(v) $ eindeutig durch $ v $ und die $c_i = f(b_i)x_i$ bestimmt.
    
    		\item Existenz: Da $(b_i)_{i\in I}$ auch linear unabhängig ist, ist jedes $v\in V$ eindeutig als Linearkombination in $(b_i)_{i\in I}$ dargestellt, damit ist durch $f:V\to W: v=\sum_{i\in I}b_ix_i \mapsto f(v):=\sum_{i\in I}c_iv_i$ eine Abbildung wohldefiniert.
    
                        Weiters ist $f\in\hom(V,W)$ wegen
                        \begin{gather*}
                                f(v+w) =\sum_{i\in I}c_i(x_i+y_i)=\sum_{i\in I}c_ix_i+ \sum_{i\in I}c_iy_i =  f(v) + f(w) \text{ für alle }\left\{
                                        \begin{array}{l}
                                                v=\sum_{i\in I}b_ix_i \in V\\
                                                w=\sum_{i\in I}b_iy_i \in V
                                        \end{array}
                                \right.
                        \end{gather*}
    
                        und
                        \begin{gather*}
                            f(vx) =\sum_{i\in I}c_i(x_ix)\Rightarrow\sum_{i\in I}(c_ix_i)x = (\sum_{i\in I}c_ix_i)x= f(v)x \text{ für }  x\in K\text{ und }v= \sum_{i\in I}b_ix_i \in V.
                        \end{gather*}
                        Damit ist die Linearität von $ f $ gezeigt.
        \end{enumerate}
    
\paragraph{Beispiel und Definition:}
	Der Dualraum $V^\ast := \hom(V,K)$ eines $K$-VRs $V$ ist ein $ K $-VR $(\subset K^V)$. Ist $\dim V=:n<\infty$ so ist $\dim V^\ast=n$.
	Ist $B=(b_i, ... ,b_n)$ eine Basis von $ V (\dim V < \infty)$, so definieren wir für $ i = \{1, ... ,n\} $ die Linearform (nach Fortsetzungssatz):
	\begin{equation*}
		b_i^\ast\in V^*:V\to K, \forall j\in \{1,...,n\}:b_i^*(b_j)=\delta_{ij}
	\end{equation*} die zu $ B $ duale Basis $ B^* $ von $V^\ast$.

%ADB_VO_03.11.15
\paragraph{Beweis:} $ V^* $ ist $ K $-VR. Wir zeigen $ V^*\subset K^V $ ist UVR.
        \begin{itemize}
                \item $ 0: V\to K $ ist linear, d.h. $ 0 \in V^* \Rightarrow V^* \neq \emptyset $
                \item Seien $ f,g \in V^* $ und $ x\in K $; dann gilt
			\begin{align*}
				\forall v,w\in V: (fx+g)(v+w) &= f(v+w)x+g(v+w)\\
                                                              &= (f(v)+ f(w))x+(g(v)+g(w))\\
                                                              &= (f(v)x+g(v))+(f(w)x+g(w))\\
                                                              &= (fx+g)(v)+(fx+g)(w)
			\intertext{genauso:}
                                \forall v\in V, y\in K: (fx+g)(vy) &= f(vy)x+g(vy)\\
                                                                   &= f(v)yx + g(v)y\\ 
                                                                   &= (f(v)x +g)y = ((fx+g)y)(v)
                        \end{align*}
                        Damit gilt: $ fx+g\in \hom (V,K) = V^* $
        \end{itemize}
	
	Da $ f,g\in V^* $ und $ x\in K $ beliebig waren, zeigt das UR-Kriterium, dass $ V^*\subset K^V $ ein UVR ist und damit selbst $ K $-VR ist.
	
\paragraph{Beweis:} $B^*$ ist Basis. Wir zeigen $B^*$ ist linear unabhängig und Erzeugendensystem.
	\begin{itemize}
            \item $ B^* $ ist linear unabhängig: Seien $ x_1,...,x_n $ so, dass
                    \begin{equation*}
                    0 = \sum_{i=1}^{n}b_i^*x_i \Rightarrow \forall j=\{1,...,n\}:0=(\sum_{i=1}^{n}b_i^*x_i)(b_j) = \sum_{i=1}^{n}b_i^*(b_j)x_i = \sum_{i=1}^{n}\delta_{ij}x_i = x_j.
                    \end{equation*}
            Also $ x_1 = ... = x_n = 0 $ und damit ist $ B^* $ linear unabhängig.
            \item $ B^* $ ist Erzeugendensystem: Sei $ f\in V^* $ beliebig, dann gilt:
            \begin{equation*}
                    \forall j = \{1,...,n\}:f(b_i) = \sum_{i=1}^{n}b_i^*(b_j)f(b_i) = (\sum_{i=1}^{n}b_i^*f(b_i))b_j \Rightarrow f = \sum_{i=1}^{n}b_i^*f(b_i)\in [B^*].
            \end{equation*}
            
            Da $ f\in V^* $ beliebig war, ist also $ V^* = [B^*]$.
	\end{itemize}
	
	Damit ist $ B^* = \{b_1^*,...,b_n^*\}$ eine Basis von $ V^* $ -- insbesondere also $ \dim V^* = n = \dim V = \dim K\cdot \dim V $.
%ADE_VO_03.11.15

\paragraph{Bemerkung:}
	Ist $\dim V = \infty$ und $B=(b_i)_{i\in I}$ eine Basis von $V$, so liefert $B^\ast=(b_i^\ast)_{i\in I}$ mit $\forall j\in I:b_i^\ast(b_j)=\delta_{ij}$ eine lineare unabhängige Familie. Diese ist jedoch kein Erzeugendensystem von $V^\ast: f\in\hom(V,K)=V^\ast$ mit $\forall j\in I:f(b_j)=1$ lässt sich nicht in $B^\ast$ linear kombinieren. Wäre $f=\sum_{i\in I}b_i^\ast x_i$, so gälte $\forall j\in I: x_j =\sum_{i\in I}b_i^\ast(b_j)x_j= \sum_{i\in I} \delta_{ij}x_j = f(b_j) = 1$.

	Das heißt, $(x_i)_{i\in I}$ wäre eine Familie in $ K $ mit $\#\{i\in I\mid x_i\neq 0\}=\infty$.

%VO_03.11.15
\paragraph{Satz:}
	$ \hom (V,W) $ ist ein VR. Die Dimension der Homomorphismen $\dim\hom (V,W) = m\cdot n$, falls $m:=\dim W<\infty, n:=\dim V< \infty$.
	
\paragraph{Beweis:}
	Addition und Skalarmultiplikation in $\hom (V,W)$ werde (wie für $K$-wertige Abbildungen oder in $V^*$) punktweise definiert:
	\begin{itemize}
		\item für $f,g \in \hom (V,W)$ setzt man $(f+g)(v) := f(v) + g(v)$ für alle $v\in V$,
		\item für $f\in \hom (V,W)$ und $x\in K$ setzt man $(fx)(v) := f(v)x$ für alle $v\in V$.
	\end{itemize}
	Die so definierten Abbildungen $f+g,fx: V\to W$ sind linear, $f+g, fx\in \hom (V,W)$, aufgrund der VR-Eigenschaften von $V$.
	
	Damit zeigt man: $\hom (V,W)$ ist $K$-VR (siehe Aufgabe 27).
	
	Seien nun $\dim V = n < \infty$ und $\dim W = m < \infty$.
	
	Wir wählen (nach BES) Basen $B = (b_1,...,b_n)$ von $V$ und $C=(c_1,...,c_m)$ von $W$ und definieren
		\begin{equation*}
			f_{ij}\in \hom (V,W) mit f_{ij}:= c_i\cdot b_j^* \text{ für } 
				\begin{cases}
					i\in I := \{1,...,m\}\\
					j\in J := \{1,...,n\}
				\end{cases}
		\end{equation*}
	Behauptung: $F=(f_{ij})_{I,J}$ ist Basis von $\hom (V,W)$.
	
	Da $(c_i)_{i\in I}$ linear unabhängig in $W$ ist, gilt für jede Famlilie $(x_{ij})_{I,J}$ in $K$:
		\begin{gather*}
			0 = \sum_{I,J} f_{ij}x_{ij} \Rightarrow \forall k \in J: 0 = \sum (f_{ij}x_{ij})(b_k)\\
			= \sum c_i b_j^* (b_k) x_{ij} = \sum_{i\in I} c_ix_{ik} \Rightarrow \forall k\in J\forall i\in I:x_{ik} = 0
		\end{gather*}
	Also ist $F$ linear unabhängig.
	
	Da $(c_i)_{i\in I}$ Erzeugendensystem von $W$ ist, existiert zu jedem (fest gegebenen) $f\in\hom (V,W)$ eine Familie $(x_{ij})_{IJ}$ in $K$, sodass
		\begin{gather*}
		\forall k\in J: f(b_k) = \sum_{i\in I} c_i x_{ik} \text{ (da $(c_i)_{i\in I}$ Erzeugendensystem)}\\
		= \sum_{I,J}c_ib_j^*(b_k)x_{ij} = \left(\sum_{IJ} f_{ij}x_{ij}\right)(b_k)\\
		\text{also (Fortsetzungssatz): } f=\sum_{I,J}f_{ij}x_{ij} \in [F].
		\end{gather*}
	
	Da $f\in\hom (V,W)$ beliebig war, gilt also $\hom (V,W) = [F]$. Damit ist $F$ Basis von $\hom (V,W)$ und $\dim\hom (V,W) = \# F = m\cdot n$.
	
\paragraph{Lemma und Definition}
	Sei $f\in \hom (V,W)$. Dann sind Bild und Kern von f:
		\begin{equation*}
			f(V) = \{f(v)\in W\mid v\in V \}\subset W \text{ bzw. } \ker (f) := \{v\in V\mid f(v) = 0 \} \subset V
		\end{equation*}
	
	UVR von $W$ bzw. $V$. Ihre Dimensionen heißen Rang und Defekt von $f$:
		\begin{equation*}
			\rg f := \dim f(V) \text{ bzw. } \dfkt f := \dim \ker f
		\end{equation*}

\paragraph{Bemerkung: }
	Da $f(0)=0$ für  $f\in \hom (V,W)$ gilt $\{o_V \}\in \ker f$ und $\{o_W \}\in f(V)$.

\paragraph{Beweis: }
	Zu zeigen: Das Bild $f(V)\subset W$ und $\ker f\subset V$ sind UVR. Nach Bemerkung gilt $f(V)\neq \emptyset$ und $\ker f \neq \emptyset$ -- wir verwenden dann das UR-Kriterium.
	
	Das Bild $f(V)$ ist UVR: $f(V) \neq \emptyset$. Es bleibt zu zeigen:
		\begin{equation*}
			\forall w_1,w_2\in f(V), \forall x\in K: w_1x+w_2 \in f(V).
		\end{equation*}
	
	Seien also $w_1 = f(v_1), w_2 = f(v_2) \in f(V)$ und $x\in K$; dann gilt:
		\begin{equation*}
			w_1x+w_2 = f(v_1)x+f(v_2) = f(v_1x+v_2)\in f(V)
		\end{equation*}
		
	Der Kern $\ker f$ ist UVR: $\ker f\neq \emptyset$; seien $v_1,v_2\in \ker f$ und $x\in K$, dann gilt:
		\begin{equation*}
			f(v_1x+v_2) = f(v_1)x+f(v_2) = 0\cdot x + 0 = 0 \Rightarrow v_1x+v_2\in \ker f
		\end{equation*}

\paragraph{Bemerkung: }
	Allgemeiner kann man für $f\in \hom (V,W)$ zeigen:
		\begin{enumerate}
			\item ist $U\subset V$ UVR, so ist $f(U)\subset W$ UVR
			\item ist $U\subset V$ UVR, so ist $f^{-1}(U) = \{v\in V\mid f(v) \in U \}\subset V$ ein UVR
		\end{enumerate}

\paragraph{Bemerkung: }
	Die Funktion $f\in \hom (V,W)$ ist genau dann injektiv, wenn $\ker f = \{o\}$. Nämlich:
		\begin{itemize}
			\item ist $f$ injektiv und $v\in \ker f$, so gilt $f(v) = 0 = f(0) \Rightarrow v=0$
			\item ist $\ker f = \{ o \}$ und sind $v,w \in V$ mit $f(v) = f(w)$, so folgt\\
				$0=f(v)-f(w) = f(v-w) \Rightarrow v-w\in \ker f = \{o\} \Rightarrow v = w$
		\end{itemize}

\paragraph{Bemerkung: }
	Eine lineare Abbildung $ f\in \hom (V,W) $ ist genau dann
		\begin{enumerate}[(i)]
			\item injektiv, wenn $ \forall S\subset V: S$ lin. unabh. $ \Rightarrow f(S) $ lin. unabh.
			\item surjektiv, wenn $ \forall E \subset V:E $ Erz. Syst. $ \Rightarrow f(E)$ Erz. Syst.
			\item bijektiv, wenn $ \forall B\subset V: B$ Basis $ \Rightarrow f(B)$ Basis
		\end{enumerate}

	Ist $ f\in \hom (V,W) $ bijektiv, so ist $ f^{-1}\in \hom (W,V) $.

\paragraph{Rangsatz: }
	Sei $ f\in \hom (V,W) $. Ist $ \dim V = n < \infty $,  so gilt $\rg f + \dfkt f = \dim V$.  Ist $ \dim V = \infty $, so gilt $ \rg f = \infty $ oder $ \dfkt f = \infty $.

%VO_05.11.15
\subparagraph{Beweis: }
	Wir nehmen an, dass $ \dfkt f = k \neq \infty $.
	Sei $ (b_1,...,b_k) $ eine Basis von $ \ker f $;
	nach BES ergänzen wir zu einer Basis $ (b_j)_{j\in J} $ von $ V $ (bemerke: $ \{1,...,k\}\subset J $).
	Wir sehen $ I:= J\setminus \{1,...,k\} $ und $ \forall i\in I: c_i := f(b_i) $.
	
	Behauptung: $(c_i)_{c\in I}$ ist eine Basis von $f(V)$.
	
	-- Lineare Unabhängigkeit: gilt für eine Linearkombination in $(c_i)_{i\in I}$:
	\begin{equation*}
		0=\sum_{i\in I}c_ix_i = \sum_{i\in I}f(b_i)x_i = f(\sum_{i\in I}b_ix_i)
	\end{equation*}
	so folgt
	
	\begin{gather*}
		\sum_{i\in I}b_ix_i \in \ker f\\
		\Rightarrow \exists y_1,...,y_n\in K:\sum_{i\in I}b_ix_i=\sum_{j=1}^{k}b_jy_j\\
		\Rightarrow 0 = \sum_{i\in I}b_ix_i - \sum_{j=1}^{k}b_jy_j\\
		\Rightarrow
		\begin{cases}
			\forall j = 1, ... ,k:y_j=0\\
			\forall i\in I: x_i = 0
		\end{cases}
		\text{da $(b_j)_{j\in J}$ linear unabhängig ist}
	\end{gather*}
			
	Insbesondere gilt also $\forall i\in I: x_i = 0$ damit folgt die Lineare Unabhängigkeit nach Lemma.
	
	-- Erzeugendensystem:
	
	Sei $w\in f(V)$, also Existiert $v\in V$ mit $w = f(v)$. Da $(b_j)_{j\in J}$ Basis von $V$ ist, existiert eine Familie $(x_j)_{j\in J}$ in $K$ so, dass 
	\begin{equation*}
		v = \sum_{j\in J} b_jx_j
	\end{equation*}
	
	Dann gilt
	\begin{gather*}
		w = f(v) = f(\sum_{j\in J} b_jx_j) = \sum_{j\in J}f(b_jx_j)\\
		J=I \cup\{{1,...,k\}} \Rightarrow \sum_{j=1}^{k}f(b_j)^{=0}x_j + \sum_{i\in I}f(b_i)^{=c_i}x_i = \sum_{i\in I}c_ix_i\in[(c_i)_{i\in I}].
	\end{gather*}
			
	Da $(c_i)_{i\in I}$ also Basis von $f(V)$ ist folgt:
			
	\begin{enumerate}[1.{ Fall}]
		\item $(\dim V = n<\infty)$ dann ist $\# J = n$ und $\# I = \# J-k$, also $\rg f = n-k = \dim V - \dfkt f.$\\
		\item $(\dim V = \infty)$, dann ist $\# J = \infty $ und damit auch $\#I =\#(J\{{1,...,k\}})=\infty $, also $\rg f= \infty$.
	\end{enumerate}
	
\paragraph{Bemerkung: }
	Die Annahme $\dfkt f = k<\infty$, im Beweis ist keine Einschränkung:
	\begin{enumerate}
		\item ist $\dim V < \infty$, so folgt $\dfkt f<\infty$, da $\ker f\subset V$ Untervektorraum ist;			
		\item $\dim V = \infty$, so ist man mit dem Beweis fertig, falls $\dfkt f = \infty$.
	\end{enumerate}
			
\paragraph{Korollar: } 
	Sei $f\in \hom(V,W)$ und $\dim W = \dim V = n<\infty$.
	Dann gilt: Ist $f$ injektiv oder surjektiv, so ist $f$ bijektiv.
	
\paragraph{Beweis: } 
	Der Rangsatz liefert:
	\begin{enumerate}
		\item Wenn $ f $ injektiv ist, dann ist $\ker f = \{{0\}}$, also ist $ \dfkt f = 0 \Rightarrow \rg f = \dim V-0 = \dim V \Rightarrow f(V) = W \Leftrightarrow f$ surjektiv
		\item $f(V) = W \Rightarrow \rg f = \dim W = \dim V \Rightarrow \dfkt f= \dim V - \rg f=0 \Rightarrow \ker f = \{{0}\}$
	\end{enumerate}
	
\paragraph{Beispiel: }
	Der Shiftoperator für Folgen $(x_i)_{i\in \mathbb{N}}$ in $K$, $s: K^{\mathbb{N}} \to K^{\mathbb{N}}: (x_i)_{i\in \mathbb{N}} \mapsto (y_i)_{i\in \mathbb{N}}$ wobei
	
	\begin{equation*}
		y_i :=
		\begin{cases}
			0 &\text{ für } i = 0\\
			x_{i-1} &\text{ für } i \neq 0
		\end{cases}
	\end{equation*}
			
	ist ein injektiver Homomorphismus, $s\in \hom(K^\mathbb{N},K^\mathbb{N})$ von $K^\mathbb{N}$ in sich (damit gilt $\dim $ Definitionsbereich $= \dim K^\mathbb{N}= \dim$ Wertebereich). Aber $s$ ist nicht surjektiv, also auch nicht bijektiv.
	
\paragraph{Übrigens: }
	Damit folgt $\dim K^\mathbb{N} =\infty$ (sonst hätte man einen Widerspruch zum Korollar).
		
\paragraph{Definition: }
	Sei $f\in \hom(V,W)$ ein Homomorphismus, dann heißt $f$:
	\begin{itemize}
		\item Endomorphismus, $f\in \operatorname{End} (V)$, falls W = V;
		\item Isomorphismus, $f\in \operatorname{Iso}(V,W)$, falls f bijektiv ist;
		\item Automorphismus, $f\in \operatorname{Aut}(V)$, falls W=V und f bijektiv ist.
	\end{itemize}
	
	Zwei K-VR V und W heißen isomorph, W $\cong$ V, falls $\operatorname{Iso}(V,W) \neq \emptyset$

\paragraph{Bemerkung: }
	Ein Isomorphismus $f\in \operatorname{Iso}(V,W)$ bildet jede Basis $B$ von $V$ auf eine Basis $C = f(B)$ von $W$ ab.
	
	Andererseits: Bildet eine lineare Abbildung $f\in \hom(V,W)$, eine Basis $B$ von $V$ auf eine Basis $C = f(B)$ von $W$ ab, so ist $f$ ein Isomorphismus.
	
	Nämlich: Ist $B$ Basis von $V$ und $ C = f(B)$ Erzeugendensystem, so ist $f$ surjektiv, da $f(V) = f ([B]) = [f(B)] = [C] = W$;
	ist $C =f(B)$ linear unabhängig, so ist $f$ injektiv, denn für
			
	\begin{gather*}
		v = \sum_{b\in B} bx_b \in \ker f \Rightarrow 0 = f(v) = f(\sum_{b\in B}bx_b) = \sum_{b\in B}f(b)x_b\\
		\Rightarrow \forall b \in B: x_b = 0 \Rightarrow v = 0, \text{ d.h., } \ker f=\{{0}\}.
	\end{gather*}
			
\paragraph{Isomorphielemma: }
	Seien $V$ und $W$ $ K $-VR mit $\dim V, \dim W < \infty$.
	Dann gilt: $V \cong W \Leftrightarrow \dim V = \dim W$.
	
\paragraph{Beweis: }
	Folgt aus obiger Bemerkung. Ausführlich:
		
	$\Rightarrow$:
	
	Annahme: $V \cong W$; sei $f\in \operatorname{Iso}(V,W)(\neq 0)$.
	Wähle eine Basis $B = (b_1, ... b_n)$ von $V$ (BES); da $f$ bijektiv, ist dann:
	\begin{equation*}
		C = f(B) = (f(b_i), ... , f(b_n))
	\end{equation*}
	
	eine Basis von W, damit ist $\dim W = n = \dim V$.
	
	$\Leftarrow$:
	
	Sei $\dim W = \dim V = n$;
	wähle Basen $B = (b_1, ... ,b_n)$ von $V$ und $C = (c_1, ... ,c_n)$ von $W$ (BES und Basissatz) und definiere $f\in \hom(V,W)$ durch (Fortsetzungssatz):
	\begin{equation*}
		\forall i = 1, ... ,n : f(b_i) = c_i
	\end{equation*}

	Da $f$ eine Basis auf eine Basis abbildet ist $f\in \operatorname{Iso}(V,W)$.
	Damit folgt also $Iso(V,W) \neq \emptyset \Rightarrow V \cong W$.
			
\paragraph{Beispiel: }
	Ist $V$ $K$-VR mit $\dim V < \infty$, so ist $V^\ast \cong V$. (Achtung: Es gibt aber viele Isomorphismen, keiner ist besonders d.h., \glqq kanonisch\grqq )
	
\paragraph{Bemerkung: }
	Ist $f\in \operatorname{Iso}(V,W)$, so ist $f^{-1}\in \operatorname{Iso}(W,V)$, denn
	\begin{gather*}
		(f\circ f^{-1})(\sum_{i\in I}v_ix_i) = f(\sum_{i\in I}f^{-1}(v_i)x_i) = \sum_{i\in I}(f\circ f^{-1})^{(= id)}(v_i)x_i\\
		\Rightarrow f^{-1}(\sum_{i\in I}v_ix_i) = \sum_{i\in I}f^{-1}(v_i)x_i
	\end{gather*}

%VO_10.11.15

%VO_10.11.15
\section{Summen, Produkte und Quotienten}
\paragraph{Definition: }
	Die Summe einer Familie $ (U_i)_{i\in I} $ von UVR $ U\subset V $ eines $ K $-VR ist die Menge
		\begin{equation*}
		\sum_{i\in I} U_i := \{\sum_{i \in I}u_i\mid \forall i\in I: u_i\in U_i \land \# \{i\in I\mid u_i \neq 0\}<\infty\}.
		\end{equation*}
		
\paragraph{Bemerkung: }
	Offenbar ist $ \sum_{i\in I} U_i\subset V $ UVR mit
	\[  \bigcup_{i\in I}U_i \subset \sum_{i\in I} U_i \Rightarrow [\bigcup_{i\in I}U_i]\subset \sum_{i\in I} U_i; \]
	andererseits gilt:
	\[ \sum_{i\in I}U_i \subset \{\sum_{j\in J}v_jx_j\mid \forall j\in Jv_j\in \bigcup_{i\in I}U_i \land \#\{j\in J\mid x_j\neq 0\}<\infty\}\subset [\bigcup_{i\in I}U_i]. \]
	
	Damit ist die Summe einer Familie $ (U_i)_{i\in I} $ gerade die lineare Hülle ihrer Vereinigung $ \bigcup_{i\in I}U_i $,
	\[ \sum_{i\in I}U_i= [\bigcup_{i\in I}U_i]. \]
		
\paragraph{Beispiel: }
	Sei $ V=\mathbb{R}^\mathbb{N} $ der Raum der reellen Folgen. Für $ n\in \mathbb{N} $ setze
		\begin{equation*}
		U_n := \{v\in \mathbb{R}^\mathbb{N}\mid \forall j\in \mathbb{N}: j>n\Rightarrow v_j = 0 \} \subset \mathbb{R}^\mathbb{N};
		\end{equation*}
		
	dann gilt $ \forall n\in \mathbb{N}: U_n\subset U_{n+1} $, und damit auch
		\begin{gather*}
		\sum_{i\leq n} U_i = U_n = \bigcup_{i\leq n}U_i \text{, aber}\\
		\sum_{i\in \mathbb{N}}U_i = \bigcup_{i\in \mathbb{N}}U_i \neq V.
		\end{gather*}
		
	Nun setze für $ i\in \{0,1\} $
		\begin{equation*}
		\tilde{U}_i := \{v\in \mathbb{R}^\mathbb{N}\mid \forall j\in \mathbb{N}: j=i\operatorname{mod} 2\Rightarrow v_j = 0\}
		\end{equation*}
		
	dann ist 
		\begin{equation*}
		\bigcup_{i\in \{0,1\}}\tilde{U}_i \neq \sum_{i\in \{0,1\}}\tilde{U}_i = V.
		\end{equation*}
\paragraph{Dimensionssatz: }
	Sind $ U_i \subset V $ UVR mit $ \dim U_i < \infty $ für $ i\in \{1,2\} $, so ist
		\begin{equation*}
		\dim (U_1+U_2) + \dim (U_1\cap U_2) = \dim U_1 + \dim U_2.
		\end{equation*}
		
	Ist $ \dim U_1 = \infty$ oder $ \dim U_2=\infty $, so ist auch $ \dim (U_1+U_2)=\infty $.
\paragraph{Beweis: }
	Seien
		\begin{itemize}
		\item $ B_0 \subset U_1\cap U_2 $ eine Basis von $ U_0 := U_1\cap U_2 $;
		\item $ S_i \subset U_i $ lin. unabh., sodass $ B_i = B_0 \cup S_i $ Basen von $ U_i $ sind ($ i = 1,2; $ BES).
		\end{itemize}
	
	Offenbar gilt dann, da $ B_i = B_0\cup S_i $ lin. unabh. sind,
		\begin{equation*}
		B_0\cap S_1 = \emptyset \text{ und } B_0\cap S_2 = \emptyset
		\end{equation*}
	und 
		\begin{equation*}
		S_1\cap S_2 \subset U_1\cap U_2 = [B_0] \Rightarrow S_1\cap S_2 = \emptyset.
		\end{equation*}
		
	Wir zeigen, dass $ B:= B_0\cup S_1\cup S_2 $ Basis von $ U_1 + U_2 =: U $ ist.
	
	$ B\subset U $ ist Erz. Syst. nach Konstruktion:
		\begin{gather*}
		\forall i\in \{1,2\} : U_i=[B_i]\subset [B]\\
		\Rightarrow U_1+U_2 = [U_1\cup U_2]\subset [B]
		\end{gather*}
	
	$ B $ ist linear unabhängig: Gegeben sei eine Linearkombination von $ 0\in U $,
		\begin{gather*}
		0 = \sum_{b\in B}bx_b = \sum_{b\in B_0}bx_b + \sum_{b\in S_1}bx_b + \sum_{b\in S_2}bx_b =: b_0 + s_1+ s_2\\
		\text{mit } b_0\in [B_0] = U_0 \text{ und } s_i\in [S_i] \text{ für } i= 1,2;
		\end{gather*}
		
	dann gilt etwa, $ B_1 = B_0 \cup S_1 $ lin. unabh.,
		\begin{equation*}
		b_0+s_1 = -s_2 \in U_1\cap [S_2]\subset U_0 \Rightarrow s_1 = 0
		\end{equation*}
		
	und damit, da $ B_2 = B_0 \cup S_2 $ lin. unabh. ist,
		\begin{equation*}
		0 = b_0 + s_1 + s_2 \Rightarrow b_0=s_2 = 0.
		\end{equation*}
	
	Mit der linearen Unabhängigkeit von $ B_0, S_1 $ und $ S_2 $ folgt dann
		\begin{equation*}
		0 = \sum_{b\in B_0}bx_b = \sum_{b\in S_1}bx_b = \sum_{b\in S_2}bx_b \Rightarrow \forall b\in B: x_b = 0.
		\end{equation*}
	
	Mit
		\begin{gather*}
		\#B + \#B_0 = (\#B_0 + \#S_1 + \#S_2) + \#B_0\\
		= (\#B_0+\#S_1)+(\#B_0 + \#S_2) = \#B_1+\#B_2
		\end{gather*}
	
	folgt dann die Behauptung.

\paragraph{Bemerkung: }
	Im Beweis haben wir benutzt: 
	
	Ist z.B. $ B_1 = B_0\cup S_1 $ lin. unabh., und $ b_0\in [B_0] $ und $ s_1\in [S_1] $ mit $ b_0 + s_1 = 0 $, so folgt $ b_0 = s_1 = 0 $:
	sind nämlich $ b_0 = \sum_{b\in B_0} bx_b $ und $ s_1 = \sum_{b\in S_1}bx_b $, so gilt 
		\begin{equation*}
		0 = b_0+s_1 = \sum_{b\in B_0} bx_b+\sum_{b\in S_1} bx_b = \sum_{b\in B_1} bx_b \Rightarrow \forall b\in B_1: x_b = 0\Rightarrow b_0 = s_1 = 0.
		\end{equation*}

\paragraph{Bemerkung: }
	Ist $ U_1\cap U_2 = \{o\} $ bzw. $ \dim (U_1\cap U_2)  = 0 $, so zeigt der Beweis auch:
		\begin{equation*}
		\forall v\in U_1+U_2\exists ! u_1 \in U_1\exists ! u_2\in U_2: v= u_1+u_2
		\end{equation*}

\paragraph{Definition: }
	Zwei UVR $ U_1,U_2 \subset V $ heißen komplementär in $ V $, falls
		\begin{equation*}
		U_1+U_2=V\text{ und } U_1\cap U_2 = \{o\}.
		\end{equation*}
		
\paragraph{Lemma: }
	Zu jedem UVR $ U\subset V $ existiert ein (in $ V $) komplementärer UVR.
	
\paragraph{Beweis: }
	Sei $ U\subset V $ UVR eines $ K $-VR $ V $.
	Seien 
		\begin{itemize}
		\item $ B\subset U $ eine Basis von $ U $;
		\item $ S\subset V $ lin. unahb., sodass $ C=B\cup S $ Basis von $ V $ ist (BES).
		\end{itemize}
	
	Definiere $ U':= [S] $. Dann ist $ U'\subset V $ UVR mit
		\begin{enumerate}[(i)]
		% TODO: wir zeigen, dass U+U' übermenge ist, wollen aber Gleichheit zeigen ... wo ist die andere inklusion?
		\item $ U+U' \supset [C] = V $, da $ C\subset U\cup U' $ Erz. Syst. von $ V $ ist;
		\item $ U\cap U' = [B]\cap [S] = \{o\}$, da $ C=B\cup S $ linear unabhängig ist.
		\end{enumerate}
\paragraph{Bemerkung: }
	Zu einem UVR $ U\subset V $ gibt es normalerweise viele komplementäre UVR $ U'\subset V $.
	Z.B.: Zu
		\begin{equation*}
		U:= \{v\in K^2\mid v_2 = 0\}
		\end{equation*}
	
	ist jeder UVR $ U' = [u']$ mit $u'_2\neq 0 $ komplementär in $ K^2 $.
	
\paragraph{Lemma \& Definition: }
	Sei $ U= \sum_{i\in I}U_i\subset V $ Summe einer Familie von UVR $ U_i\in V $; dann besitzt jeder Vektor $ u\in U $ eine eindeutige Zerlegung als Summe von $ u_i $, genau dann, wenn
		\begin{equation*}
		\forall i\in I: U_i\cap \sum_{j\in I\setminus \{i\}}U_j = \{o\}.
		\end{equation*}
		
	In diesem Falle heißt die Summe "`direkt"' und man schreibt
	\[ U = \bigoplus_{i\in I} U_i. \]
		
\paragraph{Bemerkung: }
	Eine Summe $ V = \sum_{i\in I} U_i $ ist genau dann direkt, wenn
		\begin{equation*}
		\forall i\in I: U_i, \sum_{j\in I\setminus\{i\}}U_j \subset V
		\end{equation*}
		
	komplementäre UVR in $ V $ sind.

\paragraph{Beweis: }
	Zu zeigen ist die Eindeutigkeitsaussage. Sei also $ u \in \bigoplus_{i\in I}U_i $,
		\begin{gather*}
		u = \sum_{i\in I} u_i = \sum_{i\in I} u_i' \text{ mit } \forall i\in I: n_i,n'_i\in U_i;
		\end{gather*}
	dann gilt für jedes $ i\in I$:
		\begin{equation*}
		u_i-u'_i = \sum _{j\neq i}u_j-\sum_{j\neq i} u'_j = \sum_{j\neq i}u_j-u'_j \in U_i\cap \sum_{j\neq i} U_j = \{o\},
		\end{equation*}
	da die Summe als direkt angenommen wurde; damit folgt $ \forall i \in I: u_i = u'_i $, d.h. die Zerlegung ist eindeutig.
	
	Die Umkehrung ist trivial:
		\begin{gather*}
		\exists i\in I:U_i\cap \sum_{j\neq i} U_j \neq \{o\} \Rightarrow \exists i\in I\exists u_i\in U_i\setminus\{o\}\exists (u_j)_{j\in I\setminus\{i\}}:\\
		(\forall j\in I\setminus\{i\}:u_j \in U_j)\land u_i = \sum_{j\neq i} u_j,
		\end{gather*}
		
	d.h., die Zerlegung von $ u_i\in \sum_{i\in I}U_i $ ist nicht eindeutig.

\paragraph{Bemerkung: }
	Sind $ \dim V <\infty $ und $ \# I < \infty $ so gilt
		\begin{equation*}
		\forall i\in I: \dim U_i < \infty
		\end{equation*}
		
	und es gilt die Dimensionsformel für direkte Summen (Beweis in Aufgabe 35):
		\begin{equation*}
		\dim \bigoplus_{i\in I}U_i = \sum_{i\in I} \dim U_i.
		\end{equation*}
	
	Ist insbesondere $ B=(b_1,...,b_n) $ eine Basis von $ V $, so gilt
		\begin{equation*}
		\dim V = \dim \bigoplus_{i=1}^n [b_i]=\sum_{i=1}^{n}1 = n.
		\end{equation*}
\paragraph{Bemerkung: }
	Seien $ U,U'\subset V $ komplementäre UVR, also $ V = U \oplus U' $, dann werden durch
		\begin{equation*}
		v = u+u' \mapsto
			\begin{cases}
				p(v):=u\\
				p':= u'
			\end{cases}
		\end{equation*}
	
	Endomorphismen $ p,p'\in \operatorname{End}(V) $ (wohl-)definiert, da $ n,n' $ durch $ v $ eindeutig bestimmt sind (Linearität von $ p,p' $ ist klar).
	Offenbar ist 
		\[ p(V) = U \text{ und } \ker p = U'\]
	und es gilt
		\[ p^2 := p\circ p = p \]
	und analog für $ p' $; außerdem gilt ($ \circ $ ausgelassen)
		\[ p+p' = id_V \text{ und } p'p = 0 = pp'.\]
		
\paragraph{Definition: }
	$ p\in \End(V) $ heißt Projektion, falls $ p^2 = p $ (d.h. falls $ p $ idempotent ist).
	
\paragraph{Satz: }
	Sei $ p\in \End(V) $ Projektion, dann ist $ p'= id_V-p $ Projektion mit $ pp' = p'p = 0 $. Gilt $ p+p' = id_V $ und $ pp' = 0 $ für $ p,p' \in \End(V) $, so sind $ p,p' $ Projektionen mit
		\[ V = p(V)\oplus p'(V) = \ker p \oplus \ker p'. \]
		
\paragraph{Beweis: }
	Seien $p\in \operatorname{End}(V)$ Projektion und $p' := id_V -p$; dann gilt:
		\[p\circ p' = p(id_V-p)=p-p^{2} = 0; (p^{2} = p\circ p)\]
		\[p\circ p = (\id_V-p)\circ p = p - p^{2} = 0\]
	und
		\[p' \circ p' = p'^2 = p' \circ(\id_V-p)=p'-p'\circ p = p'',\]
	d.h., $p'\in\operatorname{End}(V)$ ist Projektion.
	Anderseits: Seien $p,p' = id_V \text{ und } p' p = 0$.
		
	Dann gilt:
		\[p-p^2 = p(id_V-p) = pp' = 0\]
	d.h., $p\in\operatorname{End}(V)$ ist Projektion, damit ist auch $p'$ Projektion (erster Teil) und $p' p = 0 $. Weiters liefert
		\[\forall v\in V: v=id_v(v) = p(v) + p'(v) \Rightarrow V = p(V)+p'(V),\]

	und ist $w = p(V)= p'(v')$ für geeignete $v,v'\in V$ (d.h., $w  = p(V)\cap p'(V)$), so gilt
		\[ w = p(v) = p^2(v) = p(p(v)) = p(w) = p(p'(v')) = pp'(v') = 0 (pp' = 0), \]
	also $p(V)\cap p'(V) = {0}$ und damit $V = p(V)\oplus p'(V)$. Weiters gilt
		\[0 = p \circ p' \rightarrow p'(V)\subset \ker p\]
	und ist $v\in \ker p$, so folgt
		\[v = p(v) + p'(v) = 0 + p'(v)\in p'(V) \Rightarrow \ker p \subset p'(V).\]
	Für $p'$ gilt das Gleiche und wir haben $\ker p = p'(V)$ und $\ker p' = p(V)$.
	Damit folgt die letzte Behauptung 
		\[V = \ker p \oplus\ker p'.\]
	
\paragraph{Bemerkung: }
		Im Beweis haben wir etwas mehr bewiesen als behauptet - nämlich:
			\[ ker p = p'(V)\text{ und }\ker p' = p(V) \]
		
\paragraph{Beispiel und Definition: }	
		Sei $s\in \operatorname{End}(V)$ eine Involution d.h., 
			\[ s^2 = \id_v \] 
		und 
			\[ p_\pm := 1/2(\id_v\pm s). \]
		Offenbar gilt dann
			\[ p_{+} + p_{-} = id_V \] 
		und 
			\[ p + p_{-} = 1/4(\id_V +s)(\id_V -s) = 1/4(id_v^2-s^2)=0 \]
		also (Satz) sind $p_\pm\in\End(V)$ Projektionen mit komplementären Bildern bzw. Kernen.
		
\paragraph{Lemma und Definition: }
		Ist $(V_i)_{i\in I}$ eine Familie von K-VR $V_i$, so wird das (mengenthoretische) Produkt:
			\[V:= \prod_{i\in I}V_i=\{(v_i)_{i\in I}\mid\forall i\in I:v_i\in V_i\}\]
		mit den komponentenweise definierten VR-Operationen zu einem $ K $-VR. Dies ist der Produktraum der Familie	$(V_i)_{i\in I}$.
		
\paragraph{Beweis: } Aufgabe!

\paragraph{Bemerkung: } 
		Ist $V = \prod_{i\in I} V_i$ ein Produktraum, so erhält man einen kanonischen UVR
			\[U_i:=\{v=(v_i)_{i\in I}\in V\mid\forall j = i:v_j = 0\}\subset V,\]
		die isomorph zu $\dim(V_i)$ sind der Faktorprojektionen,
			\[\prod_i:V\to V_i:(v_j)_{j\in I} \mapsto v_i,\]
		mittels Faktor-Injektionen
			\[\iota_i:V_i\to V: v_i \mapsto(v_j)_{j\in I},\]
		wobei
			\[v_j :=
				\begin{cases}
					v_i \text{ falls } j=i\\
					0 \text{ sonst}
				\end{cases}
			\]
		Ist dann $\# I < \infty$, so erhält man
			\[\prod_{i\in I} V_i\cong \bigoplus_{i\in I}U_i (=: \bigoplus_{i\in I}V_i);\]
		ist $\#I=\infty$, so ist diese Identifikation im Allgemeinen falsch!
	
\paragraph{Beispiel: }
	Für einen Körper K liefert das n-fache Produkt den Standardraum
		\[\prod_{i=1}^{n}K = \{(x_i)_{i = 1,...,n}\mid\forall i \in \{1,...,n\}: x_i \in K\} = K^n \cong \oplus_{i=1}^n\{(x_i)_{i\in {1,...,n}}\mid\forall j\neq i: x_j = 0\};\]
	für den Raum der K-wertigen Folgen ist jedoch
		\[\prod_{i\in \mathbb{N}}K=K^{\mathbb{N}}\neq\oplus_{i\in \mathbb{N}}\{(x_i)_{i\in \mathbb{N}}\mid\forall j\neq i: x_j=0\}.\]
			
\paragraph{Lemma und Definition: }
	Sei $U\subset V$ UVR. Die Menge der Nebenklassen 
		\[V/U := \{v+U\mid v\in V\},\]
	wobei
		\[v+U:=\{v+u\mid u\in U\}\]
	die Nebenklasse zu $v\in V$ bezeichnet, wird mit den durch
		\[(v+U)+(w*U):=(v+w)+U\]
	und
		\[(V+U)x := Vx + U\]
	definierten Operationen ein Vektorraum: der Quotientenraum $V/u$.
			
\paragraph{Beweisen: }
	Zu zeigen: Wohldefiniertheit der Operationen und VR-Axiome (werden übergangen).
	
	Wohldefiniertheit der Skalarmultiplikation: Ist $x\in K$ und sind $(v+U),(v'+U)\in V/U$ gleich, $v+U = v' + U$, so gilt
		\[v+U = v'+U \Leftrightarrow v = v'\in U\] \[\Rightarrow (v-v')x \in U\]
		\[\Rightarrow vx+U=v' x+U\]
	Das Resultat der Skalarmultiplikation bringt also nicht von dem Repräsentanten $v$ einer Nebenklasse $v+U$ ab, sondern nur von der Nebenklasse.
	
	Die Wohldefiniertheit der Addition ist analog.

\end{document}
